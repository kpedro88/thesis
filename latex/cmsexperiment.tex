\chapter{Compact Muon Solenoid Experiment
\label{ch:cmsexperiment}}
\setcounter{section}{-1}

The Compact Muon Solenoid experiment is one of two general-purpose detectors at the LHC. It is located about 100\unit{m} underground at Point 5 of the LHC ring, near Cessy, France. The detector is cylindrically shaped, with total length 22\unit{m}, diameter 15\unit{m}, and weight 14000\unit{tons}. Figure \ref{fig:cms-overall} shows the overall layout of the detector.

\begin{figure}[hbt]
\begin{center}
\includegraphics[width=0.95\textwidth]{figures/cms_complete_labelled.pdf}
\caption{The layout of the CMS detector, with the subdetectors labeled and two humans shown for a height reference.}
\label{fig:cms-overall}
\end{center}
\end{figure}

The center of the detector is used as the origin of the right-handed coordinate system that describes locations and directions within the detector. The $z$-axis is assigned to the direction of the LHC beam line, pointing anti-clockwise in the direction of Beam 2. The polar angle $\theta$ is defined as the angle away from the positive $z$-axis. This angle is often transformed into pseudorapidity, defined as $\eta = -\text{ln}[\text{tan}(\theta/2)]$, which has several desirable properties. It is independent of the particle mass and energy, and it is approximately equal to the rapidity for relativistic particles. Differences in pseudorapidity are Lorentz invariant for boosts in the $z$ direction, and particle production is approximately uniform in $\eta$.

The plane transverse to the $z$-axis comprises the $x$- and $y$-axes, with the $x$-axis pointing toward the center of the LHC ring and the $y$-axis pointing upward in the normal direction. The azimuthal angle $\phi$ is defined as the angle away from the positive $x$-axis in the transverse plane, and the radial coordinate $r$ is defined as the distance from the origin in the transverse plane. The magnitude of the component of momentum in the transverse plane is labeled \pt. The total transverse momentum of every event must be conserved, so the negative vector sum of \vecpt for all particles in the event is defined as the missing transverse momentum: $\vecmet = - \sum_{i} \vecpt^{(i)}$. The magnitude of this quantity is called the missing transverse energy, denoted as \met.

As a general purpose detector, the CMS experiment detects all long-lived SM particles, except neutrinos, which can only be measured by omission in the transverse plane via \met. These particles can be categorized as electrons, photons, muons, charged hadrons, and neutral hadrons. The latter two categories are usually found grouped into cones called jets, and can originate from gluons, light quarks, bottom quarks with displaced vertices, or hadronically decaying tau leptons. The identification of particles and related objects with the CMS detector is described in more detail in Ch. \ref{ch:reconstruction}.

The CMS detector must measure these particles with sufficient accuracy to accomplish the experiment's physics goals, imposing requirements which are met by the combination of the CMS subdetectors. The inner tracker measures event vertices and charged-particle momentum, with the help of the superconducting solenoid. The measurement of muon momentum is supplemented by the muon system. The electromagnetic calorimeter measures electron and photon energy, and the hadron calorimeter measures the energy from charged and neutral hadrons. In addition, the LHC operates at a high instantaneous luminosity, approaching $10^{34}\percms$; with an expected proton-proton cross section of 100\unit{mb} at the LHC center-of-mass energies, the collision rate is approximately 1\unit{MHz}. To cope with this incredibly high collision rate, the CMS experiment employs a trigger system to select interesting events at a rate which can be stored and processed by the computing systems. The delivered luminosity is also measured by the detector, using special techniques. The following sections describe the LHC, based on Ref. \cite{LHCmachine}, and the CMS subdetector systems, based on Ref. \cite{CMSJINST}.

\section{The Large Hadron Collider}

The Large Hadron Collider is the largest machine ever built and the highest-energy collider in the world. It uses the tunnel originally constructed for the Large Electron-Positron Collider (LEP) with a circumference of 26.7\unit{km}. The tunnel is located underground in Switzerland and France, near Geneva. Figure \ref{fig:lhc-diagram} shows a diagram of the LHC with the major experiments labeled. Opposite from CMS is A Toroidal LHC ApparatuS (ATLAS) at Point 1, the other general purpose detector. To the right of ATLAS at Point 8 is the LHC beauty (LHCb) experiment, which studies flavor physics. To the left of ATLAS at Point 2 is A Large Ion Collider Experiment (ALICE), which studies heavy ion physics in Pb-Pb and p-Pb collisions.

\begin{figure}[hbt]
\begin{center}
\includegraphics[width=0.95\textwidth]{figures/lhc-pho-1997-060.png}
\caption{A diagram of the LHC with the major experiments labeled \cite{Jean-Luc:841573}.}
\label{fig:lhc-diagram}
\end{center}
\end{figure}

The LHC is designed to accelerate two beams of protons up to energies of 7\TeV each, at instantaneous luminosities up to $10^{34}\percms$. It is also designed to accelerate lead ions up to energies of 1.38\TeV per nucleon, at instantaneous luminosities up to $10^{27}\percms$. The LHC can achieve energies several orders of magnitude higher than LEP in the same tunnel by using protons instead of electrons; the larger mass of protons reduces losses from synchrotron radiation by $(m_{\Pp}/m_{\Pe})^{4} = 1836^{4} = 1.136\times10^{13}$. The use of supercooled superconducting magnets, discussed below, is also crucial. Several stages of CERN accelerators are used to inject proton beams into the LHC, as shown in Fig. \ref{fig:lhc-injectors}. These include the linear accelerator Linac2, the Proton Synchrotron Booster (PSB), the Proton Synchrotron (PS), and the Super Proton Synchrotron (SPS). For lead ions, Linac3 and the Low Energy Ion Ring (LEIR), a repurposing of the Low Energy Antiproton Ring (LEAR), are used instead of Linac2 and the PSB, respectively. The accelerated protons are grouped into bunches using radio frequency (RF) electromagnetic fields. The LHC is designed to accommodate a bunch spacing of 25\unit{ns}.

\begin{figure}[hbt]
\begin{center}
\includegraphics[width=0.95\textwidth]{figures/lhc-pho-1993-008.png}
\caption{A diagram of the CERN accelerators which form the LHC injector \cite{Jean-Luc:841568}.}
\label{fig:lhc-injectors}
\end{center}
\end{figure}

Due to size limitations in the tunnel, the two rings used to accelerate the two proton beams are formed by twin bore magnets. Each magnet has a single mechanical structure and cryostat, in which are placed two coils and two beam channels. The dipole magnet coils use superconducting NbTi Rutherford cables cooled to 1.9\unit{K}, as shown in Fig. \ref{fig:lhc-dipole}, with a design field strength of 8.33\unit{T} for acceleration of protons up to 7\TeV. This extreme cooling is accomplished using superfluid helium. In total, the LHC contains 1232 dipole magnets. Thousands of quadrupole, sextupole, octupole, and decapole magnets are used to correct and focus the beam.

\begin{figure}[hbt]
\begin{center}
\includegraphics[width=0.95\textwidth]{figures/lhc-pho-1998-299.jpg}
\caption{A diagram of an LHC dipole magnet, with the major components labeled \cite{Dailler:842253}.}
\label{fig:lhc-dipole}
\end{center}
\end{figure}

In 2012, the LHC accelerated proton beams to energies of 4\TeV each, with a peak instantaneous luminosity of $7.67\times10^{33}\percms$ and a bunch spacing of 50\unit{ns}. During that year, it delivered 23.30\fbinv of integrated luminosity to the CMS detector, of which 21.79\fbinv was recorded. In the upcoming 2015 run, the LHC will achieve its design energy, instantaneous luminosity, and bunch spacing.

\section{Tracker}

The CMS tracker is the first subdetector to measure charged particles produced in collisions at the interaction point. It is 5.8\unit{m} long and 2.5\unit{m} in diameter, covering the pseudorapidity range $-2.5 < \eta < 2.5$. Two subsystems make up the tracker: the pixel detector and the silicon strip tracker. The layout of the tracker, with these subsystems labeled, is shown in Fig. \ref{fig:tk-layout}. Due to the tracker's location close to the interaction point, it experiences severe radiation doses that are expected to range from 0.18 to 84\unit{Mrad} after 500\fbinv of data. Hence, the tracker must be robust against radiation damage, requiring operation at $-10\degC$ and influencing the design of the sensors and electronics. For tracks with momentum of 100\GeV, the tracker has a transverse momentum resolution of 1--2\% for $|\eta|<1.6$; at higher $\eta$, the reduced transverse depth of the tracker degrades the resolution.

\begin{figure}[hbt]
\begin{center}
\includegraphics[width=0.95\textwidth]{figures/CMS_tracker.pdf}
\caption{The layout of the CMS tracker, with subsystems labeled.}
\label{fig:tk-layout}
\end{center}
\end{figure}

The pixel detector is the innermost portion of the tracker. It consists of three barrel layers, collectively called BPIX, and two endcap layers, called FPIX. Each pixel cell is a hybrid silicon detector with dimensions $100\times150\mum^{2}$. The small pixel size enables precise track resolutions of 10\mum in the $r-\phi$ direction and 20\mum in the $z$ direction. In total, the BPIX comprises 48 million pixels and the FPIX comprises 18 million pixels. The pixel detector is important for many key components of CMS physics analysis. These include the reconstruction of secondary vertices from decays of tau leptons and bottom quarks, as well as producing seed tracks for the strip tracker and the high level trigger.

The silicon strip detector consists of four subsystems. The Tracker Inner Barrel (TIB) has four layers with the three-layer Tracker Inner Disks (TID) at each end; both systems' strips are 320\mum thick. Surrounding the TIB/TID is the Tracker Outer Barrel (TOB), which has six layers. The first four layers of the TOB use 500\mum thick strips, while the last two layers use 122\mum thick strips. The Tracker EndCaps (TEC) have nine disks with up to seven layers of strips, 320\mum thick in the inner four rings and 500\mum thick in the outer three rings. In total, all of these layers contain 9.3 million silicon strips.

The tracker maintained excellent performance during the 2012 run of the LHC. The pixel detector had 97.7\% of channels operational in BPIX and 92.8\% of channels operational in FPIX, while 97.5\% of channels in the strip detector were active. The hit reconstruction efficiencies were greater than 99\% for each layer of the strip detector and greater than 99.5\% for each layer of the pixel except for the first layer of BPIX, which had an efficiency greater than 99.2\% \cite{Veszpremi:2014hpa}. 

\section{Electromagnetic Calorimeter}

The electromagnetic calorimeter (ECAL) is a homogeneous calorimeter constructed entirely of lead tungstate (\pbwo) crystals. The ECAL is divided into two subsystems: the ECAL barrel (EB) and the ECAL endcap (EE). In the endcap region, there is an additional ECAL preshower (ES) detector in front of the EE. Figure \ref{fig:ecal-layout} displays these subsystems. \pbwo has a peak emission wavelength of 425\unit{nm} and many desirable material properties. These properties include high density ($8.28\unit{g/cm}^3$), short radiation length (0.89\cm), short Moli\`{e}re radius (2.2\cm), and fast decay time (6\unit{ns}). The use of homogeneous \pbwo crystals enables precise energy resolution for electromagnetic objects. For photons with $\ET \approx 60\GeV$, the energy resolution ranges from 1.1\% to 2.6\% for the EB and 2.2\% to 5.0\% for the EE. In general, the energy resolution $\sigma$ varies as a function of energy $E$ in \GeVns:
\begin{equation}
\label{eq:ecal-res} \left(\frac{\sigma}{E}\right)^{2} = \left(\frac{S}{\sqrt{E}}\right)^{2} + \left(\frac{N}{E}\right)^{2} + C^{2}.
\end{equation}
In Eq. \eqref{eq:ecal-res}, $S$ is the stochastic term, $N$ is the noise term, and $C$ is the constant term. Typical values for these terms were measured by a test beam to be $S=2.8\%$, $N=12\%$, $C=0.30\%$.

\begin{figure}[hbt]
\begin{center}
\includegraphics[width=0.95\textwidth]{figures/ECAL_transverse_section.pdf}
\caption{A diagram of the CMS ECAL, with subsystems and $\eta$ ranges labeled.}
\label{fig:ecal-layout}
\end{center}
\end{figure}

The EB contains 61200 \pbwo crystals and covers the range $|\eta|<1.479$. The crystals are arranged in a projective geometry with a tapered shape. The crystal granularity is approximately $0.0174\times0.0174$ in $\eta-\phi$, corresponding to dimensions of $22\times22\mm^{2}$ at the front face and $26\times26\mm^{2}$ at the back face. The EB has a depth of 230\mm or 25.8 radiation lengths ($X_{0}$). The scintillation light produced by the \pbwo crystals is read out using avalanche photodiodes (APDs). At 18\degC, the APDs measure approximately 4.5 photoelectrons per \MeVns. The dark current of the APDs is sensitive to radiation exposure. Over the course of the 2012 run, the dark current ranged from 0.3--1.3\muA on average, corresponding to an average noise of 47--57\MeV \cite{CMS:2013ecal}.

The EE contains 14648 \pbwo crystals and covers the range $1.479<|\eta|<3.0$. The crystals are arranged in a non-projective $x-y$ geometry, with dimensions of $28.62\times28.62\mm^{2}$ at the front face and $30\times30\mm^{2}$ at the back face. The EE has a depth of 220\mm or 24.7$\,X_{0}$. Vacuum phototriodes (VPTs) are used as the photodetectors to read out the scintillation light from the \pbwo crystals. They collect approximately 4.5 photoelectrons per \MeVns at 18\degC. During the 2012 run, the average noise ranged from 180--220\MeV, with a more dramatic increase up to 600\MeV at high $\eta$ because of the high radiation dose \cite{CMS:2013ecal}.

The ES is intended to identify neutral pions in the endcap region, covering the range $1.653<|\eta|<2.6$. It is a sampling calorimeter with two layers of lead absorber and silicon strip detectors. The first layer of lead absorber has a thickness of 2$\,X_{0}$, while the second layer has a thickness of 1$\,X_{0}$. Each layer of silicon strips is 320\mum thick and can collect 3.6\unit{fC} of charge from a minimum ionizing particle.

%add percentage of live channels for EB and EE in 2012 run?

\section{Hadron Calorimeter}

\section{Solenoid}

\section{Muon System}

\section{Trigger}

\section{Luminosity Measurement}