\chapter{Introduction
\label{ch:introduction}}

Just over a century ago, Ernest Rutherford and Niels Bohr developed a revolutionary model of the atom: a nucleus containing positively charged protons and electrically neutral neutrons, surrounded by negatively charged electrons in quantized orbitals. This model and other related models, along with the empirical results which motivated them, led to the development of quantum mechanics to describe physics at the subatomic level. As physicists investigated further, accelerating particles closer and closer to the speed of light in order to reach higher energies and smaller distances, it became necessary to merge the formulations of quantum mechanics with Einstein's special relativity to develop quantum field theories. The results of decades of these experimental tests are collected in the unified framework of the standard model of particle physics. The standard model describes all observed elementary particles and three of the fundamental forces, including electromagnetism and the weak and strong nuclear interactions. The weak nuclear interaction is responsible for most radioactivity and for nuclear fusion, while the strong force holds quarks and gluons together as protons and neutrons, and residually holds protons and neutrons together as atomic nuclei.

The last missing piece of the standard model, the Higgs boson, was discovered in 2012 by the Compact Muon Solenoid experiment at the Large Hadron Collider. This collider and the several detectors placed around it are the largest scientific experiment yet undertaken by humanity. Two beams of protons are each accelerated to 99.999997\% of the speed of light, achieving an energy level never before produced in the laboratory, and then collided together. The constituent quarks and gluons of the protons interact with each other through the fundamental forces, producing any of the particles in the standard model. Just as a radioactive element will decay into a different element by emitting radiation, the produced particles decay into the lightest stable particles, which are detected by the Compact Muon Solenoid. Each subsystem of the detector is optimized to measure certain types of particles, including photons, electrons, muons, charged hadrons such as protons, and neutral hadrons such as neutrons. The presence of non-interacting particles such as neutrinos can be inferred by balancing the momentum measured by the detector in each collision event. The unprecedented energy scale and collision rate of these experiments enabled the discovery of the Higgs boson, the first observed scalar elementary particle. The Higgs boson is an excitation of the Higgs field, which was postulated and is now confirmed to provide masses to the various elementary particles.

However, there are some phenomena which remain beyond the capability of the standard model to explain. This dissertation presents a search for several types of new particles, leptoquarks and top squarks, which are predicted by theories that might supersede the standard model. Leptoquarks are bosons which have properties of both leptons (such as electrons) and quarks (such as the constituents of protons and neutrons). The existence of leptoquarks could indicate new relationships between the leptons and quarks in the standard model. James Maxwell unified electricity and magnetism into electromagnetism, and similarly, Abdus Salam, Sheldon Glashow, and Steven Weinberg unified electromagnetism and the weak nuclear interaction into the electroweak interaction. Leptoquarks could point the way to possible solutions for the grand unification of all three fundamental forces. The theory of supersymmetry postulates a partner for each standard model particle, such as the top squark, which is the supersymmetric partner of the top quark, a heavy version of the up quarks contained in protons and neutrons. As the heaviest elementary particle, the top quark is the most likely to interact with the Higgs boson. The discovery of its supersymmetric partner the top squark would indicate how that interaction is balanced to produce the relatively light Higgs boson that has been observed.

In this dissertation, a search is conducted for scalar leptoquarks coupling to the third generation of elementary particles, including tau leptons and bottom quarks. The search for top squarks considers R-parity violating supersymmetry, in order to get around experimental constraints on R-parity conserving supersymmetry. The R-parity violating model eliminates a proposed separation between standard model and supersymmetric particles, which allows the supersymmetric particles to decay to final states involving only standard model particles. There are two cases considered for top squarks. In the first case, the top squarks decay identically to third-generation scalar leptoquarks. In the second case, the top squarks decay differently, producing a final state similar to the first case: a tau lepton, a bottom quark, and two light quarks. The existence of third-generation scalar leptoquarks and the first case of top squarks is already excluded if their mass is below values probed by earlier searches, but the high energy of the 2012 run of the Large Hadron Collider will extend the experimental reach of this search. Previously, no direct search for the second case of top squarks had ever been performed.