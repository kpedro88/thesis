%Acknowledgments

\renewcommand{\baselinestretch}{2}
\small\normalsize
\hbox{\ }
 
\vspace{-.65in}

\begin{center}
\large{Acknowledgments} 
\end{center} 

\vspace{1ex}

In today's world of massive experiments and collaborations, every physicist owes his results in part to the hard work of many others. For me, the first among those is my advisor, Professor Sarah Eno. Her advice and instruction have been invaluable, not only for improving my understanding of the intricacies of high energy physics, but also for navigating the communities and bureaucracies of the University of Maryland, the CMS collaboration, Fermilab, and CERN. I have been able to excel in the CMS collaboration in large part because of Sarah's guidance and suggestions when choosing projects and solving problems.

I am grateful to my analysis partners Matthieu Marionneau and Ketino Kaadze for their invaluable contributions to the analysis presented in this dissertation. I learned a great deal from both of them about data analysis and high energy physics. It was a superb experience to be part of a group in which everyone treated each other as equals and everyone took responsibility for their share of the work. I must also thank Youngdo Oh and Anirban Saha for generating the top squark signal samples, and the various members of the Exotica group for their many helpful comments.

I have learned and grown as a physicist while at the University of Maryland, thanks to the encouragement and support of the CMS group here. My gratitude goes to Professors Alberto Belloni, Nick Hadley, Drew Baden, Andris Skuja, and Tom Ferbel; postdocs Matthieu Marionneau, Ted Kolberg, Jeff Temple, and Ellie Twedt; scientist Dick Kellogg; and engineer Tom O'Bannon. I may never recapture the magic of this group of CMS grad students: Brian Calvert, Young Ho Shin, and Chris Anelli. I also thank Marguerite Tonjes, Brian Calvert, and Chris Ferraioli for keeping our Tier 3 cluster running, and former grad student Dinko Feren\v{c}ek for his help in preparing this dissertation. I am grateful to my dissertation committee members for volunteering their time and effort to help me improve this dissertation.

This section is too narrow to contain the full list of CMS collaborators who have aided me on the path to my doctoral degree. I have been privileged to work on many areas of CMS, including trigger operations, HCAL, jets, long-term upgrades, fast simulation, and offline software. My thanks go to all of my CMS colleagues from these groups for their help and support. I also thank the many professors, teaching assistants, and administrators in the University of Maryland physics department for their help with the courses and logistics of graduate student life.

Socially and personally, graduate school has been an unexpected delight for me. This is largely due to my housemates in Physics House, official and honorary, old and new: Nat Steinsultz, Joe Garrett, George Hine, Kiersten Ruisard, Meredith Lukow, Ginny Garrett, Neil Anderson, and Lexi Parsagian. I will dearly miss all of them when I leave Maryland. I extend that gratitude to the many friends I have made at both Maryland and CERN during my graduate career.

Finally, I thank my family -- Pedros, Gravels, Mougalians, Wrights -- for the myriad ways they have supported me during my time as a graduate student. This dissertation is dedicated to my father Philip, from whom I inherited a love of science and technology, and to the memory of my mother Lisa, whose love and support will remain with me throughout my life.