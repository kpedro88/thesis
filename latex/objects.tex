\subsection{Object Identification}

\subsubsection{Muons
\label{sec:muon-obj}}

Two working points for the quality cuts described in Sec. \ref{sec:muon-reco} are used to identify reconstructed muons. Table \ref{tab:muonWP} lists the various requirements in each working point. The leading muon in the \mutau channel is selected with the tight working point; in addition to those quality cuts, it is required to have $\pt>30\GeV$ and $|\eta|<2.1$ to match the specifications of the HLT criterion used to collect the primary dataset. The loose working point is used to veto additional muons and for selections in some control regions. Less restrictive kinematic cuts $\pt>20\GeV$ and $|\eta|<2.4$ are applied to muons identified with the loose working point.

\begin{table}[htb]
  \begin{center}
    \begin{tabular}{|l|l|}
\hline
\multicolumn{2}{|c|}{Working Point}\\
\hline
\multicolumn{1}{|c|}{Tight} & \multicolumn{1}{c|}{Loose} \\
\hline
PF muon & PF muon \\
Global muon & Global muon OR tracker muon \\
$\begin{aligned}
I^{\text{PF}}_{\mu}/\pt &< 0.12 \\
d_0 &< 0.2\cm \\
d_z &< 0.5\cm \\
\text{Global track fit } \chi^2/n_{\text{dof}} &< 10 \\
\text{Global track fit } n_{\text{muon segment}} &> 0 \\
n_{\text{hits}}(\text{pixel}) &> 0 \\
n_{\text{layers}}(\text{tracker}) &> 5 \\
n_{\text{stations}}(\text{muon}) &> 1
\end{aligned}$
&
$\begin{aligned}
I^{\text{PF}}_{\mu}/\pt &< 0.3 \\
\\
\\
\\
\\
\\
\\
\\
\end{aligned}$ \\
\hline
    \end{tabular}
    \caption{The quality cuts for the tight and loose working points of the muon identification.}
    \label{tab:muonWP}
  \end{center}
\end{table}

\subsubsection{Electrons
\label{sec:ele-obj}}

Two working points for the quality cuts described in Sec. \ref{sec:ele-reco} are used to identify reconstructed electrons. Electrons are considered to be in the barrel if they have $|\eta|<1.444$ and in the endcap if they have $1.56<|\eta|<2.5$. Table \ref{tab:eleWP} lists the various requirements in each working point. The leading electron in the \etau channel is selected with the medium working point; in addition to those quality cuts, it is required to have $\pt>30\GeV$ to match the specifications of the HLT criterion used to collect the primary dataset. The loose working point is used to veto additional electrons and for selections in some control regions. The less restrictive kinematic cut $\pt>20\GeV$ is applied to electrons identified with the loose working point.

\begin{table}[htb]
  \begin{center}
    \begin{tabular}{|r|c|c|c|c|}
      \hline
      \multirow{3}{*}{Cut Variable} & \multicolumn{4}{|c|}{Cut Value} \\
      \cline{2-5}
                                    & \multicolumn{2}{|c|}{Medium} & \multicolumn{2}{|c|}{Loose} \\
      \cline{2-5}
                                    & Barrel & Endcap & Barrel  & Endcap    \\
      \hline
      $I^{\text{PF}}_{\Pe}/\pt<$    & 0.15     & 0.15     & 0.15      & 0.15      \\      
      $\sigma_{i\eta i\eta}<$       & 0.01     & 0.03     & 0.01      & 0.03    \\ 
      $|\Delta\phi_{\text{in}}|<$   & 0.06     & 0.03     & 0.15      & 0.10    \\ 
      $|\Delta\eta_{\text{in}}|<$   & 0.004    & 0.007    & 0.007     & 0.009    \\ 
      $H/E<$                        & 0.12     & 0.10     & 0.12      & 0.10     \\ 
      $|d_0^{\text{vtx}}|<$         & 0.02     & 0.02     & 0.02      & 0.02     \\              
      $|d_z^{\text{vtx}}|<$         & 0.1      & 0.1      & 0.2       & 0.2      \\              
      $|1/E - 1/p|<$                & 0.05     & 0.05     & 0.05      & 0.05     \\
      \hline
    \end{tabular}
    \caption{The quality cuts for the medium and loose working points of the electron identification. }
    \label{tab:eleWP}
  \end{center}
\end{table}

\subsubsection{Taus
\label{sec:tau-obj}}

Reconstructed hadronically decaying tau leptons are identified using the various discriminators defined in Sec. \ref{sec:hpstau}. Different working points for each discriminator are used in each channel, as listed in Table \ref{tab:tauWP}. In addition to these discriminators, each reconstructed tau lepton is required to have $\pt>30\GeV$ and $|\eta|<2.3$.

\begin{table}[htb]
  \begin{center}
    \begin{tabular}{|c|c|c|}
      \hline
      \multirow{2}{*}{Discriminator} & \multicolumn{2}{|c|}{Working Point} \\
      \cline{2-3}
                                    & \etau & \mutau \\
      \hline
      Isolation                     & Loose (3 hits)   & Loose (3 hits) \\
      anti-\Pe                      & Loose MVA (v3)   & Loose MVA (v3) \\
      anti-$\mu$                    & Loose (v2)       & Tight (v2) \\
      \hline
    \end{tabular}
    \caption{The working points for the different tau discriminators used in the tau lepton identification. }
    \label{tab:tauWP}
  \end{center}
\end{table}

\subsubsection{Jets
\label{sec:jet-obj}}

The loose working point of the PF jet identification algorithm is used to identify reconstructed jets. The requirements for the loose working point are summarized in Table \ref{tab:jetWP}. To eliminate jets from pileup interactions, each selected jet is required to have $\pt>30\GeV$. The cut $|\eta|<2.4$ is also applied, in order to include only jets measured in the best-performing regions of the detector. To select b-tagged jets, the discriminator calculated by the CSV algorithm is required to have a value greater than 0.244. This is the loose working point of the CSV algorithm, which limits the misidentification probability to 10\%.

\begin{table}[htb]
  \begin{center}
    \begin{tabular}{|r|c|}
      \hline
      \multirow{2}{*}{Cut Variable} & Cut Value \\
      \cline{2-2}
                                    & Loose \\
      \hline
      $f_{\text{CH}}>$              & 0.0 \\
      $f_{\text{NH}}<$              & 0.99 \\
      $f_{\gamma}<$                 & 0.99 \\
      $f_{\text{EM}}<$              & 0.99 \\
      $n_{\text{charged}}>$         & 0 \\
      $n_{\text{constituents}}>$    & 1 \\
      \hline
    \end{tabular}
    \caption{The quality cuts for the loose working point of the jet identification. }
    \label{tab:jetWP}
  \end{center}
\end{table}

\subsubsection{Primary Vertices
\label{sec:vtx-obj}}

Several requirements are applied to the reconstructed primary vertices to ensure high quality \cite{CMS-PAS-TRK-10-005}. The number of degrees of freedom $n_{\text{dof}}$, defined in Sec. \ref{sec:tracks}, must be greater than 4. The longitudinal position $z$ of the vertex must obey $|z|<24\mm$, and the transverse position $\rho$ must obey $|\rho|<2\mm$. In addition, the vertex must be reconstructed by a fit to tracks. The other reconstructed objects described above can be assigned to a reconstructed vertex. The closest vertex to the track associated with the object is chosen as its vertex. For electrons, the GSF track is used; for muons, the Best Track is used; and for hadronic tau leptons, the track of the leading charged hadron is used.