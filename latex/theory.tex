\chapter{Theoretical Motivations
\label{ch:theory}}

\setcounter{section}{-1}

\section{The Standard Model}

The standard model (SM) of particle physics includes three fundamental forces and all of the particles of ordinary matter using a quantum field theory framework. When forces are weak enough, quantum field theory calculations use a perturbative method in which the leading order (LO) term is calculated, then the next-to-leading order (NLO) correction is added, then the next-to-next-to-leading order correction (NNLO), and so forth. The three fundamental forces are electromagnetism, the weak force, and the strong force. These forces are carried by spin-1 gauge bosons, while the matter particles consist of quarks and leptons, two categories of spin-1/2 fermions. In addition, the standard model contains a scalar spin-0 boson, the Higgs boson, which is part of the Higgs field that provides masses to certain gauge bosons and fermions. Figure \ref{fig:sm-particles} summarizes the particles in the standard model, including the spin, electric charge, and mass values of each particle.

\begin{figure}[hbt]
\begin{center}
\includegraphics[width=0.95\textwidth]{figures/Standard_Model_of_Elementary_Particles.pdf}
\caption{A table of all the elementary particles in the standard model, with the spin, electric charge, and mass values of each particle \cite{MissMJ}. The faint gray lines indicate which gauge bosons interact with which fermions.}
\label{fig:sm-particles}
\end{center}
\end{figure}

The electromagnetic force is mediated by photons ($\gamma$), spin-1 gauge bosons which have no mass or electric charge $Q$. This force causes interactions between electrically charged particles and has infinite range due to the masslessness of the photon. Quantum electrodynamics (QED) is the quantum field theory description of electromagnetism, which can be represented by a $U(1)$ symmetry group. The weak force is mediated by the massive \Wpm and \Z bosons and can be represented by an $SU(2)$ symmetry group. The weak force acts on particles carrying weak isospin $T$. Weak isospin is a quantum number whose third component $T_3$ is conserved in all interactions and which can be mathematically treated in the same way as angular momentum, though the two quantities are physically distinct. The massiveness of the weak carrier bosons means that the weak force has a limited range, approximately $10^{-18}\unit{m}$. The charged current weak interaction, mediated by the \Wpm bosons, is sensitive to the chirality of fermions; only left-handed fermions and right-handed antifermions participate in this interaction. The neutral current weak interaction, mediated by the \Z boson, is not sensitive to chirality.

As suggested by the inclusion of both forces in the previous paragraph, the electromagnetic and weak forces can be unified to form the electroweak force, represented by the symmetry group $SU(2) \times U(1)$. In this unification, the quantum numbers of electromagnetism and the weak force are related by a new conserved quantum number, weak hypercharge $Y$:
\begin{equation}
Y = 2(Q - T_3).
\end{equation}
The Higgs mechanism is responsible for electroweak symmetry breaking (EWSB). In order for the electroweak theory to be gauge invariant, the gauge bosons must be massless, but the \Wpm and \Z bosons are observed to have mass. The Higgs mechanism solves this dilemma via spontaneous EWSB due to its non-zero vacuum expectation value (VEV). The Higgs field consists of a doublet, with two charged particles and two neutral particles, all scalar bosons. The two charged particles and one of the neutral particles act as Goldstone bosons, combining with the \Wpm and \Z bosons to produce their masses. The remaining neutral particle is the Higgs boson, which was discovered at the LHC in 2012 \cite{NewBoson}.

The strong force, quantum chromodynamics (QCD), is mediated by gluons (\cPg), which like photons are spin-1 gauge bosons without mass or electric charge. However, gluons do possess color charge, the quantum number on which the strong force acts. Color charge is so named because the charge has three possible values, which are labeled red, green, or blue. Because gluons both mediate and participate in the strong interaction, the force between quarks does not decrease as they become spatially separated. The energy in the gluon field between the separated quarks can become large enough to form one or more quark-antiquark pairs. This phenomenon is known as confinement and prevents quarks or gluons from existing in a bare state. Correspondingly, the range of the strong force is limited to $\sim 10^{-15}\unit{m}$. Bound states of quarks and gluons, the only way they have ever been observed in nature, are called hadrons, and the formation of those bound states is called hadronization. States with one quark and one antiquark are mesons, while states with three quarks are baryons. Mesons and baryons are the two allowed types of bound states because they represent color singlets. Complementarily, as quarks get closer together, the strong force between them weakens. This behavior is known as asymptotic freedom; because short distances are equivalent to high energies, the strong interactions of quarks at a high-energy collider like the LHC can be calculated perturbatively. A residual form of the strong force acts on nucleons, protons and neutrons, to form atomic nuclei.

As mentioned, fermions are the particles of matter, which are separated into two groups: quarks and leptons. Quarks have fractional electric charge, weak isospin, and color charge, so they are affected by all three fundamental forces. There are two types of quarks: up-type quarks that have $Q = 2/3$ and down-type quarks that have $Q = -1/3$. Leptons consist of charged leptons and neutrinos. Charged leptons possess electric charge and weak isospin, while neutrinos only possess weak isospin. Three generations exist for each type of particle, with the different particles called flavors. The flavors of up-type quarks are the up, charm, and top quarks; of down-type quarks are the down, strange, and bottom quarks; of charged leptons are the electron, muon, and tau lepton; and of the neutrinos are the electron, muon, and tau neutrinos. The top quark is the heaviest elementary particle and is so heavy that it decays before hadronizing, making it an exception to the rule that bare quarks are not observed. The fermions are arranged into multiplets based on their chirality. The left-handed up- and down-type quarks are grouped together in a doublet $\cPq_L$, as are the left-handed charged leptons and neutrinos in $\ell_L$. The right-handed particles are singlets. It is important to note that right-handed neutrinos, and correspondingly left-handed antineutrinos, do not exist in the standard model. The Higgs field spontaneously provides masses to the quarks and charged leptons through a Yukawa interaction which couples the left- and right-handed versions of each flavor of particle. The quantum numbers of each type of particle are summarized in Table \ref{tab:q-num}, and the interactions among all the particles are illustrated in Fig. \ref{fig:sm-interactions}.

\begin{table}[htb]
  \begin{center}
    \def\arraystretch{2.0} %1 is the default
    \begin{tabular}{|l||l|c|c|c|}
\hline
      & Particle & $Q$ & $T_3$ & $Y$ \\
\hline
\hline
\multirow{3}{*}{Quarks}  & $\cPq_L = \displaystyle\binom{\cPqu}{\cPqd}_L$ & $\displaystyle\binom{2/3}{-1/3}$ & $\displaystyle\binom{1/2}{-1/2}$ & $1/3$  \\
                         & $\cPqu_R$                                      & $2/3$                            & 0                                & $4/3$  \\
                         & $\cPqd_R$                                      & $-1/3$                           & 0                                & $-2/3$ \\
\hline
\multirow{3}{*}{Leptons} & $\ell_L = \displaystyle\binom{\nu}{\Pe}_L$     & $\displaystyle\binom{0}{-1}$     & $\displaystyle\binom{1/2}{-1/2}$ & $-1$   \\
                         & $\Pe_R$                                        & $-1$                             & 0                                & $-2$   \\
\hline
    \end{tabular}
    \caption{The quantum numbers of each category of fermions, based on chirality and particle type: up-type quarks, down-type quarks, charged leptons, and neutrinos. The various flavors of each category, also called the first, second, and third generations of matter, possess the same quantum numbers and differ only in their masses.}
    \label{tab:q-num}
  \end{center}
\end{table}

\begin{figure}[hbt]
\begin{center}
\includegraphics[width=0.95\textwidth]{figures/Elementary_particle_interactions_in_the_Standard_Model.png}
\caption{A diagram illustrating the leading order interactions between particles in the standard model, including self-interactions \cite{Drexler}.}
\label{fig:sm-interactions}
\end{center}
\end{figure}

\section{Leptoquarks}

\section{R-Parity Violating Supersymmetry}