% input: [mainthesis.tex]
%This template was prepared by Dorothea F. Brosius of the 
%Institute for Electronics and Applied Physics, University of Maryland, College Park, MD
%The template was last updated in March 2013
%Thesis Main Page used with thesis.sty based on the
%University of Maryland Electronic Thesis and Dissertation (ETD) Style Guide

\documentclass[12pt]{thesis}  %12pt is larger than 11pt
%\usepackage[pctex32]{graphics}
\usepackage{titlesec}
   \titleformat{\chapter}
      {\normalfont\large}{Chapter \thechapter:}{1em}{}

\usepackage{longtable}
\usepackage{graphicx}
\usepackage[nocompress]{cite}
\usepackage{notoccite}
\usepackage{lscape}
\usepackage{indentfirst}
%\usepackage{latexsym}
\usepackage{multirow}
\usepackage{bigstrut}
%\usepackage{tabls}
\usepackage{wrapfig}
\usepackage{slashbox}
\usepackage{supertabular}
%\usepackage{subeqn}
\usepackage{subfigure}
\usepackage{amsmath}
\usepackage{amssymb}
\usepackage{xspace}
\usepackage{setspace}
\usepackage[pagebackref=true,breaklinks=true,linktocpage=true,pdfstartview=FitH,plainpages=false,pdfpagelabels]{hyperref}
\usepackage{array}

%%%%%%%%%%%%% ptdr definitions %%%%%%%%%%%%%%%%%%%%%
\usepackage{ptdr-definitions}
%%%%%%%%%%%%% Symbols %%%%%%%%%%%%%%%%%%%
% input: [symbols.tex]
%math
%\newcommand\triplet[3]{\left(\genfrac{}{}{0pt}{0}{#1}{\genfrac{}{}{0pt}{0}{#2}{#3}}\right)}
\newcommand\triplet[4][c]{\left(%
        \begin{array}{@{}#1@{}}#2\\#3\\#4%
\end{array}%
\right)%
}
\newcommand\doublet[3][c]{\left(%
        \begin{array}{@{}#1@{}}#2\\#3%
\end{array}%
\right)%
}

\newcommand{\tauh}{\ensuremath{\tau_{\text{h}}}\xspace}
\newcommand{\Pe}{\ensuremath{\cmsSymbolFace{e}}\xspace}
\newcommand{\Pep}{\ensuremath{\cmsSymbolFace{e}^{+}}\xspace}
\newcommand{\Pem}{\ensuremath{\cmsSymbolFace{e}^{-}}\xspace}
\newcommand{\Pp}{\ensuremath{\cmsSymbolFace{p}}\xspace}
\newcommand{\mutau}{\ensuremath{\mu\tauh}\xspace}
\newcommand{\etau}{\ensuremath{\Pe\tauh}\xspace}
%\newcommand{\etau}{\ensuremath{e\tauh}\xspace}
\newcommand{\emu}{\ensuremath{\Pe\mu}\xspace}
\newcommand{\ltau}{\ensuremath{\ell\tauh}\xspace}
\newcommand{\pbwo}{\ensuremath{\text{PbWO}_{4}}\xspace}
\newcommand{\MLQ}{\ensuremath{M_{\text{LQ}}}\xspace}

%units
\newcommand{\muA}{\ensuremath{\,\mu\text{A}}\xspace}
\newcommand{\degC}{\ensuremath{\,\de\text{C}}\xspace}

\def\BAL{\ensuremath{B_{\met}}}
\def\MET{\ensuremath{{E\!\!\!/}_{\mathrm{T}}}}
\def\METV{\ensuremath{\vec{E\!\!\!/}_{\mathrm{T}}}}
\def\QT{\ensuremath{{q}_{\mathrm{T}}}}
\def\QTV{\ensuremath{\vec{q}_{\mathrm{T}}}}
\def\PT{\ensuremath{{p}_{\mathrm{T}}}}
\def\ET{\ensuremath{{E}_{\mathrm{T}}}}
\def\HT{\ensuremath{{H}_{\mathrm{T}}}}


\def\eslash{\ensuremath{{\hbox{$E$\kern-0.6em\lower-.05ex\hbox{/}\kern0.10em}}}}
\def\vecmet{\mbox{$\vec{\eslash}_\text{T}$}\xspace} %missing ET vector
\def\vecet{\mbox{$\vec{E}_\text{T}$}\xspace} % ET vector
\def\vecpt{\mbox{$\vec{p}_\text{T}$}\xspace} % PT vector
\def\met{\mbox{$\eslash_\text{T}$}\xspace} %missing ET, no space
\def\mex{\mbox{$\eslash_x$}\xspace} %missing Ex
\def\mey{\mbox{$\eslash_y$}\xspace} %missing Ey
\def\mepar{\mbox{$\eslash_\parallel$}\xspace}
\def\meperp{\mbox{$\eslash_\perp$}\xspace}
\def\metvec{\mbox{$\vec{\met}$}\xspace}
\def\metvecrec{\mbox{$\vec{\met}^{\rm rec}$}\xspace}
\def\metvecgen{\mbox{$\vec{\met}^{\rm gen}$}\xspace}
\def\metgen{\mbox{$\met^{\rm gen}$}\xspace}
\def\metparl{\mbox{$\mepar^{\rm rec}$}\xspace}
\def\metperp{\mbox{$\meperp^{\rm rec}$}\xspace}
\def\deltamet{\mbox{$\Delta\met$}\xspace}
\def\pthat{\mbox{$\hat{p}_T$}\xspace}
\def\hslash{{\hbox{$H$\kern-0.8em\lower-.05ex\hbox{/}\kern0.10em}}}
\def\MHT{\mbox{$\hslash_\text{T}$}\xspace}
\def\mht{\mbox{$\hslash_\text{T}$}\xspace}
\def\sumet{\mbox{$\sum E_\text{T}$}\xspace}
\def\scalht{\mbox{$H_\text{T}$}\xspace}
\def\etmiss{\eslash_T}
\def\qt{\ensuremath{{q}_{\mathrm{T}}}}
\def\bigeslash{{\hbox{$E$\kern-0.38em\lower-.05ex\hbox{/}\kern0.10em}}}  
\def\bigmet{\mbox{$\bigeslash_T$}\xspace} 
\def\bighslash{{\hbox{$H$\kern-0.6em\lower-.05ex\hbox{/}\kern0.10em}}}  
\def\bigmht{\mbox{$\bighslash_T$}\xspace} 
\def\vecut{\mbox{$\vec{u}_T$}\xspace}

\def\Lum{\mbox{$\mathcal{L}$}\xspace}
\def\Acc{\mbox{$\mathcal{A}$}\xspace}
\def\eps{\mbox{$\epsilon$}\xspace}

%%%%
%%%% Additional definitions by GHM
%%%%
\def\ERROR#1#2{ \ensuremath{ \pm #1\, (\textrm{#2}) }  }
\def\RESA#1#2#3{ \ensuremath{ #1 \ERROR{#2}{#3} } }
\def\RESB#1#2#3#4#5{ \ensuremath{ \RESA{#1}{#2}{#3} \ERROR{#4}{#5} } }
\def\RESC#1#2#3#4#5#6#7{ \ensuremath{ \RESB{#1}{#2}{#3}{#4}{#5} \ERROR{#6}{#7} } }
\def\RESD#1#2{ \ensuremath{ #1 \pm #2 } }
\def\EFF#1#2{ \ensuremath{ ( \RESA{#1}{#2}{stat.} )\% } }
\def\EFFA#1#2{ \ensuremath{ ( {#1} \pm {#2} )\% } }
\def\EFFB#1#2#3{ \ensuremath{ ( \RESA{#1}{#2}{stat.} \ERROR{#3}{syst.} )\% } }
\def\SIGBR#1#2{  \ensuremath{ \sigma \left( \pp \to #1 X \right) \times {\cal{B}} \left( #1 \to #2 \right) } }
\def\SIGBRSHORT#1{\ensuremath{ [\, \sigma\times{\cal{B}}\, ](#1) }}
\def\SIGSHORT#1{\ensuremath{ \sigma(#1) }}
\def\RESE#1#2#3#4{ \ensuremath{ \RESC{#1}{#2}{stat.}{#3}{syst.}{#4}{lumi.} }}
\def\RESF#1#2#3{ \ensuremath{ \RESB{#1}{#2}{stat.}{#3}{syst.} }}
\def\RATWZ#1#2{ \ensuremath{ {
 \frac{ \sigma(\pp\rightarrow \PW X)\times {\cal{B}}(\PW\rightarrow #1)  }
      { \sigma(\pp\rightarrow \PZ X)\times {\cal{B}}(\PZ\rightarrow #2)  }   }  } }
\def\RESRATWZ#1#2#3#4#5{ \ensuremath{ \RATWZ{#1}{#2} &=& 
                                   #3 \ERROR{#4}{stat.} \ERROR{#5}{syst.} } }
\def\RATWW#1#2{ \ensuremath{ {
 \frac{ \sigma(\pp\rightarrow \PWp X)\times {\cal{B}}(\PWp\rightarrow #1)  }
      { \sigma(\pp\rightarrow \PWm X)\times {\cal{B}}(\PWm\rightarrow #2)  }   }  } }
\def\RESRATWW#1#2#3#4#5{ \ensuremath{ \RATWW{#1}{#2} &=& 
                                   #3 \ERROR{#4}{stat.} \ERROR{#5}{syst.} } }
\def\THEORYRATIO#1#2{\ensuremath{ \RESD{#1}{#2} }}
\def\EPS#1{ \ensuremath{ \epsilon_{\textrm{#1}} } }
\def\EPSTNPALL#1{ \ensuremath{ \EPS{TNP-WP{#1}-ALL}  }  }
\def\EPSTNPREC{ \ensuremath{ \EPS{TNP-REC} }  }
\def\EPSTNPTRG{ \ensuremath{ \EPS{TNP-TRG} }  }
\def\EPSTNPWP#1{ \ensuremath{ \EPS{TNP-WP{#1} } }  }
\def\EPSTNPTRGWP#1{ \ensuremath{ \EPS{TNP-TRG{#1}} }  } 
\def\XVL#1#2{ \ensuremath{#1_{#2}} }
\def\NSIG#1{ \ensuremath{\XVL{N}{#1} } }
\def\EPSB#1{ \ensuremath{\XVL{\varepsilon}{#1} } }
\def\RHO#1{ \ensuremath{\XVL{\rho}{#1} } }
\def\AGEN#1{ \ensuremath{\XVL{A}{#1} } }
\def\APRIM#1{ \ensuremath{\XVL{ {F} }{#1} } }
\def\RPM{\ensuremath{R_{\scriptscriptstyle{+/-}} }}
\def\RWZ{\ensuremath{R_{\scriptscriptstyle{\PW/\PZ}} }}


%%%%%%%%%%%%%%%%%%%%%%%%%%%%%%%%%%%%%%%%%%%%%%%%%%%%%%%%%%%%%%%%%%%%%%
\newcommand{\THELUMI} {\ensuremath{{2.88\pm 0.32}~\mathrm{pb}^{-1}}}%


%%%%%%%%%%%%%%%  Additional definitions %%%%%%%%%%%%%%%%%%%%%%%%
%\renewcommand{\PJgy}{\ensuremath{\mathrm{J}\hspace{-.08em}/\hspace{-.14em}\psi}}
\renewcommand{\ttbar}{\ensuremath{\mathrm{t}\overline{\mathrm{t}}}\xspace}
\newcommand{\singletop}{\ensuremath{\mathrm{single-t}}}
\newcommand{\pp}{\ensuremath{{\Pp\Pp}}\xspace}%
\newcommand{\PZ}{\ensuremath{{\mathrm{Z}}}\xspace}%
\newcommand{\PV}{\ensuremath{{\mathrm{V}}}\xspace}%
\newcommand{\rts}{\ensuremath{\sqrt{s}}\xspace}%
\newcommand{\ra}{\ensuremath{\rightarrow}}%


%\def\Zll{\ensuremath{Z \to \ell\ell}\xspace}
%\def\Wmn{\ensuremath{W \to \mu\nu}\xspace}
%\def\Wen{\ensuremath{W \to e\nu}\xspace}
%\def\Wln{\ensuremath{W \to \ell\nu}\xspace}

\newcommand{\MN}{\ensuremath{\Pgm\nu}}%
\newcommand{\MpN}{\ensuremath{\Pgmp\nu}}%
\newcommand{\MmN}{\ensuremath{\Pgmm\overline{\nu}}}%
\newcommand{\EN}{\ensuremath{\Pe\nu}}%
\newcommand{\EpN}{\ensuremath{\Pep\nu}}%
\newcommand{\EmN}{\ensuremath{\Pem\overline{\nu}}}%
\newcommand{\TN}{\ensuremath{\Pgt\nu}}%
\newcommand{\TpN}{\ensuremath{\Pgt^+\nu}}%
\newcommand{\TmN}{\ensuremath{\Pgt^-\overline{\nu}}}%
\newcommand{\LN}{\ensuremath{\ell\nu}}%
\newcommand{\LpN}{\ensuremath{\ell^+\nu}}%
\newcommand{\LmN}{\ensuremath{\ell^-\overline{\nu}}}%

\newcommand{\EpEm}{\ensuremath{\Pep\Pem}}%
\newcommand{\MpMm}{\ensuremath{\Pgmm\Pgmp}}%
\newcommand{\TpTm}{\ensuremath{\tau^+\tau^-}}%
\newcommand{\LpLm}{\ensuremath{\ell^+\ell^-}}%
%\newcommand{\EE}{\ensuremath{\Pe\Pe}}%
%\newcommand{\MM}{\ensuremath{\Pgm\Pgm}}%
%\newcommand{\TT}{\ensuremath{\tau\tau}}%
%\newcommand{\LL}{\ensuremath{\ell\ell}}%
\newcommand{\MW}{\ensuremath{{m}_\PW}}%
\newcommand{\MZ}{\ensuremath{{m}_\PZ}}%
\newcommand{\MT}{\ensuremath{{M}_{\mathrm{T}}}}%
\newcommand{\MLL}{\ensuremath{{M}_{\LpLm}}}%

\newcommand{\Wmn}{\ensuremath{\PW \ra \MN}}%
\newcommand{\Wpmn}{\ensuremath{\PWp \ra \MpN}}%
\newcommand{\Wmmn}{\ensuremath{\PWm \ra \MmN}}%
\newcommand{\Zmm}{\ensuremath{\PZ \ra \MpMm}\xspace}%
\newcommand{\Znn}{\ensuremath{\PZ \ra \nu\nu}}%
\newcommand{\ZLL}{\ensuremath{\PZ \ra \ell\ell}\xspace}%
\newcommand{\ppWmn}{\pp \ra \PW + X \ra \MN + X}%
\newcommand{\ppWpmn}{\pp \ra \PWp + X \ra \MpN + X}%
\newcommand{\ppWmmn}{\pp \ra \PWm + X \ra \MmN + X}%
\newcommand{\ppZmm}{\pp \ra \PZ + X \ra \MpMm + X}%

\newcommand{\Wen}{\ensuremath{\PW \ra \EN}}%
\newcommand{\Wpen}{\ensuremath{\PWp \ra \EpN}}%
\newcommand{\Wmen}{\ensuremath{\PWm \ra \EmN}}%
\newcommand{\Zee}{\ensuremath{\PZ \ra \EpEm}}%
\newcommand{\ppWen}{\pp \ra \PW + X \ra \EN + X}%
\newcommand{\ppWpen}{\pp \ra \PWp + X \ra \EpN  + X}%
\newcommand{\ppWmen}{\pp \ra \PWm + X \ra \EmN + X}%
\newcommand{\ppZee}{\pp \ra \PZ + X \ra \EpEm + X}%

\newcommand{\Wtn}{\ensuremath{\PW \ra \TN}}%
\newcommand{\Wptn}{\ensuremath{\PWp \ra \TpN}}%
\newcommand{\Wmtn}{\ensuremath{\PWm \ra \TmN}}%
\newcommand{\Ztt}{\ensuremath{\PZ \ra \TpTm}\xspace}%
\newcommand{\Zttll}{\ensuremath{\PZ \ra \TpTm/\LpLm}\xspace}%

\newcommand{\Wln}{\ensuremath{\PW \ra \LN}}%
\newcommand{\Wpln}{\ensuremath{\PWp \ra \LpN}}%
\newcommand{\Wmln}{\ensuremath{\PWm \ra \LmN}}%
\newcommand{\Zll}{\ensuremath{\PZ \ra \LpLm}\xspace}%
\newcommand{\ppZll}{\pp \ra \PZ + X \ra \LpLm + X}%
\newcommand{\ppWln}{\pp \ra \PW + X \ra \LN + X}%
\newcommand{\ppWpln}{\pp \ra \PWp + X \ra \LpN  + X}%
\newcommand{\ppWmln}{\pp \ra \PWm + X \ra \LmN + X}%

\newcommand{\gammaZ}{\ensuremath{\PZ/\gamma^{*}}}
\newcommand{\gammaZmm}{\mbox{$ \gammaZ\rightarrow \MpMm$}}
\newcommand{\gammaZee}{\mbox{$ \gammaZ\rightarrow \EpEm$}}
\newcommand{\gammaZtt}{\mbox{$ \gammaZ\rightarrow \TpTm$}}
\newcommand{\gammaZll}{\mbox{$ \gammaZ\rightarrow \LpLm$}}

\def\ZZ{\ensuremath{\PZ\PZ}\xspace}
\def\WZ{\ensuremath{\PW\PZ}\xspace}
\def\WW{\ensuremath{\PW\PW}\xspace}
\def\NN{\ensuremath{\nu \nu }\xspace }
\def\ZZllnn{\ensuremath{\ZZ \to \ell\ell\nu\nu}\xspace}
\def\ZZllll{\ensuremath{\ZZ \to 4\ell}\xspace}
\def\ZZeenn{\ensuremath{\ZZ \to \Pe\Pe\nu\nu}\xspace}
\def\ZZmmnn{\ensuremath{\ZZ \to \Pgm\Pgm\nu\nu}\xspace}
\def\WWllnn{\ensuremath{\WW \to \LN\LN}\xspace}
\def\WWeenn{\ensuremath{\WW \to \Pe\nu\Pe\nu}\xspace}
\def\WWmmnn{\ensuremath{\WW \to \Pgm\nu\Pgm\nu}\xspace}
\def\WWemnn{\ensuremath{\WW \to \Pe\nu\Pgm\nu}\xspace}
\def\WZllln{\ensuremath{\WZ \to \LN\ell\ell}\xspace}
\def\Zllj{\ensuremath{\ZLL + X}\xspace}
\def\WJ{\ensuremath{\PW+{\mathrm{jets}} }\xspace}
\def\WJo{\ensuremath{\PW+{\mathrm{1-jet}} }\xspace}
\def\ZJ{\ensuremath{\PZ+{\mathrm{jets}} }\xspace}
\def\VJ{\ensuremath{\PV+{\mathrm{jets}} }\xspace}
\def\GJ{\ensuremath{\gamma+{\mathrm{jets}} }\xspace}
\def\GamJ{\ensuremath{\gamma+{\mathrm{jets}} }\xspace}
\def\GZJ{\ensuremath{\gamma/\PZ+{\mathrm{jets}} }\xspace}
\def\VGam{\ensuremath{\PV+\Pgg }\xspace}
\def\WGam{\ensuremath{\PW+\Pgg }\xspace}
\def\ZGam{\ensuremath{\PZ+\Pgg }\xspace}


\newcommand{\mz}{\mbox{$m_{\PZ}$}}
\newcommand{\hta}{\mbox{$\eta$}}
\newcommand{\fh}{\mbox{$\phi$}}
\newcommand{\etot}{\mbox{$\epsilon_{tot}$}}
\newcommand{\eclustering}{\mbox{$ \epsilon_{clustering}$}}
\newcommand{\etracking}{\mbox{$ \epsilon_{tracking}$}}
\newcommand{\egsfele}{\mbox{$ \epsilon_{gsfele}$}}
\newcommand{\epreselection}{\mbox{$ \epsilon_{preselection}$}}
\newcommand{\eisolation}{\mbox{$ \epsilon_{isolation}$}}
\newcommand{\eclassification}{\mbox{$ \epsilon_{classification}$}}
\newcommand{\eelID}{\mbox{$ \epsilon_{elID}$}}
\newcommand{\etrigger}{\mbox{$ \epsilon_{trigger}$}}

\newcommand{\DE}{$\Delta\eta_{in}$}
\newcommand{\DP}{$\Delta\phi_{in}$}
\newcommand{\SEE}{$\sigma_{\eta\eta}$~}
\newcommand{\SEP}{$\sigma_{\eta\phi}$}
\newcommand{\SPP}{$\sigma_{\phi\phi}$}
\newcommand{\SXY}{$\sigma_{XY}$}
\newcommand{\pth}{\hat{p}_{\perp}}
\newcommand{\Lint}{\ensuremath{{\cal L}_{\mathrm{int}}}}
\newcommand{\IECAL}    {I^{\textrm{rel}}_{\scriptscriptstyle{\textrm{ECAL}}}}%
\newcommand{\IHCAL}    {I^{\textrm{rel}}_{\scriptscriptstyle{\textrm{HCAL}}}}%
\newcommand{\ITRK}     {I^{\textrm{rel}}_{\scriptscriptstyle{\textrm{trk}}}}%
\newcommand{\IRelComb} {I^{\textrm{rel}}_{\scriptscriptstyle{\textrm{comb}}}}%
\newcommand{\Nev}    {N_{\mathrm{ev}}}
\newcommand{\Nsig}   {N_{\mathrm{sig}}}
\newcommand{\Nsel}   {N_{\mathrm{sel}}}
\newcommand{\Nbg}    {N_{\mathrm{bg}}}
\newcommand{\rhoeff} {\rho_{\mathrm{eff}}}
\newcommand{\etaSC}  {\eta_{\mathrm{SC}}}


\def\effmc{\ensuremath{\epsilon_{\mathrm{sim}}}\xspace}
\def\effdt{\ensuremath{\epsilon_{\mathrm{data}}}\xspace}
\def\PUMET{\ensuremath{{\MET}^{\mathrm{PU}}}}
\def\PUMETMIN{\ensuremath{{\MET}^{\mathrm{PU,min}}}}
\def\CORRMET{\ensuremath{{\MET}^{\mathrm{corr}}}}

\def\sieie{\ensuremath{\sigma_{i\eta i\eta}}}
\def\sipip{\ensuremath{\sigma_{i\phi i\phi}}}

\def\ST{\ensuremath{S_{\text{T}}}\xspace}

\def\MassTJ{\ensuremath{\mathrm{M}(\tau,\mbox{j})}\xspace}
\def\MassLJ{\ensuremath{\mathrm{M}(\ell,\mbox{j})}\xspace}



\def\tauf{\ensuremath{\tau_{f}}\xspace}

%\def\symexamp{stone}{\stone}
%\def\symexamp{sttwo}{\sttwo}
\def\stone{\ensuremath{\tilde{\rm{t}}_1}}
\def\sttwo{\ensuremath{\tilde{\rm{t}}_2}}

%\def\effdt{\ensuremath{\varepsilon_{\rm data}}\xspace}
%\def\effmc{\ensuremath{\varepsilon_{\rm MC }}\xspace}

%
% results
%



% yields

% integrated luminosity
\newcommand{\thelumi}{\ensuremath{19.7~\fbinv}\xspace}

%acceptance, efficiencies



% fake rate numbers

\def\tfmc{\ensuremath{2.22 \pm 0.50 \%}\xspace}
\def\tfd{\ensuremath{2.44 \pm 0.53 \%}\xspace} 



% background Numbers
\def\ttbbkgElLMass{\ensuremath{ 11.0\pm1.48 \ \mathrm{(stat.)} }\xspace}
\def\dibbkgElLMass{\ensuremath{ 2.14\pm0.81 \ \mathrm{(stat.)} }\xspace}
\def\zttbkgElLMass{\ensuremath{ 0.31\pm0.09 \ \mathrm{(stat.)} }\xspace}

\def\ttbbkgElHMass{\ensuremath{ 6.89\pm1.12 \ \mathrm{(stat.)} }\xspace}
\def\dibbkgElHMass{\ensuremath{ 2.14\pm0.81 \ \mathrm{(stat.)} }\xspace}
\def\zttbkgElHMass{\ensuremath{ 0.17\pm0.07 \ \mathrm{(stat.)} }\xspace}


\def\ttbbkgMuLMass{\ensuremath{ 38.1\pm2.87 \ \mathrm{(stat.)} }\xspace}
\def\dibbkgMuLMass{\ensuremath{ 5.02\pm1.51 \ \mathrm{(stat.)} }\xspace}
\def\zttbkgMuLMass{\ensuremath{ 0.51\pm0.12 \ \mathrm{(stat.)} }\xspace}

\def\ttbbkgMuHMass{\ensuremath{ 27\pm2.42 \ \mathrm{(stat.)} }\xspace}
\def\dibbkgMuHMass{\ensuremath{ 5.02\pm1.51 \ \mathrm{(stat.)} }\xspace}
\def\zttbkgMuHMass{\ensuremath{ 0.41\pm0.11 \ \mathrm{(stat.)} }\xspace}

% uncertainty numbsers

\def\UncPUMax{\ensuremath{6 \%}\xspace}
\def\UncPU{\ensuremath{ 3 \% }\xspace}

\def\UncBTS{\ensuremath{ 3 \% }\xspace}
\def\UncBTB{\ensuremath{ 2 \% }\xspace}
\def\UncBSF{\ensuremath{ 4 \% }\xspace}
\def\UncLSF{\ensuremath{ 10 \% }\xspace}
\def\UncBT{\ensuremath{ 2\mbox{-}3 \% }\xspace}
\def\UncBTSEff{\ensuremath{ 1 \% }\xspace}
\def\UncBTBEff{\ensuremath{ 2 \% }\xspace}
\def\UncBSFSig{\ensuremath{ 2 \% }\xspace}
\def\UncBSFtt{\ensuremath{ 2 \% }\xspace}
\def\UncBSFsinglet{\ensuremath{ 1 \% }\xspace}
\def\UncBSFZtt{\ensuremath{ <1 \% }\xspace}
\def\UncBSFEWK{\ensuremath{ 1 \% }\xspace}
\def\UncLSFSig{\ensuremath{ 2 \% }\xspace}
\def\UncLSFtt{\ensuremath{ 2 \% }\xspace}
\def\UncLSFZtt{\ensuremath{ 7 \% }\xspace}
\def\UncLSFEWK{\ensuremath{ 6 \% }\xspace}
\def\UncLSFsinglet{\ensuremath{ 3 \% }\xspace}

\def\UncBMTS{\ensuremath{ 2 \% }\xspace}
\def\UncBMTB{\ensuremath{ 5 \%  }\xspace}
\def\UncBMT{\ensuremath{ 2\mbox{-}5 \% }\xspace}
\def\UncBMTSEff{\ensuremath{ 1\% }\xspace}
\def\UncBMTBEff{\ensuremath{ 2\mbox{-}12\% }\xspace}

\def\UncTTBNorm{\ensuremath{ 19\mbox{-}22\% }\xspace}
\def\UncZttNorm{\ensuremath{ 2 \%}\xspace}
\def\UncDibNorm{\ensuremath{ 5-14 \% }\xspace} %5,9,14 -> WZ,WW,ZZ
\def\UncSingleTopNorm{\ensuremath{ 14 \% }\xspace} %14 -> t data ; 4-6-9 MC (s,tW,t)

\def\UncFake{\ensuremath{ 16\mbox{-}24\% }\xspace}
\def\UncFakeSEff{\ensuremath{ 0\% }\xspace}
\def\UncFakeBEff{\ensuremath{ 22\% }\xspace}
%\def\UncFakeMu{\ensuremath{ \% }\xspace}

\def\UncElESEB{\ensuremath{ 1 \% }\xspace}
\def\UncElESEE{\ensuremath{ 2.5 \% }\xspace}


\def\UncMuES{\ensuremath{ 1 \% }\xspace}
\def\UncTauES{\ensuremath{ 3 \% }\xspace}
\def\UncTauER{\ensuremath{ 10 \% }\xspace}
\def\UncJES{\ensuremath{ \sim 4 \% }\xspace}
\def\UncJER{\ensuremath{ 5\mbox{-}10 \% }\xspace}

\def\UncLumi{\ensuremath{2.6 \% }\xspace}
\def\UncTrig{\ensuremath{ 2 \% }\xspace}

\def\UncElId{\ensuremath{ 2 \% }\xspace}
\def\UncElIdSEff{\ensuremath{ 2 \% }\xspace}
\def\UncElIdBEff{\ensuremath{ 2 \% }\xspace}

\def\UncMuId{\ensuremath{ 2 \% }\xspace}
\def\UncMuIdSEff{\ensuremath{ 2 \% }\xspace}
\def\UncMuIdBEff{\ensuremath{ 2 \% }\xspace}

\def\UncTauId{\ensuremath{ 6 \% }\xspace}
\def\UncTauIdSEff{\ensuremath{ 6 \% }\xspace}
\def\UncTauIdBEff{\ensuremath{ 6 \% }\xspace}

%\def\TotUncSig{\ensuremath{ 6.5 \% }\xspace}
\def\TotUncSig{\ensuremath{ 8.6-13.9 \% }\xspace}
\def\TotUncBkg{\ensuremath{ 23-32\%}\xspace}
\def\TotUncBkgFake{ \ensuremath{ 16\mbox{-}24\%}\xspace}
\def\TotUncBkgTtb{ \ensuremath{ 13\mbox{-}17\%}\xspace}


%% tau/jet ES ER
\def\UncJESBkg{\ensuremath{0-7\%}\xspace}
\def\UncJERBkg{\ensuremath{0-5\%}\xspace}
\def\UncTESBkg{\ensuremath{5-19\%}\xspace}
\def\UncTERBkg{\ensuremath{20 \%}\xspace}

\def\UncJESSig{\ensuremath{\sim 1\%}\xspace}
\def\UncJERSig{\ensuremath{\sim 1\%}\xspace}
\def\UncTESSig{\ensuremath{0-5\%}\xspace}
\def\UncTERSig{\ensuremath{1-9\%}\xspace}


\newcommand{\tbsp}{\rule{0pt}{18pt}} %used to get a vertical distance after \hline
\renewcommand{\baselinestretch}{2}
\setlength{\textwidth}{5.9in}
\setlength{\textheight}{9in}
\setlength{\topmargin}{-.50in}
%\setlength{\topmargin}{0in}    %use this setting if the printer makes the the top margin 1/2 inch instead of 1 inch
\setlength{\oddsidemargin}{.55in}
\setlength{\parindent}{.4in}
\pagestyle{empty}
\setcounter{tocdepth}{3}
\setcounter{secnumdepth}{3}

\begin{document}

% input: [Abstract.tex]
%Abstract Page 

\hbox{\ }

\renewcommand{\baselinestretch}{1}
\small \normalsize

\begin{center}
\large{{ABSTRACT}} 

\vspace{3em} 

\end{center}

\begin{singlespace}

\noindent
\begin{tabular}{@{}ll}
Title of dissertation:    & {\large  Search for Pair Production of}\\
                          & {\large  Third-Generation Scalar Leptoquarks} \\
                          & {\large  and R-Parity Violating Top Squarks} \\
                          & {\large  in Proton-Proton Collisions at $\sqrt{s} = 8\TeV$} \\
                          & \\
                          & {\large  Kevin Pedro, Doctor of Philosophy, 2014} \\
                          & \\
Dissertation directed by: & {\large  Professor Sarah C. Eno} \\
                          & {\large  Department of Physics } \\
\end{tabular}

\end{singlespace}

\vspace{3em}

\renewcommand{\baselinestretch}{2}
\large \normalsize

insert abstract here


 % input: [Titlepage.tex]
%Titlepage

\thispagestyle{empty}
\hbox{\ }
\vspace{1in}
\renewcommand{\baselinestretch}{1}
\small\normalsize
\begin{center}

\large{Search for Pair Production of Third-Generation Scalar Leptoquarks and R-Parity Violating Top Squarks in Proton-Proton Collisions at $\sqrt{s} = 8\TeV$}\\
\ \\
\ \\
\large{by} \\
\ \\
\large{Kevin Pedro}%Your full name as it appears in University records.
\ \\
\ \\
\ \\
\ \\
\normalsize
Dissertation submitted to the Faculty of the Graduate School of the \\
University of Maryland, College Park in partial fulfillment \\
of the requirements for the degree of \\
Doctor of Philosophy \\
2014
\end{center}

\vspace{7.5em}

\noindent Advisory Committee: \\
Professor Sarah C. Eno, Chair/Advisor \\
Professor Nick Hadley \\
Assistant Professor Alberto Belloni \\
Associate Professor Zackaria Chacko \\
Professor Alice Mignerey, Dean's Representative
 % input: [Copyright.tex]
%Copyright

\thispagestyle{empty}
\hbox{\ }

\vfill
\renewcommand{\baselinestretch}{1}
\small\normalsize

\vspace{-.65in}

\begin{center}
\large{\copyright \hbox{ }Copyright by\\
Kevin Pedro  %Type your name as it appears in University records
\\
2014}
\end{center}

\vfill
 
%Pages from this point start at lower-case Roman number ii)
\pagestyle{plain}
\pagenumbering{roman}
\setcounter{page}{2}

%\include{Preface}  %(if present, start at lower-case Roman number ii)
%\include{Foreword} %(if present, lower-case Roman)
% input: [Dedication.tex]
%Dedication

\renewcommand{\baselinestretch}{2}
\small\normalsize
\hbox{\ }
 
\vspace{-.65in}

\begin{center}
\large{Dedication}
\end{center} 

\begin{center}
To my parents, Philip and Lisa
\end{center} % input: [Acknowledgements.tex]
%Acknowledgments

\renewcommand{\baselinestretch}{2}
\small\normalsize
\hbox{\ }
 
\vspace{-.65in}

\begin{center}
\large{Acknowledgments} 
\end{center} 

\vspace{1ex}

insert acknowledgments here
 
\renewcommand{\baselinestretch}{1}
%\begin{singlespace}
\normalsize
\phantomsection
\addcontentsline{toc}{chapter}{Table of Contents}
\protect\tableofcontents %(required, lower-case Roman)
\newpage
\protect\listoftables %(if present, lower-case Roman)
\newpage
\protect\listoffigures %(if present, lower-case Roman)
\newpage
%\end{singlespace}

% LIST OF ABBREVIATIONS
\phantomsection
\addcontentsline{toc}{chapter}{List of Abbreviations}
% input: [Abbreviations.tex]
%List of Abbreviations

\renewcommand{\baselinestretch}{1}
\small\normalsize
\hbox{\ }

\vspace{-4em}

\begin{center}
\large{List of Abbreviations}
\end{center} 

\vspace{3pt}

\begin{longtable}[l]{@{}l@{\ \ \ \ \ \ \ \ \ \ \ \ }l}
ALICE      & A Large Ion Collider Experiment \\
APD        & Avalanche Photodiode \\
APV        & Atomic Parity Violation \\
ASIC       & Application-Specific Integrated Circuits \\
ATLAS      & A Toroidal LHC ApparatuS \\
BDT        & Boosted Decision Tree \\
BICEP      & Background Imaging of Cosmic Extragalactic Polarization \\
BPTX       & Beam Pick-up Timing for the eXperiments \\
BRW        & Buchm\"{u}ller-R\"{u}ckl-Wyler \\
BSM        & Beyond Standard Model \\
CDF        & Cumulative Distribution Function \\
CERN       & European Organization for Nuclear Research \\
CH         & Charged Hadron \\
CKM        & Cabibbo-Kobayashi-Maskawa \\
CL         & Confidence Level \\
CMS        & Compact Muon Solenoid \\
CMSSW      & CMS Software \\
CP         & Charge-Parity \\
CPU        & Central Processing Unit \\
CSC        & Cathode Strip Chamber \\
CSCTF      & Cathode Strip Chamber Track Finder \\
CTEQ       & Coordinated Theoretical-Experimental Project on QCD \\
CTF        & Combinatorial Track Finder \\
DAQ        & Data Acquisition \\
DGLAP      & Dokshitzer-Gribov-Lipatov-Altarelli-Parisi \\
DT         & Drift Tube \\
DTTF       & Drift Tube Track Finder \\
EB         & ECAL Barrel \\
ECAL       & Electromagnetic Calorimeter \\
EE         & ECAL Endcap \\
EM         & Electromagnetic \\
ES         & ECAL Preshower \\
EWSB       & Electroweak Symmetry Breaking \\
FCNC       & Flavor-Changing Neutral Current \\
FPGA       & Field-Programmable Gate Array \\
FSR        & Final-State Radiation \\
GSF        & Gaussian Sum Filter \\
GUT        & Grand Unified Theory \\
HB         & HCAL Barrel \\
HCAL       & Hadron Calorimeter \\
HE         & HCAL Endcap \\
HF         & HCAL Forward \\
HO         & HCAL Outer \\
HPD        & Hybrid Photodiode \\
HLT        & High-Level Trigger \\
IP         & Interaction Point \\
ISR        & Initial-State Radiation \\
L1         & Level 1 \\
L1A        & Level-1 Accept \\
LEAR       & Low Energy Antiproton Ring \\
LEIR       & Low Energy Ion Ring \\
LEP        & Large Electron-Positron Collider \\
LHC        & Large Hadron Collider \\
LHCb       & Large Hadron Collider beauty \\
LO         & Leading Order \\
LQ         & Leptoquark \\
LSP        & Lightest Supersymmetric Particle \\
LUT        & Lookup Table \\
mBRW       & minimal Buchm\"{u}ller-R\"{u}ckl-Wyler \\
MB         & Muon Barrel \\
MC         & Monte Carlo \\
ME         & Muon Endcap \\
MET        & Missing Transverse Energy \\
MIP        & Minimum Ionizing Particle \\
MSTW       & Martin-Stirling-Thorne-Watt \\
MVA        & Multivariate \\
NH         & Neutral Hadron \\
NLO        & Next-to-Leading Order \\
NNLO       & Next-to-Next-to-Leading Order \\
PD         & Primary Dataset \\
PF         & Particle Flow \\
PDF        & Parton Distribution Function \\
PDF        & Probability Density Function \\
PMNS       & Pontecorvo-Maki-Nakagawa-Sakata \\
PMT        & Photomultiplier Tube \\
PS         & Proton Synchrotron \\
PSB        & Proton Synchrotron Booster \\
PU         & Pileup \\
QED        & Quantum Electrodynamics \\
QCD        & Quantum Chromodynamics \\
RF         & Radio Frequency \\
RMS        & Root Mean Square \\
RPC        & Resistive Plate Chamber \\
RPC        & R-Parity Conserving \\
RPV        & R-Parity Violating \\
SiPM       & Silicon Photomultiplier \\
SLHA       & SUSY Les Houches Accord \\
SM         & Standard Model \\
SPS        & Super Proton Synchrotron \\
SUSY       & Supersymmetry \\
TCS        & Trigger Control System \\
TEC        & Tracker End Cap \\
TIB        & Tracker Inner Barrel \\
TID        & Tracker Inner Disks \\
TOB        & Tracker Outer Barrel \\
TPG        & Trigger Primitive Generator \\
TTC        & Timing, Trigger and Control \\
VEV        & Vacuum Expectation Value \\
VPT        & Vacuum Phototriode \\
WIMP       & Weakly Interacting Massive Particle \\
WLS        & Wavelength-Shifting \\
\end{longtable}

\newpage
\setlength{\parskip}{0em}
\renewcommand{\baselinestretch}{2}
\normalsize

%Pages from this point start at Arabic numeral 1
\setcounter{page}{1}
\pagenumbering{arabic}
% input: [introduction.tex]
\chapter{Introduction
\label{ch:introduction}}
% input: [theory.tex]
\chapter{Theoretical Motivations
\label{ch:theory}}

\setcounter{section}{-1}

\section{The Standard Model}

The standard model (SM) of particle physics includes three fundamental forces and all of the particles of ordinary matter using a quantum field theory framework. When forces are weak enough, quantum field theory calculations use a perturbative method in which the leading order (LO) term is calculated, then the next-to-leading order (NLO) correction is added, then the next-to-next-to-leading order correction (NNLO), and so forth. The three fundamental forces are electromagnetism, the weak force, and the strong force. These forces are carried by spin-1 gauge bosons, while the matter particles consist of quarks and leptons, two categories of spin-1/2 fermions. In addition, the standard model contains a scalar spin-0 boson, the Higgs boson, which is part of the Higgs field that provides masses to certain gauge bosons and fermions. Figure \ref{fig:sm-particles} summarizes the particles in the standard model, including the spin, electric charge, and mass values of each particle. The standard model and quantum field theory are described in more detail in Refs. \cite{Griffiths} and \cite{Peskin}.

\begin{figure}[hbt]
\begin{center}
\includegraphics[width=0.95\textwidth]{figures/Standard_Model_of_Elementary_Particles.pdf}
\caption{A table of all the elementary particles in the standard model, with the spin, electric charge, and mass values of each particle \cite{MissMJ}. The faint gray lines indicate which gauge bosons interact with which fermions.}
\label{fig:sm-particles}
\end{center}
\end{figure}

The electromagnetic force is mediated by photons ($\gamma$), spin-1 gauge bosons which have no mass or electric charge $Q$. This force causes interactions between electrically charged particles and has infinite range due to the masslessness of the photon. Quantum electrodynamics (QED) is the quantum field theory description of electromagnetism, which can be represented by a $U(1)$ symmetry group. The weak force is mediated by the massive \Wpm and \Z bosons and can be represented by an $SU(2)$ symmetry group. The weak force acts on particles carrying weak isospin $T$. Weak isospin is a quantum number whose third component $T_3$ is conserved in all interactions and which can be mathematically treated in the same way as angular momentum, though the two quantities are physically distinct. The massiveness of the weak carrier bosons means that the weak force has a limited range, approximately $10^{-18}\unit{m}$. The charged current weak interaction, mediated by the \Wpm bosons, is sensitive to the chirality of fermions; only left-handed fermions and right-handed antifermions participate in this interaction. The neutral current weak interaction, mediated by the \Z boson, is not sensitive to chirality.

As suggested by the inclusion of both forces in the previous paragraph, the electromagnetic and weak forces can be unified to form the electroweak force, represented by the symmetry group $SU(2) \times U(1)$. In this unification, the quantum numbers of electromagnetism and the weak force are related by a new conserved quantum number, weak hypercharge $Y = 2(Q - T_3)$. The Higgs mechanism is responsible for electroweak symmetry breaking (EWSB). In order for the electroweak theory to be gauge invariant, the gauge bosons must be massless, but the \Wpm and \Z bosons are observed to have mass. The Higgs mechanism solves this dilemma via spontaneous EWSB due to its non-zero vacuum expectation value (VEV). The Higgs field consists of a doublet, with two charged particles and two neutral particles, all scalar bosons. The two charged particles and one of the neutral particles act as Goldstone bosons, combining with the \Wpm and \Z bosons to produce their masses. The remaining neutral particle is the Higgs boson, which was discovered at the LHC in 2012 \cite{NewBoson}.

The strong force, quantum chromodynamics (QCD), is mediated by gluons (\cPg) and can be represented by an $SU(3)$ symmetry group. Gluons, like photons, are spin-1 gauge bosons without mass or electric charge. However, gluons do possess color charge, the quantum number on which the strong force acts. Color charge is so named because the charge has three possible values, which are labeled red, green, or blue. Because gluons both mediate and participate in the strong interaction, the force between quarks does not decrease as they become spatially separated. The energy in the gluon field between the separated quarks can become large enough to form one or more quark-antiquark pairs. This phenomenon is known as confinement and prevents quarks or gluons from existing in a bare state. Correspondingly, the range of the strong force is limited to ${\sim} 10^{-15}\unit{m}$. Bound states of quarks and gluons, the only way they have ever been observed in nature, are called hadrons, and the formation of those bound states is called hadronization. States with one quark and one antiquark are mesons, while states with three quarks are baryons. Mesons and baryons are the two allowed types of bound states because they represent color singlets. Complementarily, as quarks get closer together, the strong force between them weakens. This behavior is known as asymptotic freedom; because short distances are equivalent to high energies, the strong interactions of quarks at a high-energy collider like the LHC can be calculated perturbatively. A residual form of the strong force acts on nucleons, protons and neutrons, to form atomic nuclei.

As mentioned, fermions are the particles of matter, which are separated into two groups: quarks and leptons. Quarks have fractional electric charge, weak isospin, and color charge, so they are affected by all three fundamental forces. There are two types of quarks: up-type quarks that have $Q = 2/3$ and down-type quarks that have $Q = -1/3$. Leptons consist of charged leptons and neutrinos. Charged leptons possess electric charge and weak isospin, while neutrinos only possess weak isospin. Three generations exist for each type of particle, with the different particles called flavors. The flavors of up-type quarks are the up, charm, and top quarks; of down-type quarks are the down, strange, and bottom quarks; of charged leptons are the electron, muon, and tau lepton; and of the neutrinos are the electron, muon, and tau neutrinos. The charged current weak interaction mixes the different flavors of quarks, with the amount of mixing between any two flavors given by the unitary Cabibbo-Kobayashi-Maskawa (CKM) matrix. The top quark is the heaviest elementary particle and is so heavy that it decays before hadronizing, making it an exception to the rule that bare quarks are not observed. Quarks possess an additively conserved quantum number called baryon number $B$, which is defined as $B = \frac{1}{3}(n_{\cPq} - n_{\overline{\cPq}})$. Similarly, lepton number $L$ is defined for leptons as $L = n_{\ell} - n_{\overline{\ell}}$. Specific lepton flavor numbers $L_{\Pe}$, $L_{\mu}$, $L_{\tau}$ are defined for each flavor pair of leptons.

The fermions are arranged into multiplets based on their chirality. The left-handed up- and down-type quarks are grouped together in a doublet $\cPq_L$, as are the left-handed charged leptons and neutrinos in $\ell_L$. The right-handed particles are singlets. It is important to note that right-handed neutrinos, and correspondingly left-handed antineutrinos, do not exist in the standard model. The Higgs field spontaneously provides masses to the quarks and charged leptons through a Yukawa interaction which couples the left- and right-handed versions of each flavor of particle. The quantum numbers of each type of particle are summarized in Table \ref{tab:q-num}, and the interactions among all the particles are illustrated in Fig. \ref{fig:sm-interactions}.

\begin{table}[htb]
  \begin{center}
%    \def\arraystretch{2.0} %1 is the default
    \begin{tabular}{|l||l|r|r|r|r|r|}
\hline
      & \multicolumn{1}{c|}{Particle} & \multicolumn{1}{c|}{$Q$} & \multicolumn{1}{c|}{$T_3$} & \multicolumn{1}{c|}{$Y$} & \multicolumn{1}{c|}{$B$} & \multicolumn{1}{c|}{$L$} \\
\hline
\hline
\multirow{3}{*}{Quarks}  
\rule{0pt}{24pt}         & $\cPq_L = \doublet[r]{\cPqu}{\cPqd}_L$ & $\doublet[r]{2/3}{-1/3}$ & $\doublet[r]{1/2}{-1/2}$ & $1/3$  & $1/3$ & 0 \\
                         & $\cPqu_R$                              & $2/3\hphantom{\bigg)}$   & $0\hphantom{\bigg)}$     & $4/3$  & $1/3$ & 0 \\
                         & $\cPqd_R$                              & $-1/3\hphantom{\bigg)}$  & $0\hphantom{\bigg)}$     & $-2/3$ & $1/3$ & 0 \\
\hline
\hline
\multirow{2}{*}{Leptons} 
\rule{0pt}{24pt}         & $\ell_L = \doublet[r]{\nu}{\Pe}_L$     & $\doublet[r]{0}{-1}$     & $\doublet[r]{1/2}{-1/2}$ & $-1$   & 0     & 1 \\
                         & $\Pe_R$                                & $-1\hphantom{\bigg)}$    & $0\hphantom{\bigg)}$     & $-2$   & 0     & 1 \\
\hline
    \end{tabular}
    \caption{The quantum numbers of each category of fermions, based on chirality and particle type: up-type quarks, down-type quarks, charged leptons, and neutrinos. The various flavors of each category, also called the first, second, and third generations of matter, possess the same quantum numbers and differ only in their masses.}
    \label{tab:q-num}
  \end{center}
\end{table}

\begin{figure}[hbt]
\begin{center}
\includegraphics[width=0.95\textwidth]{figures/Elementary_particle_interactions_in_the_Standard_Model.png}
\caption{A diagram illustrating the leading order interactions between particles in the standard model, including self-interactions \cite{Drexler}.}
\label{fig:sm-interactions}
\end{center}
\end{figure}

\addtocounter{section}{-1}
\renewcommand{\thesection}{\thechapter.{$\frac{1}{2}$}}
\section{Beyond the Standard Model}
\renewcommand{\thesection}{\thechapter.\arabic{section}} % back to regular numbering

The predictions of the standard model have been confirmed by decades of precise experimental tests. However, as an effective field theory, its domain of applicability is ultimately limited; at a high enough energy, the theory will break down. Further, the standard model is unable to account for some observations. These limitations and indications of new phenomena motivate various searches for physics beyond the standard model (BSM), include the searches which will be presented in this dissertation.

The most obvious limitation of the standard model is that it depends on 19 free parameters which must be determined by experiment. These parameters include the nine fermion masses, the Higgs mass and VEV, the gauge couplings for the three forces, the three mixing angles and one phase from the CKM matrix, and finally the QCD vacuum angle. It is conceivable that a BSM theory could reduce the number of free parameters. Electroweak unification and charge quantization suggest that a Grand Unified Theory (GUT) could unify all three fundamental forces, at an expected energy scale of ${\sim} 10^{16}\GeV$. The observation of neutrino oscillations is now well established, indicating that neutrinos have small but non-zero masses. The oscillation of neutrinos between mass and flavor eigenstates is described by the Pontecorvo-Maki-Nakagawa-Sakata (PMNS) matrix. Like the CKM matrix for quarks, the PMNS matrix is unitary and has four parameters: three mixing angles and one phase. The values of the neutrino masses have not been directly measured, but are indirectly limited to the \eVns scale. Because right-handed neutrinos do not exist in the Standard Model, neutrinos cannot gain mass via a Yukawa interaction with the Higgs field. Though the Higgs mass is a parameter of the SM, it can be calculated in many BSM theories. Unless unnatural fine-tuning occurs, the presence of new massive particles makes this calculation produce a value near the Planck scale of $10^{19}\GeV$, orders of magnitude higher than the observed value of 125\GeV, which characterizes the electroweak scale. This is known as the hierarchy problem. To construct a natural theory which avoids such fine-tuning while making predictions that agree with observation, it is necessary to cancel divergent contributions to the Higgs mass. The hierarchy problem is an important motivation to search for new physics at the LHC \cite{Morrissey20121}.

Astrophysics provides numerous indications of the need for BSM theories. Most notably, gravity is not included in the standard model. A successful unification of general relativity and quantum field theory has not been achieved, due to the difficulty of constructing a renormalizable theory for the spin-2 graviton. At the Planck scale, gravitational effects become comparable to SM interactions, which calls for a new theory. The measurement of galactic rotation curves and galaxy cluster collisions \cite{BulletCluster} indicates that ${\sim} 85\%$ of the matter in the universe is dark matter, which interacts gravitationally but not electromagnetically. Dark matter is likely to be a new particle not present in the standard model. The most popular type of dark matter candidate is a weakly interacting massive particle (WIMP) \cite{Morrissey20121}, but many other candidates have been proposed. The dearth of antimatter relative to the amount of matter in the universe requires greater violation of charge-parity (CP) symmetry than is contained in the SM. The accelerating expansion of the universe, observed using standard candle supernovae \cite{Supernova98,Supernova99}, implies that ${\sim} 70\%$ of the energy of the universe exists in a unknown form called dark energy. Dark energy may be due to the cosmological constant of the universe, which can be related to the energy of the vacuum in quantum field theory. However, the standard model predicts that this cosmological constant would be $10^{120}$ times larger than the observed value, a clear failure of the theory. The recent tentative evidence for cosmic inflation from the Background Imaging of Cosmic Extragalactic Polarization (BICEP) experiment \cite{BICEP} implies the existence of the inflaton, a new scalar field.

The following sections discuss the theories of leptoquarks and R-parity violating supersymmetry. The existence of leptoquarks can be a consequence of grand unification or other theories that address the parallels between leptons and quarks in the standard model. Supersymmetry is motivated by basic considerations of quantum field theory, the hierarchy problem, grand unification, and dark matter. The introduction of R-parity violation in supersymmetry evades existing limits on signatures with large missing transverse energy due to the stability of the lightest supersymmetric particle, while retaining the other desirable characteristics of supersymmetry. R-parity violating supersymmetry can also act as a signature generator to suggest novel searches which might discover or rule out other BSM theories \cite{EvansSigGen}.

\section{Leptoquarks}

Many BSM theories include a deeper relationship between leptons and quarks. Such theories introduce a class of particles called leptoquarks (LQs), which possess lepton number, baryon number, color charge, and fractional electric charge. Leptoquarks are bosons, either scalar with spin 0 or vector with spin 1. The values of the LQ quantum numbers, including those listed previously as well as weak isospin, which they may or may not possess, are model-dependent. A total fermion number $F$ can be defined as $F = 3B + L$ to characterize the combinations of lepton and baryon numbers found in LQs. The possible values are $F=0$ for LQs which couple to $\overline{\ell}\cPq$ or $\ell\overline{\cPq}$ pairs and $|F|=2$ for coupling to $\ell\cPq$ or $\overline{\ell}\overline{\cPq}$ pairs.

The first BSM theory to include leptoquarks was Pati-Salam $SU(4)$ \cite{SU4}, a GUT which casts lepton number as the fourth type of color charge, hence the $SU(4)$ symmetry instead of the SM $SU(3)$. Another GUT, Georgi-Glashow $SU(5)$ \cite{GUT}, also contains leptoquarks. In general, grand unified theories group leptons and quarks together in multiplets. However, the symmetry breaking in these theories typically occurs at the GUT scale, rendering the expected LQ masses very large and therefore unable to be directly produced at colliders. $E_6$ superstring theory \cite{SUPERSTR} can also contain LQs, as it behaves similarly to GUTs below the string compactification scale, which is typically near the Planck scale. In other models, leptoquarks may be composite particles \cite{LQ3b}. These include extended technicolor theories \cite{TC3}, which postulate a new strong interaction similar to QCD and provide spontaneous masses to SM fermions using technifermions. A techniquark and anti-technilepton can bind together to form a technimeson which interacts with SM fermions as a leptoquark with a model-dependent coupling. Technicolor, though, has become disfavored with the confirmation of the Higgs mechanism for EWSB and fermion masses.

The Buchm\"{u}ller-R\"{u}ckl-Wyler (BRW) model of leptoquarks includes all renormalizable Lagrangian terms compatible with the standard model \cite{BRW,BRWerr}. There are several constraints imposed in the BRW model:
\begin{enumerate}
\item LQ interactions are dimensionless, in order to be renormalizable.
\item LQ interactions are invariant under the overall SM symmetry $SU(3) \times SU(2) \times U(1)$.
\item LQ interactions conserve $B$ and $L$, to avoid contributions to proton decay.
\item LQs couple only to SM particles.
\end{enumerate}
Further consideration of experimental limits imposes two additional constraints, creating the minimal BRW (mBRW) model:
\begin{enumerate}
\setcounter{enumi}{4}
\item LQ couplings are chiral, involving either only left-handed fermions or only right-handed fermions, due to limits on otherwise chirally-suppressed decays such as $\pi^{+} \rightarrow \Pe^{+} \nu_{\Pe}$.
\item LQ couplings involve only a single generation of leptons and quarks, to evade limits on flavor-changing neutral currents (FCNCs).
\end{enumerate}
The Lagrangian terms are given for scalar LQs in Eq. \eqref{eq:Lagrangian-SLQ} and for vector LQs in Eq. \eqref{eq:Lagrangian-VLQ} below, using the ``Aachen'' notation as specified in Ref. \cite{ModelIndLQ}.
\begin{align}
\label{eq:Lagrangian-SLQ}
\mathcal{L}_{S} = &\hphantom{+~}(\lambda_{L,S_{0}} \overline{\cPq}_{L}^{c} i\sigma_{2} \ell_{L} + \lambda_{R,S_{0}} \overline{\cPqu}_{R}^{c} \Pe_{R})S_{0}^{\dagger}
+ \lambda_{R,\widetilde{S}_{0}} \overline{\cPqd}_{R}^{c} \Pe_{R} \widetilde{S}_{0}^{\dagger} \nonumber \\
&+ (\lambda_{L,S_{1/2}} \overline{\cPqu}_{R} \ell_{L} + \lambda_{R,S_{1/2}} \overline{\cPq}_{L} i\sigma_{2} \Pe_{R})S_{1/2}^{\dagger}
+ \lambda_{L,\widetilde{S}_{1/2}} \overline{\cPqd}_{R} \ell_{L} \widetilde{S}_{1/2}^{\dagger} \nonumber \\
&+ \lambda_{L,S_{1}} \overline{\cPq}_{L}^{c} i\sigma_{2}\boldsymbol{\sigma} \ell_{L} \cdot \boldsymbol{S}_{1}^{\dagger} + \text{h.c.}
\end{align}
\begin{align}
\label{eq:Lagrangian-VLQ}
\mathcal{L}_{V} = &\hphantom{+~}(\lambda_{L,V_{0}} \overline{\cPq}_{L} \gamma_{\mu} \ell_{L} + \lambda_{R,V_{0}} \overline{\cPqd}_{R} \gamma_{\mu} \Pe_{R})V_{0}^{\mu\dagger}
+ \lambda_{R,\widetilde{V}_{0}} \overline{\cPqu}_{R} \gamma_{\mu} \Pe_{R} \widetilde{V}_{0}^{\mu\dagger} \nonumber \\
&+ (\lambda_{L,V_{1/2}} \overline{\cPqd}_{R}^{c} \gamma_{\mu} \ell_{L} + \lambda_{R,V_{1/2}} \overline{\cPq}_{L}^{c} \gamma_{\mu} \Pe_{R})V_{1/2}^{\mu\dagger}
+ \lambda_{L,\widetilde{V}_{1/2}} \overline{\cPqu}_{R}^{c} \gamma_{\mu} \ell_{L} \widetilde{V}_{1/2}^{\mu\dagger} \nonumber \\
&+ \lambda_{L,V_{1}} \overline{\cPq}_{L} \gamma_{\mu}\boldsymbol{\sigma} \ell_{L} \cdot \boldsymbol{V}_{1}^{\mu\dagger} + \text{h.c.}
\end{align}
In these equations, scalar leptoquarks are denoted by $S$ and vector leptoquarks are denoted by $V$. The subscripts 0, 1/2, and 1 denote singlet, doublet, and triplet states, respectively. The $\widetilde{S}$ and $\widetilde{V}$ leptoquarks have different quantum numbers compared to the corresponding $S$ and $V$ leptoquarks. The coupling constants for the Yukawa couplings between leptoquarks, leptons, and quarks are represented by $\lambda$, with the chirality $L$ or $R$ and the leptoquark type indicated in the subscript. The generation indices for the couplings and fermion multiplets are suppressed. The Pauli matrices are denoted by $\sigma_{i}$ and the Dirac matrices by $\gamma_{\mu}$. The Hermitian conjugate terms are indicated as ``h.c.'' The quantum numbers for the different types of mBRW leptoquarks are listed in Table \ref{tab:lq-num}.

\begin{table}[htb]
  \begin{center}
%    \def\arraystretch{3.0} %1 is the default
    \begin{tabular}{|l||l|r|r|r|r|}
\hline
      & \multicolumn{1}{c|}{Particle} & \multicolumn{1}{c|}{$Q$} & \multicolumn{1}{c|}{$T_3$} & \multicolumn{1}{c|}{$Y$} & \multicolumn{1}{c|}{$F$} \\
\hline
\hline
\multirow{10}{*}{Scalar} & $S_{0}$               & $-1/3\hphantom{\bigg)}$                            & $0\hphantom{\bigg)}$                              & $-2/3$ & $2$ \\
                         & $\widetilde{S}_{0}$   & $-4/3\hphantom{\bigg)}$                            & $0\hphantom{\bigg)}$                              & $-8/3$ & $2$ \\
\rule{0pt}{24pt}         & $S_{1/2}$             & $\doublet[r]{-2/3}{-5/3}$ & $\doublet[r]{1/2}{-1/2}$ & $-7/3$ & $0$ \\
\rule{0pt}{24pt}         & $\widetilde{S}_{1/2}$ & $\doublet[r]{1/3}{-2/3}$  & $\doublet[r]{1/2}{-1/2}$ & $-1/3$ & $0$ \\
\rule{0pt}{36pt}         & $S_{1}$               & $\triplet[r]{2/3}{-1/3}{-4/3}$       & $\triplet[r]{1\vphantom{/}}{0\vphantom{/}}{-1\vphantom{/}}$             & $-2/3$ & $2$ \\
\hline
\hline
\multirow{10}{*}{Vector} & $V_{0}$               & $-2/3\hphantom{\bigg)}$                            & $0\hphantom{\bigg)}$                              & $-4/3$  & $0$ \\
                         & $\widetilde{V}_{0}$   & $-5/3\hphantom{\bigg)}$                            & $0\hphantom{\bigg)}$                              & $-10/3$ & $0$ \\
\rule{0pt}{24pt}         & $V_{1/2}$             & $\doublet[r]{-1/3}{-4/3}$ & $\doublet[r]{1/2}{-1/2}$ & $-5/3$  & $2$ \\
\rule{0pt}{24pt}         & $\widetilde{V}_{1/2}$ & $\doublet[r]{2/3}{-1/3}$  & $\doublet[r]{1/2}{-1/2}$ & $1/3$   & $2$ \\
\rule{0pt}{36pt}         & $V_{1}$               & $\triplet[r]{1/3}{-2/3}{-5/3}$       & $\triplet[r]{1\vphantom{/}}{0\vphantom{/}}{-1\vphantom{/}}$             & $-4/3$  & $0$ \\
\hline
    \end{tabular}
    \caption{The quantum numbers of the different types of scalar and vector leptoquarks in the mBRW model.}
    \label{tab:lq-num}
  \end{center}
\end{table}

As color-charged particles, leptoquarks are primarily produced by strong interactions in $\Pp\Pp$ collisions. For pair production of leptoquarks, these interactions include gluon-gluon fusion and quark-antiquark annihilation, whose LO forms are shown in Fig. \ref{fig:lq-diagrams}. An additional contribution to quark-antiquark annihilation may proceed through the Yukawa coupling $\lambda$ of the leptoquark to the quark and lepton pair. However, the ratio $\MLQ/\lambda$, where $\MLQ$ is the leptoquark mass, is restricted by limits from low-energy processes including $\pi^{+} \rightarrow \Pe^{+} \nu_{\Pe}$ and atomic parity violation. The limits on $\MLQ/\lambda$ range from 1800--6400\GeVcc, depending on the type of leptoquark \cite{Leurer:1993em, MuchAdo, LQreview}. The expected accessible mass range for leptoquarks at the LHC with $\sqrt{s}=14\TeV$ ranges from 900--1200\GeVcc for scalar LQs and 1200--1500\GeVcc for vector LQs, depending on the desired number of events \cite{LQPairHad}. Up to these masses, $\lambda$ will be small enough that its contribution to leptoquark production can be neglected. This applies both to the Yukawa-based pair production diagram in Fig. \ref{fig:lq-diagrams} and the single production diagrams from quark-gluon scattering in Fig. \ref{fig:lq-single}.

\begin{figure}[hbt]
\begin{center}
\includegraphics[width=0.49\textwidth]{figures/LO_FD_LQ_pair_a.pdf}
\includegraphics[width=0.49\textwidth]{figures/LO_FD_LQ_pair_b.pdf}
\includegraphics[width=0.49\textwidth]{figures/LO_FD_LQ_pair_c.pdf}
\includegraphics[width=0.49\textwidth]{figures/LO_FD_LQ_pair_d.pdf}
\includegraphics[width=0.49\textwidth]{figures/LO_FD_LQ_pair_e.pdf}
\includegraphics[width=0.49\textwidth]{figures/LO_FD_LQ_pair_f.pdf}
\caption{The LO diagrams for leptoquark pair production from quark-antiquark annihilation (top) and gluon-gluon fusion (middle, bottom).}
\label{fig:lq-diagrams}
\end{center}
\end{figure}

\begin{figure}[hbt]
\begin{center}
\includegraphics[width=0.49\textwidth]{figures/LO_FD_single_LQ_a.pdf}
\includegraphics[width=0.49\textwidth]{figures/LO_FD_single_LQ_b.pdf}
\caption{The LO diagrams for single leptoquark production from quark-gluon scattering. The LQ is produced in association with a lepton.}
\label{fig:lq-single}
\end{center}
\end{figure}

Table \ref{tab:lq-xsec} lists the pair production cross sections for a range of scalar leptoquark masses. They have been calculated using the CTEQ6 parton distribution functions (PDFs) \cite{CTEQ6r1,CTEQ6r2} with $K$-factors applied to include NLO corrections from QCD \cite{LQxsec}. Theoretical uncertainties are calculated by propagating the PDF uncertainty and varying the factorization/renormalization scale $\mu$. These cross sections are sensitive only to the leptoquark mass and spin, so they are largely model-independent. For vector leptoquarks, the cross sections may be modified by anomalous triple and quartic gauge couplings. The decay widths for scalar and vector leptoquarks can be calculated according to Eqs. \eqref{eq:SLQ-width} and \eqref{eq:VLQ-width}, respectively \cite{BRW}:
\begin{align}
\label{eq:SLQ-width} \Gamma_{S} &= \sum_{i}{\frac{\lambda_{i}^{2}}{16\pi}\MLQ}, \\
\label{eq:VLQ-width} \Gamma_{V} &= \sum_{i}{\frac{\lambda_{i}^{2}}{24\pi}\MLQ}.
\end{align}
Eqs. \eqref{eq:SLQ-width} and \eqref{eq:VLQ-width} sum over all Yukawa couplings for a given leptoquark. Given the limits on $\lambda$ discussed above, leptoquarks accessible at the LHC with a single Yukawa coupling can be expected to have a fractional decay width of less than 0.1--0.2\%.

\begin{table}[htb]
\begin{center}
{\footnotesize
\begin{tabular}{|l||c|c||c|c|}
\hline
$\MLQ$ & $\sigma (\mu = \MLQ)$ & $\delta (\text{PDF}) $ & $\sigma (\mu = \MLQ/2)$ & $\sigma (\mu = 2\MLQ)$ \\
\hline
\hline
 200 & 17.4 & 1.24 & 15.0 & 19.7  \\
 250 & 5.26 & 0.487 & 4.54 & 5.94  \\
 300 & 1.89 & 0.214 & 1.63 & 2.13  \\
 350 & 0.77 & 0.102 & 0.663 & 0.866  \\
 400 & 0.342 & 0.052 & 0.295 & 0.385  \\
 450 & 0.163 & 0.0278 & 0.14 & 0.183  \\
 500 & 0.082 & 0.0155 & 0.0704 & 0.0922  \\
 550 & 0.0431 & 0.00893 & 0.037 & 0.0485  \\
 600 & 0.0235 & 0.0053 & 0.0201 & 0.0265  \\
 650 & 0.0132 & 0.00322 & 0.0113 & 0.0149  \\
 700 & 0.00761 & 0.002 & 0.00648 & 0.00858  \\
 750 & 0.00448 & 0.00126 & 0.00381 & 0.00506  \\
 800 & 0.00269 & 0.00081 & 0.00228 & 0.00304  \\
 850 & 0.00164 & 0.000527 & 0.00139 & 0.00186  \\
 900 & 0.00101 & 0.000347 & 0.000856 & 0.00115  \\
 950 & 0.000634 & 0.000231 & 0.000534 & 0.000722  \\
 1000 & 0.000401 & 0.000155 & 0.000337 & 0.000458  \\
\hline
\end{tabular}
}
\caption{The pair production cross sections for a range of scalar leptoquark masses at $\sqrt{s}=8\TeV$. Theoretical uncertainties from the PDFs and from varying the factorization/renormalization scale $\mu$ from $\mu=\MLQ/2$ to $\mu=2\MLQ$ are indicated.}
\label{tab:lq-xsec}
\end{center}
\end{table}

In this dissertation, a search is performed for pair production of scalar leptoquarks decaying to third generation fermions. The symbol $B$ is used for the branching fraction for the decay $\text{LQ} \rightarrow \tau \cPqb$. This results a final state with two tau leptons and two bottom quarks. One tau lepton is required to decay leptonically: $\tau \rightarrow \ell \overline{\nu_{\ell}} \nu_{\tau}$, where $\ell$ can be a muon or an electron, which are collectively called light leptons. The other tau lepton is required to decay hadronically, denoted as \tauh; see Sec. \ref{sec:hpstau} for more information about hadronic decays of tau leptons. These decays result in two channels based on the leptonic decay of the tau, which are labeled as \etau and \mutau, or collectively \ltau when the light lepton flavor is unimportant. Both bottom quarks hadronize into b-jets, as described in Sec. \ref{sec:b-tagging}. Currently, the strongest mass limits on such third-generation scalar leptoquarks come from direct searches. Assuming $B=100\%$, the lower limit is approximately 530\GeV, set by both the CMS \cite{CMSLQ3} and ATLAS \cite{ATLASLQ3} experiments using 4.7--4.8\fbinv of data from $\Pp\Pp$ collisions with $\sqrt{s}=7\TeV$. Indirect limits from low-energy processes are discussed in Refs. \cite{ModelIndLQ,Leurer:1993em, MuchAdo, LQreview}.

%talk about reinterpreted search to cover low B?

\section{R-Parity Violating Supersymmetry}
% input: [cmsexperiment.tex]
\chapter{Compact Muon Solenoid Experiment
\label{ch:cmsexperiment}}
\setcounter{section}{-1}

The Compact Muon Solenoid experiment is one of two general-purpose detectors at the LHC. It is located about 100\unit{m} underground at Point 5 of the LHC ring, near Cessy, France. The detector is cylindrically shaped, with total length 22\unit{m}, diameter 15\unit{m}, and weight 14000\unit{tons}. Figure \ref{fig:cms-overall} shows the overall layout of the detector.

\begin{figure}[hbt]
\begin{center}
\includegraphics[width=0.95\textwidth]{figures/cms_complete_labelled.pdf}
\caption{The layout of the CMS detector, with the subdetectors labeled and two humans shown for a height reference.}
\label{fig:cms-overall}
\end{center}
\end{figure}

The center of the detector, the interaction point (IP), is used as the origin of the right-handed coordinate system that describes locations and directions within the detector. The $z$-axis is assigned to the direction of the LHC beam line, pointing anti-clockwise in the direction of Beam 2. The polar angle $\theta$ is defined as the angle away from the positive $z$-axis. This angle is often transformed into pseudorapidity, defined as $\eta = -\text{ln}[\text{tan}(\theta/2)]$, which has several desirable properties. It is independent of the particle mass and energy, and it is approximately equal to the rapidity for relativistic particles. Differences in pseudorapidity are Lorentz invariant for boosts in the $z$ direction, and particle production is approximately uniform in $\eta$.

The plane transverse to the $z$-axis comprises the $x$- and $y$-axes, with the $x$-axis pointing toward the center of the LHC ring and the $y$-axis pointing upward in the normal direction. The azimuthal angle $\phi$ is defined as the angle away from the positive $x$-axis in the transverse plane, and the radial coordinate $r$ is defined as the distance from the origin in the transverse plane. The magnitude of the component of momentum in the transverse plane is labeled \pt. The total transverse momentum of every event must be conserved, so the negative vector sum of \vecpt for all particles in the event is defined as the missing transverse momentum: $\vecmet = - \sum_{i} \vecpt^{(i)}$. The magnitude of this quantity is called the missing transverse energy, denoted as \met.

As a general purpose detector, the CMS experiment detects all long-lived SM particles, except neutrinos, which can only be measured by omission in the transverse plane via \met. These particles can be categorized as electrons, photons, muons, charged hadrons, and neutral hadrons. The latter two categories are usually found grouped into cones called jets, and can originate from gluons, light quarks, bottom quarks with displaced vertices, or hadronically decaying tau leptons. The identification of particles and related objects with the CMS detector is described in more detail in Ch. \ref{ch:reconstruction}.

The CMS detector must measure these particles with sufficient accuracy to accomplish the experiment's physics goals, imposing requirements which are met by the combination of the CMS subdetectors. The inner tracker measures event vertices and charged-particle momentum, with the help of the superconducting solenoid. The measurement of muon momentum is supplemented by the muon system. The electromagnetic calorimeter measures electron and photon energy, and the hadron calorimeter measures the energy from charged and neutral hadrons. In addition, the LHC operates at a high instantaneous luminosity, approaching $10^{34}\percms$; with an expected proton-proton cross section of 100\unit{mb} at the LHC center-of-mass energies, the collision rate is approximately 1\unit{MHz}. To cope with this incredibly high collision rate, the CMS experiment employs a trigger system to select interesting events at a rate slow enough to be stored and processed by the computing systems. The delivered luminosity is also measured by the detector, using several different techniques. The following sections describe the LHC, based on Ref. \cite{LHCmachine}, and the CMS subdetector systems, based on Ref. \cite{CMSJINST}.

\section{The Large Hadron Collider}

The Large Hadron Collider is the largest machine ever built and the highest-energy collider in the world. It uses the tunnel originally constructed for the Large Electron-Positron Collider (LEP) with a circumference of 26.7\unit{km}. The tunnel is located underground in Switzerland and France, near Geneva. Figure \ref{fig:lhc-diagram} shows a diagram of the LHC with the major experiments labeled. Opposite from CMS is A Toroidal LHC ApparatuS (ATLAS) at Point 1, the other general purpose detector. To the right of ATLAS at Point 8 is the LHC beauty (LHCb) experiment, which studies flavor physics. To the left of ATLAS at Point 2 is A Large Ion Collider Experiment (ALICE), which studies heavy ion physics in Pb-Pb and p-Pb collisions.

\begin{figure}[hbt]
\begin{center}
\includegraphics[width=0.95\textwidth]{figures/lhc-pho-1997-060.png}
\caption{A diagram of the LHC with the major experiments labeled \cite{Jean-Luc:841573}.}
\label{fig:lhc-diagram}
\end{center}
\end{figure}

The LHC is designed to accelerate two beams of protons up to energies of 7\TeV each, at instantaneous luminosities up to $10^{34}\percms$. It is also designed to accelerate lead ions up to energies of 1.38\TeV per nucleon, at instantaneous luminosities up to $10^{27}\percms$. The LHC can achieve energies several orders of magnitude higher than LEP in the same tunnel by using protons instead of electrons; the larger mass of protons reduces losses from synchrotron radiation by $(m_{\Pp}/m_{\Pe})^{4} = 1836^{4} = 1.136\times10^{13}$. The use of supercooled superconducting magnets, discussed below, is also crucial. Several stages of CERN accelerators are used to inject proton beams into the LHC, as shown in Fig. \ref{fig:lhc-injectors}. These include the linear accelerator Linac2, the Proton Synchrotron Booster (PSB), the Proton Synchrotron (PS), and the Super Proton Synchrotron (SPS). For lead ions, Linac3 and the Low Energy Ion Ring (LEIR), repurposed from the Low Energy Antiproton Ring (LEAR), are used instead of Linac2 and the PSB, respectively. The accelerated protons are grouped into bunches using radio frequency (RF) electromagnetic fields. The LHC is designed to accommodate a bunch spacing of 25\unit{ns}.

\begin{figure}[hbt]
\begin{center}
\includegraphics[width=0.95\textwidth]{figures/lhc-pho-1993-008.png}
\caption{A diagram of the CERN accelerators which form the LHC injector \cite{Jean-Luc:841568}.}
\label{fig:lhc-injectors}
\end{center}
\end{figure}

Due to size limitations in the tunnel, the two rings used to accelerate the two proton beams are formed by twin bore magnets. Each magnet has a single mechanical structure and cryostat, in which are placed two coils and two beam channels. The dipole magnet coils use superconducting NbTi Rutherford cables cooled to 1.9\unit{K}, as shown in Fig. \ref{fig:lhc-dipole}, with a design field strength of 8.33\unit{T} for acceleration of protons up to 7\TeV. This extreme cooling is accomplished using superfluid helium. In total, the LHC contains 1232 dipole magnets. Thousands of quadrupole, sextupole, octupole, and decapole magnets are used to correct and focus the beam.

\begin{figure}[hbt]
\begin{center}
\includegraphics[width=0.95\textwidth]{figures/lhc-pho-1998-299.jpg}
\caption{A diagram of an LHC dipole magnet, with the major components labeled \cite{Dailler:842253}.}
\label{fig:lhc-dipole}
\end{center}
\end{figure}

In 2012, the LHC accelerated proton beams to energies of 4\TeV each, with a peak instantaneous luminosity of $7.67\times10^{33}\percms$ and a bunch spacing of 50\unit{ns}. During that year, it delivered 23.30\fbinv of integrated luminosity to the CMS detector, of which 21.79\fbinv was recorded \cite{LumiPublic}. In the upcoming 2015 run, the LHC will achieve its design energy, instantaneous luminosity, and bunch spacing.

\section{Tracker}
\label{sec:tracker}

The CMS tracker is the first subdetector to measure charged particles produced in collisions at the IP. It is 5.8\unit{m} long and 2.5\unit{m} in diameter, covering the pseudorapidity range $-2.5 < \eta < 2.5$. Two subsystems make up the tracker: the pixel detector and the silicon strip tracker. The layout of the tracker, with these subsystems labeled, is shown in Fig. \ref{fig:tk-layout}. Due to the tracker's location close to the IP, it experiences severe radiation doses that are expected to range from 0.18 to 84\unit{Mrad} after 500\fbinv of data. Hence, the tracker must be robust against radiation damage, requiring operation at $-10\degC$ and influencing the design of the sensors and electronics. For tracks with momentum of 100\GeV, the tracker has a transverse momentum resolution of 1--2\% for $|\eta|<1.6$; at higher $\eta$, the reduced transverse depth of the tracker degrades the resolution.

\begin{figure}[hbt]
\begin{center}
\includegraphics[width=0.95\textwidth]{figures/CMS_tracker.pdf}
\caption{The layout of the CMS tracker, with subsystems labeled.}
\label{fig:tk-layout}
\end{center}
\end{figure}

The pixel detector is the innermost portion of the tracker. It consists of three barrel layers, collectively called BPIX, and two endcap layers, called FPIX. Each pixel cell is a hybrid silicon detector with dimensions $100\times150\mum^{2}$. The small pixel size enables precise track resolutions of 10\mum in the $r$-$\phi$ direction and 20\mum in the $z$ direction. In total, the BPIX comprises 48 million pixels and the FPIX comprises 18 million pixels. The pixel detector is important for many key components of CMS physics analysis. These include the reconstruction of secondary vertices from decays of tau leptons and bottom quarks, as well as producing seed tracks for the strip tracker and the high level trigger.

The silicon strip detector consists of four subsystems. The Tracker Inner Barrel (TIB) has four layers with the three-layer Tracker Inner Disks (TID) at each end; both systems' strips are 320\mum thick. Surrounding the TIB/TID is the Tracker Outer Barrel (TOB), which has six layers. The first four layers of the TOB use 500\mum thick strips, while the last two layers use 122\mum thick strips. The Tracker EndCaps (TEC) have nine disks with up to seven layers of strips, 320\mum thick in the inner four rings and 500\mum thick in the outer three rings. In total, all of these layers contain 9.3 million silicon strips.

The tracker maintained excellent performance during the 2012 run of the LHC. The pixel detector had 97.7\% of channels operational in BPIX and 92.8\% of channels operational in FPIX, while 97.5\% of channels in the strip detector were active. The hit reconstruction efficiencies were greater than 99\% for each layer of the strip detector and greater than 99.5\% for each layer of the pixel except for the first layer of BPIX, which had an efficiency greater than 99.2\% \cite{Veszpremi:2014hpa}. 

\section{Electromagnetic Calorimeter}

The electromagnetic calorimeter (ECAL) is a homogeneous calorimeter constructed entirely of lead tungstate (\pbwo) crystals. The ECAL is divided into two subsystems: the ECAL barrel (EB) and the ECAL endcap (EE). In the endcap region, there is an additional ECAL preshower (ES) detector in front of the EE. Figure \ref{fig:ecal-layout} displays these subsystems. \pbwo has a peak emission wavelength of 425\unit{nm} and many desirable material properties. These properties include high density ($8.28\unit{g/cm}^3$), short radiation length (0.89\cm), short Moli\`{e}re radius (2.2\cm), and fast decay time (6\unit{ns}). The use of homogeneous \pbwo crystals enables precise energy resolution for electromagnetic objects. For photons with $\ET \approx 60\GeV$, the energy resolution ranges from 1.1\% to 2.6\% for the EB and 2.2\% to 5.0\% for the EE. In general, the energy resolution $\sigma$ varies as a function of energy $E$ in \GeVns:
\begin{equation}
\label{eq:ecal-res} \left(\frac{\sigma}{E}\right)^{2} = \left(\frac{S}{\sqrt{E}}\right)^{2} + \left(\frac{N}{E}\right)^{2} + C^{2}.
\end{equation}
In Eq. \eqref{eq:ecal-res}, $S$ is the stochastic term, $N$ is the noise term, and $C$ is the constant term. Typical values for these terms were measured by a test beam to be $S=2.8\%$, $N=12\%$, $C=0.30\%$.

\begin{figure}[hbt]
\begin{center}
\includegraphics[width=0.95\textwidth]{figures/ECAL_transverse_section.pdf}
\caption{A diagram of the CMS ECAL, with subsystems and $\eta$ ranges labeled.}
\label{fig:ecal-layout}
\end{center}
\end{figure}

The EB contains 61200 \pbwo crystals and covers the range $|\eta|<1.479$. The crystals are arranged in a projective geometry with a tapered shape. The crystal granularity is approximately $0.0174\times0.0174$ in $\eta$-$\phi$, corresponding to dimensions of $22\times22\mm^{2}$ at the front face and $26\times26\mm^{2}$ at the back face. The EB has a depth of 230\mm or 25.8 radiation lengths ($X_{0}$). The scintillation light produced by the \pbwo crystals is read out using avalanche photodiodes (APDs). At 18\degC, the APDs measure approximately 4.5 photoelectrons per \MeVns. The dark current of the APDs is sensitive to radiation exposure. Over the course of the 2012 run, the dark current ranged from 0.3--1.3\muA on average, corresponding to an average noise of 47--57\MeV \cite{CMS:2013ecal}.

The EE contains 14648 \pbwo crystals and covers the range $1.479<|\eta|<3.0$. The crystals are arranged in a non-projective $x$-$y$ geometry, with dimensions of $28.62\times28.62\mm^{2}$ at the front face and $30\times30\mm^{2}$ at the back face. The EE has a depth of 220\mm or 24.7$\,X_{0}$. Vacuum phototriodes (VPTs) are used as the photodetectors to read out the scintillation light from the \pbwo crystals. They collect approximately 4.5 photoelectrons per \MeVns at 18\degC. During the 2012 run, the average noise ranged from 180--220\MeV, with a more dramatic increase up to 600\MeV at high $\eta$ because of the high radiation dose \cite{CMS:2013ecal}.

The ES is intended to identify neutral pions in the endcap region, covering the range $1.653<|\eta|<2.6$. It is a sampling calorimeter with two layers of lead absorber and silicon strip detectors. The first layer of lead absorber has a thickness of 2$\,X_{0}$, while the second layer has a thickness of 1$\,X_{0}$. Each layer of silicon strips is 320\mum thick and can collect 3.6\unit{fC} of charge from a minimum ionizing particle.

%add percentage of live channels for EB and EE in 2012 run? and HE raddam?

\section{Hadron Calorimeter}

The hadron calorimeter (HCAL) is a sampling calorimeter which measures the energy of hadronic particles. The HCAL is especially important for measuring neutral hadrons, which do not leave tracks in the tracker. In addition, by containing all hadronic activity in each event within $|n|<5$, the HCAL enables the measurement of \met caused by neutrinos and other theoretical weakly-interacting particles. The HCAL consists of four subsystems. Three of these subsystems use similar technology: the HCAL barrel (HB), the HCAL endcap (HE), and the HCAL outer (HO). The fourth subsystem, the HCAL forward (HF), uses an alternative technology necessary to survive the high radiation doses at its forward location. The locations of the various HCAL subsystems in the CMS detector are shown in Fig. \ref{fig:hcal-layout}. The calorimeter system, combining the ECAL and the HCAL, can measure jets with a resolution $\sigma/E \approx 100\% / \sqrt{E\,[\GeVns]} \oplus 5\%$ that varies with the jet energy $E$.

\begin{figure}[hbt]
\begin{center}
\includegraphics[width=0.95\textwidth]{figures/HCAL_subdet.pdf}
\caption{The layout of the HCAL subsystems HB, HE, HO, and HF in the CMS detector.}
\label{fig:hcal-layout}
\end{center}
\end{figure}

The HB is a 16-layer sampling calorimeter covering the range $|\eta|<1.3$. The absorbing layers are made of C26000 cartridge brass, composed of 70\% copper and 30\% zinc. Cartridge brass has a density of $8.53\unit{g/cm}^3$, a radiation length of 1.49\cm, and a nuclear interaction length of 16.42\cm. The first absorbing layer in the HB is a 40-mm-thick steel plate. The next eight absorbing layers are 50.5-mm-thick brass plates, and the subsequent six absorbing layers are 56.5-mm-thick brass plates. The last absorbing layer is a 75-mm-thick steel plate. The overall thickness of the HB absorber ranges from 5.82$\,\lambda_{0}$ at $\eta=0$ to 10.6$\,\lambda_{0}$ at $|\eta|=1.3$. The EB in front of the HB has a thickness of 1.1$\,\lambda_{0}$ and can measure the electromagnetic portions of early developing hadronic showers. The scintillating layers consist of 3.7-mm-thick Kuraray SCSN81 plastic scintillator, a polystyrene base doped with fluors. The exception is Layer 16, which has a thickness of 9\mm, in order to sample more from late developing hadronic showers. At the front of the HB, before the first absorbing layer, is the scintillator Layer 0, which is 9\mm of Bicron BC408 plastic scintillator, a polyvinyltoluene base doped with fluors. Layer 0 samples the energy deposited by hadronic showers in the dead material between the EB and the HB. The scintillator tiles are arranged in a projective geometry with a granularity of $0.087\times0.087$ in $\eta$-$\phi$. In total, the HB has 16 $\eta$ divisions called towers, 36 $\phi$ divisions, and approximately 70000 tiles. The light from the scintillators is collected by Kuraray Y-11 green wavelength shifting (WLS) fiber, which is placed in a $\sigma$-shaped groove in the scintillator tiles. The wavelength-shifted light from multiple layers is brought together and read out by hybrid photodiodes (HPDs). These photodetectors were chosen because of their large dynamical range and low sensitivity to magnetic fields.

The thinness of the HB at low $\eta$ prevents it from fully containing hadronic showers, so the HO is added to act as an extension of the calorimeter system. The HO uses the same scintillator tile technology as the HB: 3.7-mm-thick SCSN81 with Y-11 WLS fiber and granularity $0.087\times0.087$ in $\eta$-$\phi$, read out by HPDs. The HO is divided into five rings, each with a width of 2.536\unit{m} in the $z$ direction, based on the structure of the iron return yoke outside of the solenoid. In the central Ring 0, the HO has two scintillating layers, one inside the solenoid and one outside of it. In the other rings, the HO has one scintillating layer outside of the solenoid. The thickness of the absorbing iron layer formed by the solenoid is 19.5\cm, extending the total depth of the calorimeter system to a minimum of 11.8$\,\lambda_{0}$.

The HE is a 17-layer sampling calorimeter covering the range $1.3<|\eta|<3.0$. Each absorbing layer consists of 79-mm-thick cartridge brass, the same material used for the HB absorbing layers. The scintillating layers use the same technology as the HB and the HO. In total, the HE contains 20916 scintillator tiles. The granularity of the tiles is the same as HB for $|eta|<1.6$; for higher $\eta$, they become coarser at approximately $0.17\times0.17$ in $\eta$-$\phi$. Unlike the HB, the scintillating layers in each tower are split into multiple groups called depths before being read out by HPDs. A diagram of the depth segmentation scheme is shown in Fig. \ref{fig:hcal-depths}. This depth segmentation allows for more precise recalibration of the HE, which experiences a higher radiation dose than the HB. Towers 27, 28, and 29, which are the closest to the beamline, have three readout depths, while the other towers have two readout depths. The crossover region between the HB and the HE, towers 15 and 16, also utilize depth segmentation. As in the HB, Layer 0 in the HE consists of 9-mm-thick BC408 to sample from the dead material between the EE and the HE. The combined calorimeter system, including both the EE and the HE, has an approximate thickness of 10$\,\lambda_{0}$.

The HF covers the range $3.0<|\eta|<5.0$, with no ECAL in front of it. It consists of a steel absorber structure with a thickness of 165\cm or 10$\,\lambda_{0}$. Polymer-cladded quartz fibers with diameter 800\mum are embedded in the steel absorber. The fibers are bundled together to form 13 towers in a non-projective $x$-$y$ geometry with granularity $0.175\times0.175$ in $\eta$-$\phi$. Over 1000\unit{km} of fiber is used in the HF. Half of the fibers run for the full 165\cm depth of the detector, while the other half start 22\cm into the detector. Each type of fiber is read out separately in order to distinguish electromagnetic showers from hadronic showers. The fibers measure particle showers using Cherenkov light, which is read out by photomultiplier tubes (PMTs). They measure approximately 1 photoelectron for every 4\GeV of energy deposited. This alternative design was necessary to ensure the radiation hardness of the HF, parts of which can experience 100\unit{Mrad/year}.

\begin{figure}[hbt]
\begin{center}
\includegraphics[width=0.95\textwidth]{figures/HCAL_tower_segmentation.pdf}
\caption{A diagram of the depth segmentation scheme in the HB and the HE.}
\label{fig:hcal-depths}
\end{center}
\end{figure}

%add percentage of live channels for HCAL in 2012 run?

\section{Solenoid}

The superconducting solenoid is the central feature of the CMS detector. The solenoid provides a magnetic field of 3.8\unit{T} within the volume formed by its diameter of 6\unit{m} and length of 12.5\unit{m}. This strong magnet field is necessary so that high energy charged particles bend sufficiently for the tracker to measure their momenta accurately. At full current, the solenoid has a stored energy of 2.35\unit{GJ}. The magnet is constructed from a 4-layer winding of reinforced NbTi conductor, cooled to 4.5\unit{K}. It is split into five rings of equal length. The cold mass of the magnet is 220 tons, and the high ratio between the stored energy and the cold mass, 11.6\unit{KJ/kg}, causes a significant mechanical deformation of 0.15\% when the magnet is powered. Figure \ref{fig:solenoid} shows an artistic rendering of the solenoid.

\begin{figure}[hbt]
\begin{center}
\includegraphics[width=0.95\textwidth]{figures/CMS_solenoid.jpg}
\caption{An artistic rendering of the CMS solenoid, showing the five rings placed inside the cryostat, along with the support structure.}
\label{fig:solenoid}
\end{center}
\end{figure}

\section{Muon System}
\label{sec:muon-system}

The identification and measurement of muons is a major focus of the CMS experiment. The CMS muon system comprises three subsystems, each utilizing different gaseous particle detection technology. In the barrel region, drift tubes (DTs) are used. In the endcap region, cathode strip chambers (CSCs) are used. Resistive plate chambers (RPCs) are also used in both regions. The muon systems are built into the iron yoke, which consists of five barrel rings and six endcap disks weighing 10000 tons in total. The yoke confines the outer magnetic field from the return flux from the solenoid and absorbs stray hadrons. The layout of the muon system is shown in Fig. \ref{fig:muon-system}. The momentum resolution of the muon system by itself is approximately 9\% for $\pt<200\GeV$ with low $\eta$ and low $p$. For 1\TeV muons, the resolution varies between 15\% and 40\%, depending on $|\eta|$. When the muon system measurements are combined with measurements from the tracker, the 1\TeV muon resolution is improved to 5\%.

\begin{figure}[hbt]
\begin{center}
\includegraphics[width=0.95\textwidth]{figures/CMS_muon_system.pdf}
\caption{The layout of the muon system, with the three subsystems labeled.}
\label{fig:muon-system}
\end{center}
\end{figure}

The DTs are divided into four stations, which together cover the range $|\eta|<1.2$ and are labeled MB1 through MB4 (Muon Barrel). The first three stations each contain twelve chambers divided into three groups of four. Two of the groups of four measure the $r$-$\phi$ coordinate of muons, while the third group of four measures the $z$ coordinate. MB4 does not include a group of chambers that measures $z$. All four stations together contain 250 DTs with a total of 172000 sensitive wires. The gas used in the DTs is a mixture of 85\% Ar and 15\% $\text{CO}_2$, and the anode wires are gold-plated stainless steel with a diameter of 50\mum. Within $|\eta|<0.8$, the MB stations alone can reconstruct high-\pt muon tracks with an efficiency exceeding 95\%. The global $r$-$\phi$ resolution is 100\mum.

The CSCs are also divided into four stations and cover the range $0.9<|\eta|<2.4$. The four stations are labeled ME1 through ME4 (Muon Endcap). Each station is divided into several groups as follows: ME1 has three groups of 72 CSCs; ME2 and ME3 each have one group of 36 CSCs and another group of 72 CSCs; ME4 has one group of 36 CSCs. The total number of CSCs is thus 468. The cathode strips are arranged in the radial direction to measure the $r$-$\phi$ coordinate, while the anode wires are perpendicular to the strips to measure $\eta$. There are approximately 220000 cathode strip readout channels and 180000 anode wire readout channels. The CSC gas mixture is set at 40\% AR, 50\% $\text{CO}_2$, and 10\% $\text{CF}_4$. The cathode strips are formed from a fiberglass/epoxy material called FR4, coated with 36-$\mu$m-thick copper. The anode wires are made of gold-plated tungsten with a diameter of 50\mum. The first group of ME1 CSCs uses slightly thinner wire with 30\mum diameter and has some other slightly modified properties.

To complement the DTs and CSCs, RPCs are installed in both the barrel and endcap regions, covering the range $|\eta|<1.6$. The RPCs are primarily used to provide muon trigger information, due to fast tagging capabilities which allow them to precisely identify the bunch crossing time of muon candidate events. MB1 and MB2 each have one internal and one external group of RPCs, relative to the DTs; MB3 and MB4 each have two internal groups of RPCs. These groups together comprise 480 chambers. In the endcap, there are three RPC stations mounted in concentric circles on the iron yoke disks, with a total of 144 chambers. The RPCs are parallel plate detectors filled with a gas mixture of 96.2\% $\text{C}_2\text{H}_2\text{F}_4$, 3.5\% $i\text{C}_4\text{H}_{10}$, and 0.3\% $\text{SF}_6$.

%add percentage of live channels for muon system in 2012 run?

\section{Trigger}

The CMS trigger is necessary to reduce the roughly 1\unit{MHz} rate of collision events produced by the LHC to a rate which can be stored and processed. The trigger system consists of two stages. The first stage uses hardware and is called the Level-1 (L1) Trigger. The L1 Trigger is designed to have a maximum output rate of 100\unit{kHz}. The second stage is the High Level Trigger (HLT), which uses software and reduces the output rate to $\mathcal{O}(100\unit{Hz})$.

The L1 Trigger uses custom-built programmable electronics, including field-programmable gate arrays (FPGAs), application-specific integrated circuits (ASICs), and memory lookup tables (LUTs). All of the subdetectors send input to the L1 Trigger, which is organized into local, regional, and global components as shown in Fig. \ref{fig:L1-trigger}. The local components, Trigger Primitive Generators (TPGs), are constructed from energy deposits in the calorimeters and track segments or hit patters from the muon system.

The TPGs from the ECAL, the HCAL, and the HF are combined into the Regional Calorimeter Trigger (RCT). The RCT groups calorimeter trigger towers into regions, which are made up of four towers in the barrel and endcap, and one tower in the HF. These regions are used to determine electron and photon candidates, as well as transverse energy sums (\sumet) and tau-jet vetoes. The RCT also passes MIP and isolation information to the muon triggers. Using information from the RCT, the Global Calorimeter Trigger (GCT) determines jet candidates and counts, providing up to four jets and four tau-jets from the central HCAL and four jets from the HF. The GCT also determines total \ET, \met, and \HT, which is calculated as \sumet for all jets above a certain threshold.

In parallel, the muon DT, CSC, and RPC systems each produce their own local triggers. The Regional Muon Trigger (RMT) contains the DT and CSC Track Finders (DTTF, CSCTF) which make tracks using segments from their respective subdetectors. As mentioned in Sec. \ref{sec:muon-system}, the RPCs act as a dedicated trigger using their timing resolution of 1\unit{ns} to determine bunch crossing times. The Global Muon Trigger (GMT) combines the information from the RMT and RPCs to produce up to four muon candidates in each of the barrel and endcap regions. These candidates include the following information: \pt, charge, $\eta$, $\phi$, a quality code, MIP, and isolation.

Finally, the Global Trigger (GT) combines the GCT and GMT candidates and quantities to decide whether or not to keep the event, based on a set of L1 triggers with different criteria. The GT also uses information from the Trigger Control System (TCS) regarding the status of the subdetector readout and data acquisition systems. The Timing, Trigger, and Control (TTC) system is used to return the GT decision, called the Level-1 Accept (L1A), to the various subdetectors. This entire process is completed within 3.2\mus.

\begin{figure}[hbt]
\begin{center}
\includegraphics[width=0.95\textwidth]{figures/L1_architecture.png}
\caption{The architecture of the L1 Trigger.}
\label{fig:L1-trigger}
\end{center}
\end{figure}

The HLT further analyzes events which pass the L1A decision. Using a farm of roughly 1000 commercial processors comprised of over 13000 central processing units (CPUs), it approximately emulates the full offline reconstruction algorithms described in Ch. \ref{ch:reconstruction}. Like the L1 Trigger, the HLT uses a set of triggers with different criteria, called the trigger menu. Different trigger menus are constructed for various conditions, including different instantaneous luminosity levels and different types of collisions or measurements. The selected menu can be changed during the operation of the detector in response to new conditions. Events which pass the HLT decision are sorted into primary datasets (PDs) with minimal overlap. The HLT output includes several streams, including monitoring and calibration streams in addition to the primary stream of physics events.

During the 2012 run, the L1 Trigger operated at rates up to 100\unit{kHz} with only 3\% dead time \cite{Brooke:2013hnf}. The HLT operated at rates up to 1\unit{kHz} and took an average of 200\unit{ms} to process an event \cite{Trocino:2014jya}. This processing speed is two orders of magnitude faster than the full offline reconstruction. Around 600\unit{Hz} of HLT output data was ``parked'', or recorded to disk in a raw format before being processed. This data parking scheme took advantage of the impending 2013-2014 long shutdown for collider and detector upgrades, when CPUs would be available to process the extra data. The parked data was collected using alternative trigger menus designed to enable precision measurements and new physics searches in unexplored kinematic regions. CMS also engaged in ``data scouting'', performing simple analyses on partially-reconstructed events, for example keeping only HLT-reconstructed jets, from processes which would normally have rates too high to be accepted by the trigger. These simple analyses searched for noticeable deviations from SM predictions in order to indicate possible advantageous changes to the trigger menus.

\section{Luminosity Measurement}

The fine resolution of the CMS pixel detector (Sec. \ref{sec:tracker}) implies that a given pixel will tend to be activated by one track at most per bunch crossing. A minimum bias interaction creates an average of 200 clusters, with each cluster containing an average of 5 pixels \cite{CMS-PAS-LUM-12-001}. Even with 100 pileup events per bunch crossing, the pixel detector will have an occupancy as low as 0.1\%. The number of pixel hits should thus scale linearly with the number of interactions per bunch crossing for instantaneous luminosities up to and even beyond the LHC design performance. Equation \eqref{eq:pixel-lumi} shows how the instantaneous luminosity $\mathcal{L}$ is related to the average number of pixel clusters per event $\langle n \rangle$ \cite{CMS-PAS-LUM-13-001}:
\begin{equation} \label{eq:pixel-lumi}
\mathcal{L} = \frac{\nu \langle n \rangle}{\sigma_\text{vis}}.
\end{equation}
Here, $\nu = 11246\unit{Hz}$ is the LHC revolution frequency, $\langle n \rangle$ is defined as $\mu n_{1}$ where $\mu$ is the number of collisions per bunch crossing and $n_{1}$ is the average number of clusters per collision, and the visible cross section $\sigma_\text{vis}$ is defined as $\sigma_\text{T} n_{1}$ where $\sigma_\text{T}$ is the total inelastic cross section. A Van der Meer scan is used to calibrate $\sigma_\text{vis}$. In 2012, CMS measured the total integrated luminosity with a systematic uncertainty of 2.6\% using this method.

The HF is used as a second method of measuring the luminosity. This is possible because the HF can be run safely during unstable beams \cite{CMS-PAS-LUM-13-001}. Information from the HF can be used to measure the luminosity in two different ways. The average fraction of empty towers can be related to the mean number of interactions per crossing, or the average transverse energy per tower can be linearly related to the luminosity. It can make an online determination of the average luminosity to a statistical uncertainty of 1\% in under 1\unit{s}. However, the calibration of this measurement can drift over long time periods due to changes in the gain of the HF PMTs. In practice, the increase in pileup interactions observed during the 2012 run moves the HF response into a nonlinear regime, limiting the accuracy of this method. Because of these limitations, the HF method is utilized primarily as a cross-check for the pixel cluster counting method.
% input: [reconstruction.tex]
\chapter{Event Reconstruction
\label{ch:reconstruction}}

Particles created in proton-proton collisions pass through the CMS detector and leave signals in different subdetectors. Figure \ref{fig:cms-slice} shows examples of typical signals for different types of particles. Each type of particle has a different characteristic signature from which it can be identified using information from the various subdetectors. Muons, electrons, and charged hadrons create tracks in the tracker, while photons and neutral hadrons do not. Muons also create hits in the muon systems. Electrons and photons deposit energy in the ECAL, while charged and neutral hadrons deposit most of their energy in the HCAL.

\begin{figure}[hbt]
\begin{center}
\includegraphics[width=0.95\textwidth]{figures/CMS_slice.png}
\caption{A cross-sectional view of the CMS detector with all subdetectors labeled and examples of signals left by muons, electrons, charged hadrons, neutral hadrons, and photons \cite{CMS-slice}.}
\label{fig:cms-slice}
\end{center}
\end{figure}

The raw output from each subdetector is processed in several steps in order to reconstruct the different types of particles \cite{TDR-software}. The first step is local reconstruction, which involves the creation of reconstructed hits or ``RecHits'' for each subsystem of each subdetector. The tracker RecHits include information about the positions of signals in the form of clusters, which are combinations of contiguous strips or pixels. The muon system RecHits also provide the positions of signals. In the DT and CSC subsystems, these RecHits can be combined into three-dimensional track segments. The ECAL and HCAL RecHits contain the energy, position, and time of energy deposits from traversing particles.

In the second step, global reconstruction, the RecHits from the different subsystems of a given subdetector are combined and further processed. In the tracker, pattern recognition algorithms are employed to reconstruct tracks for various cases, including displaced vertices, low \pt tracks, and high \pt tracks. The ECAL and HCAL RecHits are combined into calorimeter towers or ``CaloTowers'' using a projective $\eta$-$\phi$ geometry. ``Standalone'' muons are created by the muon system global reconstruction, which associates RecHits and track segments in a radial trajectory, accounting for bending by the residual magnetic field, and uses a vertex-constrained fit.

High-level reconstruction is the final step, in which information from all subdetectors is used to reconstruct various types of particles as precisely as possible. The particle types used in this search include electrons, muons, taus, jets, and b-jets. The reconstruction algorithms for these particles will be described in more detail in the following sections of this chapter. Many of these algorithms use a particle flow technique that is unique to CMS and will be described in Sec. \ref{sec:particle-flow}.

The CMS experiment uses detailed simulations to predict the performance of the detector and reconstruction algorithms and to model various physics processes. Each simulated event is generated from the interaction of partons in proton-proton collisions and the decays of the resulting particles, based on theoretical calculations. The traversal of the final state particles through the CMS detector is simulated, and after some processing, the results can be treated identically to the raw data from real events and used as input for the reconstruction software.

\section{Event Generation}

Protons are most basically considered to be composed of two up quarks and one down quark. However, as a QCD bound state, it is more accurate to consider those three quarks as valence quarks, the most prominent features in a quark-gluon sea of virtual particles. At high energies such as those present in LHC collisions, the presence of additional quarks and gluons, collectively called partons, becomes significant. The fraction of the momentum of a proton $A$ carried by a parton $a$ is defined as $x_a$, and the parton distribution function (PDF) for that parton is $f_{a/A}(x_a,Q^2)$, where $Q^2$ is the momentum scale of the interaction, typically the square of the total four-momentum in a collision. PDFs are calculated using experimental data sets, with different groups taking different approaches to analyzing the data and modeling proton behavior. The CMS experiment primarily uses PDFs from Martin-Stirling-Thorne-Watt (MSTW) and the Coordinated Theoretical-Experimental Project on QCD (CTEQ). Figure \ref{fig:pdf-mstw} shows an example set of PDFs calculated at NLO by MSTW \cite{MSTW09}. At high momentum scales, even bottom quarks may be present in the quark-gluon sea of a proton. The dependence of the PDFs on $Q^2$ is given by the Dokshitzer-Gribov-Lipatov-Altarelli-Parisi (DGLAP) equations \cite{QuarkGluon}.

\begin{figure}[hbt]
\begin{center}
\includegraphics[width=0.95\textwidth]{figures/mstw2008nlo68cl_allpdfs.pdf}
\caption{PDFs calculated at NLO by MSTW, plotted against momentum fraction $x$ for two different values of the momentum scale $Q^2$ \cite{MSTW09}.}
\label{fig:pdf-mstw}
\end{center}
\end{figure}

The DGLAP equations allow large logarithmic factors, for example from collinear emission of gluons, to be included in the definition of the PDFs. This statement is the QCD factorization theorem, which applies to all hard-scattering processes. With this theorem, the interaction of parton $a$ in proton $A$ with parton $b$ in proton $B$, producing the final state $X$, can be written simply:
\begin{equation} \label{eq:hard-scatter}
\sigma_{AB \rightarrow X}(s) = \int{\text{dx}_{a}\text{dx}_{b} f_{a/A}(x_a,\mu_{F}^{2}) f_{b/A}(x_b,\mu_{F}^{2}) \hat{\sigma}_{ab \rightarrow X}(\hat{s},\mu_{R}^{2})}
\end{equation}
For such an interaction at high energy, the center-of-mass energy is $\sqrt{\hat{s}} = \sqrt{x_a x_b s}$, where $\sqrt{s}$ is the center-of-mass energy of the proton-proton system. There are two scales in Eq. \ref{eq:hard-scatter}: the factorization scale $\mu_{F}$ which separates long-distance physics from short-distance physics, and the renormalization scale $\mu_{R}$ of the QCD running coupling. Values for these scales are typically chosen to be characteristic of the hard scattering, with $\mu_{F} = \mu_{R}$. In addition to the primary hard-scattering process, the incoming protons and outgoing final state particles may radiate photons and gluons in processes called, respectively, initial state radiation (ISR) and final state radiation (FSR). The remnant partons which did not participate in the primary hard scatter may undergo soft interactions; these are collectively considered to be the ``underlying event''. All of these possible interactions in a proton-proton collision are visualized in Fig. \ref{fig:pp-collision}.

\begin{figure}[hbt]
\begin{center}
\includegraphics[width=0.95\textwidth]{figures/pp_collision_schematic.pdf}
\caption{An illustration of a proton-proton collision, showing the hard scattering, ISR, FSR, and the underlying event \cite{QuarkGluon}.}
\label{fig:pp-collision}
\end{center}
\end{figure}

Various event generation programs exist to simulate proton-proton collisions. The simulation of the hard scattering relies on a $2\rightarrow2$ matrix element to calculate $\hat{\sigma}_{ab \rightarrow X}(\hat{s},\mu_{R}^{2})$. Usually, the LO matrix element and PDFs are used for the simulation, and the results are scaled to NLO or NNLO using a $K$-factor derived from the ratio of the relevant cross sections. The program \PYTHIA \cite{Sjostrand:2006za} uses a parton showering approach to model ISR and FSR. Other generators such as \MADGRAPH \cite{MadGraph} use alternate approaches which more accurately simulate additional hard radiation outside of the primary hard scattering process, but must be combined with a program like \PYTHIA for showering from soft and collinear radiation. \POWHEG \cite{Alioli:2010xd} is a generator which uses NLO matrix elements and PDFs by matching them with parton shower contributions to prevent double counting. After the initial event generation, \MADGRAPH and \POWHEG are interfaced with \PYTHIA for hadronization. In addition, a specialized program called \TAUOLA \cite{TAUOLA} can be applied for accurate handling of tau lepton decays.

\section{Detector Simulation}

The response of the CMS detector to a proton-proton collision event is simulated using the stable final state particles output by event generators. The progression of each particle through the detector is tracked using the \GEANTfour software \cite{geant4nim,geant4ieee}. The geometry of every subdetector and subsystem is carefully simulated to ensure the accuracy of the simulation. The conditions of the real detector, including alignment and calibration changes, are measured periodically and stored in a database which can be used to configure the simulation. \GEANTfour includes customizable physics lists containing models of various physical processes to simulate the interactions of particles with matter. The detector simulation creates simulated hits or ``SimHits'' for each subdetector from the deposition of energy or charge based on those interactions. The effects of photodetectors and readout electronics on these SimHits are then simulated, mimicing the real subdetectors' data acquisition processes.

\section{Tracks and Vertices
\label{sec:tracks}}

Hits from the different tracker subsystems are reconstructed into charged-particle tracks using the Combinatorial Track Finder (CTF) algorithm \cite{TrackingJINST}. CTF is an iterative algorithm which finds the easiest tracks first, in order to remove the associated hits from consideration. This reduces the complexity of finding more difficult tracks in subsequent iterations. Each iteration follows the same four-step procedure, varying the type of seed used and the selection criteria applied.
\begin{enumerate}
\item A seed is generated using only a few hits.
\item Additional hits are added to the track based on the extrapolated trajectory of the seed.
\item The parameters of the track are estimated using a fit which considers all hits in the trajectory.
\item Selection criteria are applied to determine the quality of the track, and tracks which do not pass the selection are excluded.
\end{enumerate}

The types of seeds are categorized based on the number of hits included and the source of those hits. Initial iterations use pixel triplet and pair seeds, created from three and two pixel hits, respectively. These are the highest-quality seeds and are used to reconstruct prompt tracks, those emitting from primary vertices near the IP. A subsequent iteration uses a mixed triplet seed, containing 1--3 pixel hits and ${<}3$ strip hits. This iteration typically finds displaced tracks from heavy flavor decays, nuclear interactions, and photons which convert to \EpEm\xspace pairs in the tracker. The final iterations use strip pair seeds, consisting of two matched hits from the strip detectors, usually generated by charged particles which did not enter the pixel detector.

The iterations that use seeds with strip hits may also find prompt tracks which lack pixel hits. The specific sequence of iterations has been modified several times to improve the computing and physics performance of CMS tracking \cite{Tracking2012}. Table \ref{tab:tracking} lists the sequence used during the 2012 run. The selection criteria for each iteration are given, including cuts on \pt, transverse impact parameter $d_0$, and longitudinal impact parameter $z_0$. Some of the $z_0$ cuts are in terms of $\sigma$, the length of the beam spot in the $z$-direction used as a Gaussian standard deviation.

\begin{table}[htb]
  \begin{center}
    \begin{tabular}{llllll}
\hline
step  & seed type & seed subdetectors & \pt $[\GeVcns]$ & $d_0$ [cm] & $|z_0|$ \\
\hline
0     & triplet   & pixel             & ${>}0.6$     & ${<}0.02$  & ${<}4.0\sigma$ \\
1     & triplet   & pixel             & ${>}0.2$     & ${<}0.02$  & ${<}4.0\sigma$ \\
2     & pair      & pixel             & ${>}0.6$     & ${<}0.015$ & ${<}0.09\cm$ \\
3     & triplet   & pixel             & ${>}0.3$     & ${<}1.5$   & ${<}2.5\sigma$ \\
4     & triplet   & pixel/TIB/TID/TEC & ${>}0.5$--0.6 & ${<}1.5$   & ${<}10.0\cm$ \\
5     & pair      & TIB/TID/TEC       & ${>}0.6$     & ${<}2.0$   & ${<}10.0\cm$ \\
6     & pair      & TOB/TEC           & ${>}0.6$     & ${<}2.0$   & ${<}30.0\cm$ \\
\hline
    \end{tabular}
    \caption{The sequence of tracking iterations used during the 2012 run, including information on the seeds and selection criteria used in each step \cite{Tracking2012}.}
    \label{tab:tracking}
  \end{center}
\end{table}

The reconstructed tracks are used to reconstruct the primary vertices from the event \cite{TrackingJINST}. This includes both the main hard scatter vertex and additional vertices from pileup collisions. First, selection requirements are imposed on the tracks, in order to consider only prompt tracks near the IP. The selection requirements include cuts on the significance of $d_0$, the number of pixel and strip hits in the track, and the normalized $\chi^2$ from the fit of the track trajectory. The tracks which pass the selection requirements are clustered together using their $z$-coordinates, determined at each track's closest approach to the beam spot. A deterministic annealing algorithm is used to perform this clustering, in which each track may have a different probability to be associated with each vertex. The algorithm uses analogues of statistical mechanics quantities, slowly reducing the ``temperature'' and minimizing the ``free energy'' during each temperature iteration.

Once the deterministic annealing algorithm produces a list of vertex candidates, an adaptive vertex fitter is applied to each vertex candidate. This fitter weights each track in the vertex based on the agreement between the track and vertex positions. Using those weights, it fits the parameters of the vertex, including the $(x,y,z)$ position, the covariance matrix, and the number of degrees of freedom, $n_{\text{dof}} = -3 + 2 \sum{w_i}$, where $w_i$ is the weight for the $i$th track. The variable $n_{\text{dof}}$ can be used to select vertices which correspond to actual proton-proton interactions, as it is closely related to the number of tracks compatible with the vertex.

%add 2012 performance (efficiency, fake rate, vtx resolution)? references only have 2011 performance...

\section{Particle Flow
\label{sec:particle-flow}}

The CMS experiment uses a technique called particle flow (PF) to combine information from all subdetectors in order to identify all stable particles in each event \cite{CMS-PAS-PFT-09-001}. As described at the beginning of this chapter and shown in Fig. \ref{fig:cms-slice}, each type of stable particle is expected to create signals in a certain subset of the subdetectors. The performance of the PF algorithm was validated using early CMS data \cite{CMS-PAS-PFT-10-002,CMS-PAS-PFT-10-003}, along with newer CMS data for more recent uses of the algorithm \cite{Beaudette:2014cea}, demonstrating significant improvement over simpler approaches. The reconstructed particles are known as PF candidates, which can be treated as input particles by the various high-level reconstruction algorithms. PF candidates can also be used to compute the isolation of a reconstructed object, by summing the energy of candidates topologically close to the selected object.

The RecHits from the local reconstruction process are used to create the basic elements for this technique: tracks and clusters. Charged-particle tracks are created from tracker RecHits using an iterative algorithm as described in Sec. \ref{sec:tracks}, and muon tracks are created from muon system RecHits. Calorimeter energy deposits are grouped into clusters by identifying seed hits as local energy maxima exceeding a certain threshold, and then adding neighboring hits with energy above subsystem-specific thresholds meant to eliminate photodetector noise. Further removal of noise from the calorimeters is performed by rejecting clusters with characteristics matching those expected from leading sources of noise. Tracks and clusters are associated together using a linking algorithm that determines if they were likely produced by the same particle. The algorithm considers a possible link between each algorithm based on the $\eta$-$\phi$ distance between a charged-particle track and a cluster, accounting for propagation in the magnetic field, or between two clusters. For links between a charged-particle track and a muon track, the $\chi^{2}$ value from a global fit is used as the link distance. Groups of elements are associated based on minimizing the link distance and are called ``blocks''.

PF reconstruction algorithms classify the blocks as different types of particles. When a block is classified as a certain type of particle, it is removed from the list of unclassified blocks. A global muon block is accepted as a PF muon if the momentum of the combined charged-particle and muon tracks agrees with the momentum of the charged-particle track alone. The energy expected to have been deposited by the PF muons in the calorimeters due to minimum ionization is subtracted from the clusters. The remaining charged-particle tracks are checked for compatibility with electrons, which tend to leave short tracks and radiate energy via bremsstrahlung. A Gaussian Sum Filter (GSF) is applied to the compatible tracks in order to identify spatially-matched ECAL clusters, and the combination of a track and one or more clusters is classified as a PF electron.

The remaining tracks are linked to clusters to form PF charged hadrons, if the total cluster energy is similar to but smaller than the total track momentum. More than one track can link to a given cluster, but for a given track, only the link to the closest cluster is kept. This reflects the coarser segmentation of the calorimeter system as compared to that of the tracker. In cases where the total cluster energy is significantly smaller than the total track momentum, additional PF muons may be found using tracks from the block, and some tracks may be classified as fake and removed from consideration. Finally, an excess of energy in the clusters, above the total track momentum, is assumed to come from neutral particles. The excess energy is typically classified as a PF photon. If the total excess cluster energy in a block is larger than the total ECAL energy in the cluster, a PF neutral hadron is created from the excess energy remaining after assigning the ECAL excess energy to a PF photon. The remaining clusters not linked to any tracks are used to create PF photons in ECAL and PF neutral hadrons in HCAL.

\section{Electrons}

The electron candidates reconstructed by the PF algorithm are considered to be ``tracker-driven'' \cite{CMS-PAS-EGM-10-004}. This approach is suitable for low-\pt electrons and electrons produced by jets. For higher-\pt electrons, an alternative ``ECAL-driven'' approach is used. ECAL clusters are grouped into superclusters to account for bremsstrahlung photons radiated by electrons as they traverse the tracker, as well as the spread of energy in the $\phi$-direction due to the magnetic field \cite{ElectronReco}. Similarly to the PF algorithm, these superclusters are matched with track seeds and a GSF is used to reconstruct the trajectory of the electron track. Using a mixture of Gaussians to select the electron track better accounts for the energy loss from bremsstrahlung, as compared to the standard CMS track finding procedure \cite{ElectronGSF}. The lists of tracker-driven and ECAL-driven electron candidates are compared to avoid double counting.

Quality cuts on various kinematic variables are applied to the GSF electron candidates to identify whether or not they are genuine electrons \cite{ElectronCutBased}. A set of cut values is called a working point, and multiple working points are defined based on the strictness of the cut values. The kinematic variables used in the quality cuts are described here, and the specific values of the quality cuts used in the dissertation will be listed in Sec. \ref{sec:blah}. The $\eta$ width of the supercluster, $\sigma_{i\eta i\eta}$, is taken from the covariance matrix of the $\eta$ positions of the included crystals compared to the seed cluster. A modified $\eta$ variable which accounts for the crystal spacing is used, and each crystal's contribution is weighted using the logarithm of the ratio of its energy to the seed cluster energy \cite{EgammaShowerShape}. The differences in position of the supercluster, $(\eta_{\text{sc}},\phi_{\text{sc}})$, and the extrapolated track, $(\eta_{\text{in}}^{\text{extrap}},\phi_{\text{in}}^{\text{extrap}})$, are defined as $|\Delta \eta_{\text{in}}| = |\eta_{\text{sc}} - \eta_{\text{in}}^{\text{extrap}}|$ and $|\Delta \phi_{\text{in}}| = |\phi_{\text{sc}} - \phi_{\text{in}}^{\text{extrap}}|$. The leakage energy $H$ in the HCAL tower located behind the ECAL seed cluster is compared to the energy of the seed cluster $E$ in the variable $H/E$. The transverse and longitudinal impact parameters of the vertex associated with the electron, $d_{0}^{\text{vtx}}$ and $z_{0}^{\text{vtx}}$, are used. The variable $|1/E - 1/p|$, comparing the electron energy and momentum, is also considered. Finally, the isolation is computed by summing the \pt of charged hadron (CH), neutral hadron (NH), and photon ($\gamma$) PF candidates within a cone of $\Delta R < 0.3$ of the electron candidate, including a correction for contributions from pileup based on the median energy per area $\rho$ and the effective area of the electron $A_{\text{eff}}$:
\begin{equation}
I^{\text{PF}}_{\Pe} = \sum_{\Delta R < 0.3}{\pt^{(\text{CH})}} + \text{max}\left( \sum_{\Delta R < 0.3}{\pt^{(\text{NH})}} + \sum_{\Delta R < 0.3}{\pt^{(\gamma)}} - \rho A_{\text{eff}}, 0 \right).
\end{equation}
The relative isolation $I^{\text{PF}}_{\Pe}/\pt^{(\Pe)}$, scaled by the \pt of the electron, is used for the quality cuts.

\section{Muons}

The PF muon candidates are used for muon reconstruction. In addition, two supplementary methods are used to reconstruct muons \cite{CMS-PAS-MUO-10-002}. The first method considers all charged-particle tracks from the tracker, above minimal \pt and $p$ cuts, and creates a tracker muon from any track whose extrapolated position matches a track segment in the muon system. This method is efficient for low-momentum muons. The second method produces global muons. This method starts with a standalone muon from a track segment in the muon system and finds a matching track from the tracker. A global muon fit is then performed using both the muon system and tracker tracks, which can provide better momentum resolution for high-\pt muons. The global and track muon candidates, along with any remaining standalone muons which were not matched to a charged-particle track, are combined into one collection in order to prevent double counting.

As with electrons, working points are defined based on sets of quality cuts with different strictness. These cuts can include minimum numbers of muon system hits and segments, as well as minimum numbers of pixel hits and overall tracker hits. The reduced $\chi^2$ from the global muon fit is considered. The distances between the primary vertex and the transverse and longitudinal impact parameters $d_0$ and $d_z$ of the tracker track are also used. The isolation is calculated using PF candidates within a cone of $\Delta R < 0.4$ of the muon:
\begin{equation}
I^{\text{PF}}_{\mu} = \sum_{\Delta R < 0.4}{\pt^{(\text{CH})}} + \text{max}\left( \sum_{\Delta R < 0.4}{\pt^{(\text{NH})}} + \sum_{\Delta R < 0.4}{\pt^{(\gamma)}} - \Delta\beta \sum_{\Delta R < 0.4}{\pt^{(\text{PU})}}, 0 \right).
\end{equation}
A $\Delta\beta$ pileup (PU) correction is applied using the \pt of PU particles, which are identified as charged PF candidates from a different vertex than the muon. The $\Delta\beta$ factor is assigned a value of 0.5 based on the expected ratio of charged to neutral particles in pileup \cite{CMS-PAS-PFT-10-002}. Again, cuts are made on the relative isolation $I^{\text{PF}}_{\mu}/\pt^{(\mu)}$. The selections on these quantities are intended to minimize the contribution from cosmic ray muons, muons from heavy flavor decays, and leakage from hadronic showers. They also ensure precise measurement of the muon \pt. Section \ref{sec:blah} will provide the specific cuts for the working points used in the dissertation.

\section{Jets}

Due to QCD confinement, strongly interacting particles, quarks and gluons, cannot exist in a bare state. They immediately form multiple color singlet bound states, hadrons, using the energy from the gluon field in a process called hadronization. These hadrons tend to be produced in a narrow spray, which is called a jet \cite{Salam:2009jx}. To reconstruct a jet, the component particles must be clustered together. CMS uses the anti-\kt algorithm \cite{Cacciari:2008gp} to perform this clustering. This algorithm is a specific case of a generalized iterative cone algorithm, using the following equations:
\begin{align}
d_{ij} &= \text{min}(p_{\text{T}i}^{2p},p_{\text{T}j}^{2p})\frac{\Delta R_{ij}^{2}}{R^{2}}, \\
\Delta R_{ij}^{2} &= (\eta_i - \eta_j)^2 + (\phi_i - \phi_j)^{2}, \\
d_{iB} &= p_{\text{T}i}^{2p}.
\end{align}
Two distance variables are defined: the distance between particles $i$ and $j$, $d_{ij}$, and the distance between particle $i$ and the beam, $d_{iB}$. These distances are weighted by the \pt of the particles as indicated, and $R$ is a size parameter. At each iteration, both distance variables are calculated for all particles. If the minimum distance is $d_{iB}$, particle $i$ is considered to be a fully-clustered jet and removed from the list of particles. Otherwise, particles $i$ and $j$ from the minimum $d_{ij}$ are grouped together in the particle list. The algorithm continues to iterate until the particle list is empty. The anti-\kt algorithm is a special case of these equations with parameter $p=-1$. It has many desirable properties, including infrared and collinear safety, and the creation of circular jets with radius $R$ as shown in Fig. \ref{fig:ak5-example}.

\begin{figure}[hbt]
\begin{center}
\includegraphics[width=0.95\textwidth]{figures/08021189v2-fig.pdf}
\caption{An example of jets reconstructed with the anti-\kt algorithm, demonstrating the characteristic circular shape \cite{Cacciari:2008gp}.}
\label{fig:ak5-example}
\end{center}
\end{figure}

CMS chooses the value $R = 0.5$ for the size parameter and uses PF candidates for clustering, producing PF jets \cite{CMS-PAS-PFT-09-001, CMS-PAS-PFT-10-002}. Alternative jet methods use CaloTowers and charged-particle tracks separately or together, but these do not perform as well as PF jets. Typically, 65\% of jet energy goes into charged particles, 25\% goes into photons, and 10\% goes into neutral hadrons. Using PF candidates takes advantage of the excellent energy and position resolution for charged particles and photons provided by the combination of the tracker and the calorimeters in the PF algorithm.

Once a jet is reconstructed, several important types of corrections are applied to its energy response \cite{CMS-JEC}. First, minimum bias events are used to estimate the contributions from electronic noise and pileup, and those are subtracted in the offset correction. Next, the energy response is made uniform in $\eta$ using the multiplicative relative correction, derived from dijet events. Finally, the multiplicative absolute correction is derived from \GZJ events, exploiting the precise energy resolution of the ECAL and the tracker, and applied to make the energy response uniform in \pt.

To ensure the quality of reconstructed jets used in data analysis, a set of variables is used for PF jet identification \cite{CMS-AN-2010-003}. These variables include: the fraction of neutral hadrons in the jet, the fraction of neutral EM particles, the fraction of charged hadrons, the fraction of charged EM particles, the number of constituents, and the multiplicity of charged particles. Multiple sets of cuts on these variables are defined as working points \cite{PFJetID}. The specific working point used in this dissertation will be given in Sec. \ref{sec:blah}.

\section{Taus
\label{sec:hpstau}}

Tau leptons decay into hadrons approximately 64.76\% of the time \cite{PDG}. These hadronic decays produce objects similar to jets, but typically narrower and more isolated. For this reason, PF jets are used as the basis for reconstructing hadronically decaying tau leptons (hadronic taus or $\tauh$s). The Hadron Plus Strips (HPS) algorithm is used to identify tau leptons \cite{TauPerfCMS,Calabria:1516071}. The vast majority of \tauh decays consist of a tau neutrino $\nu_{\tau}$, one or three charged hadrons $h^{-}$ that are either $\pi^{-}$ or $K^{-}$, and zero or more neutral hadrons $\pi^{0}$ that almost immediately decay to two photons. The HPS algorithm only considers the visible decay products, so the $\nu_{\tau}$ is ignored. Table \ref{tab:tauh-decay} lists the leading hadronic decays, including intermediate hadronic resonances when present.

\begin{table}[htb]
  \begin{center}
    \begin{tabular}{|l|l|l|l|}
\hline
Decay                                                       & Resonance   & Mass (\MeVccns) & Branching fraction (\%) \\
\hline
$\tau^{-} \rightarrow h^{-} \nu_{\tau}$                     &             &                 & 11.53\% \\
$\tau^{-} \rightarrow h^{-} \pi^{0} \nu_{\tau}$             & $\rho^{-}$  & 775             & 25.95\% \\
$\tau^{-} \rightarrow h^{-} \pi^{0} \pi^{0} \nu_{\tau}$     & $a_{1}^{-}$ & 1230            & 9.52\% \\
$\tau^{-} \rightarrow h^{-} h^{+} h^{-} \nu_{\tau}$         & $a_{1}^{-}$ & 1230            & 9.80\% \\
$\tau^{-} \rightarrow h^{-} h^{+} h^{-} \pi^{0} \nu_{\tau}$ &             &                 & 4.76\% \\
\hline
    \end{tabular}
    \caption{The leading hadronic decays of tau leptons, including branching fractions and intermediate hadronic resonances \cite{PDG}. The symbol $h^{-}$ can be either $\pi^{-}$ or $K^{-}$. }
    \label{tab:tauh-decay}
  \end{center}
\end{table}

The strips in the HPS algorithm consist of PF photon candidates. Starting with the most energetic photon in the PF jet, the strip is built using an iterative search for other photons within a range $0.20\times0.05$ in $\eta$-$\phi$ around the center of the strip. Each iteration accepts the most energetic photon found and then recalculates the four-momentum of the updated strip. The strip is complete once no remaining photons are found in the given window, and it is kept if it passes a minimum \pt cut. The procedure is repeated with any remaining ungrouped photons in the jet. The formation of strips from photons accounts for spreading of their energy due to the effect of the magnetic field on conversions in the tracker.

Using the constituents of the PF jet, all combinations of strips with one or three charged hadrons are tested, with the charged hadrons assumed to be pions. All strips and charged hadrons must be found within a cone of $\Delta R = \text{max}(\text{min}(0.10,3.0/\pt^{\tauh}),0.05)$, where $\pt^{\tauh}$ is the \pt of the \tauh candidate in \GeVns. Several decay mode topologies are included in the algorithm, with mass ranges enforced for candidates based on the expectation of hadronic resonances in decays \cite{CMS-AN-2014-008}. Figure \ref{fig:tau-modes} shows a simple diagram of the different topologies.
\begin{enumerate}
\item Single hadron, when no strips are found.
\item One hadron + one strip, when the $\pi^0$ decay creates one strip from two narrowly separated photons. The invariant mass of the \tauh candidate, $M_{\tauh}$, must be in the range $0.3 < M_{\tauh} < \text{max}(1.3,\text{min}(1.3\sqrt{\pt^{\tauh}/200},2.1))\GeV$.
\item One hadron + two strips, when the $\pi^0$ decay creates one strip from two narrowly separated photons. In this case, the invariant mass of the two strips combined, $M_{\text{strips}}$, must be in the range $50 < M_{\text{strips}} < 200\MeV$, and $M_{\tauh}$ must be in the range $0.4 < M_{\tauh} < \text{max}(1.2,\text{min}(1.2\sqrt{\pt^{\tauh}/200},2.0))\GeV$.
\item Three hadrons, which requires all three charged hadron candidates to originate from the same primary vertex and to have the appropriate electric charges. $M_{\tauh}$ must be in the range $0.8 < M_{\tauh} < 1.5\GeV$.
\end{enumerate}
If more than one combination of PF constituents passes these decay mode finding requirements, the combination with the highest $\pt^{\tauh}$ is selected.

\begin{figure}[hbt]
\begin{center}
\includegraphics[width=0.95\textwidth]{figures/DP2014_015-fig.pdf}
\caption{A simple diagram of the different hadronic tau decay mode topologies reconstructed by the HPS algorithm \cite{CMS-DP-2014-015}.}
\label{fig:tau-modes}
\end{center}
\end{figure}

Isolation is an important tool in discriminating between $\tauh$s and jets. The isolation variable is computed using charged hadron and photon PF candidates within a cone of $\Delta R < 0.5$ around the \tauh candidate. A $\Delta\beta$ PU correction is applied using PU particles within a cone of $\Delta R < 0.8$ which originate from a different vertex than the \tauh candidate, with the factor $\Delta\beta = 0.4576$ \cite{CMS-DP-2014-015}.
\begin{equation}
I^{\text{PF}}_{\tauh} = \sum_{\Delta R < 0.5}{\pt^{(\text{CH})}} + \text{max}\left( \sum_{\Delta R < 0.5}{\pt^{(\gamma)}} - \Delta\beta \sum_{\Delta R < 0.8}{\pt^{(\text{PU})}}, 0 \right).
\end{equation}
The HPS algorithm uses the absolute isolation $I^{\text{PF}}_{\tauh}$ for the different working point quality cuts. The isolation discrimination also requires that each track associated with the \tauh contain at least three hits in the tracker. In addition to isolation, it is necessary to discriminate against electrons and muons which are misidentified as $\tauh$s. The anti-electron discriminator uses a multivariate (MVA) approach, training boosted decision trees (BDTs) using numerous variables depending on different cases of \tauh. These cases include: the possible association of the primary charged hadron in the \tauh with a GSF track; the possible association of the \tauh with a GSF electron within a cone of $\Delta R < 0.3$; whether or not the \tauh includes strips; and whether the \tauh $\eta$ coordinate lies in the EB or EE range. Multiple trainings for the anti-electron MVA discriminator were performed and multiple working points are defined. The anti-muon discriminator uses a cut-based approach, with multiple sets of cuts and working points defined. The cuts include requirements to minimize the activity in the muon system in the direction of the \tauh and to veto MIP muons based on energy and momentum. The specific working points for each discriminator used in this dissertation will be described in Sec. \ref{sec:blah}.
%The \tauh candidate must not match any muon track segments or any hits in the two outermost muon stations within a cone of $\Delta R < 0.5$. In the single hadron decay mode, the charged hadron in the \tauh must have $(E_{\text{ECAL}} + E_{\text{ECAL}})/p_{\text{track}} > 0.2$.

\section{b-tagging
\label{sec:b-tagging}}

Bottom quarks are associated with many interesting physical signatures, including decays of top quarks and Higgs bosons, as well as many BSM theories, including supersymmetry, that privilege or otherwise relate to third-generation fermions. Jets generated by bottom quark hadronization, b-jets, can be identified using certain properties which set them apart from jets from lighter quarks or gluons \cite{BTV-12-001}. Bottom quarks form hadrons with relatively long lifetimes whose decay products tend to have high \pt. Numerous algorithms have been created to identify b-jets based on information from tracks or reconstructed secondary vertices from b-hadron decays. The most successful of these algorithms is the Combined Secondary Vertex (CSV) algorithm, which uses secondary vertices and adds information from the tracks. The CSV algorithm calculates a likelihood-based discriminator to separate b-jets and other jets. Loose, medium, and tight working points are defined for this discriminator, based on setting the probability of misidentifying a light quark or gluon jet as a b-jet to be 10\%, 1\%, and 0.1\%, respectively \cite{CMS-PAS-BTV-13-001}.

For a given jet, the CSV algorithm subjects the tracks within that jet to additional purity requirements, beyond those specified in Sec. \ref{sec:tracks}. The track must fall within a cone of $\Delta R < 0.3$ relative to the direction of the jet and the distance between the track and the jet at closest approach must be less than 0.07\cm. The track must include at least two pixel hits and at least eight total tracker hits, with $\chi^2/n_{\text{dof}} < 5$. The impact parameters between the track and the primary vertex must satisfy $d_0 < 0.2\cm$ and $d_z < 17\cm$. Finally, the track must have $\pt>1\GeVc$ and decay length less than 5\cm, where decay length is the distance between the primary vertex and the closest approach of the track and the jet. The significance of the track's impact parameter $S_{\text{ip}}$, defined as the value of the impact parameter divided by its uncertainty, is a powerful observable. The number of high-quality tracks in the jet is also used.

Secondary vertices are reconstructed using the adaptive vertex fitter, which was described in Sec. \ref{sec:tracks}. Candidates are rejected if ${\geq}65\%$ of their tracks are also associated with the primary vertex; if they are outside a cone $\Delta R < 0.5$ with respect to the jet direction; or if they are radially separated from the primary vertex by more than 2.5\cm with an invariant mass close to the $\cmsSymbolFace{K}^{0}$ mass. The CSV algorithm assigns jets into one of three categories based on the presence of a secondary vertex: real vertex, pseudo-vertex, and no vertex. In the pseudo-vertex case, the algorithm uses tracks with $S_{\text{ip}} > 2$ to create an effective secondary vertex when the adaptive vertex fitter fails. In the no vertex case, the algorithm defaults to use only the track-based variables described previously. Otherwise, the vertex variables used include:
\begin{itemize}
\item The significance of the flight distance between the secondary and primary vertices in the transverse plane.
\item The mass of the secondary vertex.
\item The number of tracks in the secondary vertex.
\item The ratio between the energy in the secondary vertex tracks and the energy in all the tracks in the jet.
\item The $\Delta\eta$ values between each the secondary vertex track and the jet.
\item The transverse impact parameter significance for the track that pushes the invariant mass of the secondary vertex above 1.5\GeVcc, the threshold for charm quarks. This is calculated by sorting the tracks by $S_{\text{ip}}$ and combining them one by one.
\end{itemize}
% input: [analysis.tex]
\chapter{Data Analysis
\label{ch:analysis}}

\section{Data Samples}

\subsection{Observed Data}

\subsection{Monte Carlo}

\section{Selection and Optimization}

\subsection{Object Identification}

\subsubsection{Muons}

\subsubsection{Electrons}

\subsubsection{Taus}

\subsubsection{Jets}

\subsection{Event Selection}

\subsubsection{Preselection}

\subsubsection{Main Selection}

\subsubsection{Final Selections}

\section{Background Estimations}

\subsection{Irreducible Background (ttbar)}

\subsection{Reducible Background (fake tau)}

\subsection{Reducible Background (QCD)}

\subsection{Other Backgrounds}

\section{Systematic Uncertainties}

\section{Results}
% input: [conclusions.tex]
\chapter{Conclusions
\label{ch:conclusions}}

\appendix
   \titleformat{\chapter}
      {\normalfont\large}{Appendix \thechapter:}{1em}{}
% input: [limits.tex]
\chapter{Full CLs Shape-Based Limits
\label{ch:limits}}

To set limits using the modified frequentist $\text{CL}_{s}$ procedure \cite{Read:CLs}, two hypotheses are defined. The first is the null or background-only hypothesis $H_{0}$ or $b$, and the second is the alternate or signal plus background hypothesis $H_{1}$ or $s+b$.

$\mathcal{P}(\theta; N_{H_{i}})$ is defined as the Poisson probability to observe $\theta$ events in data given the hypothesis $H_{i}$ which predicts $N_{H_{i}}$ events. This probability can be defined generally for the whole sample, but also per bin for a histogram of some quantity, e.g. \ST, and/or per channel.

To obtain this probability, it is necessary to integrate over all of the nuisance parameters:
\begin{equation}
\mathcal{P}(\theta; N_{H_{i}}) = \int \mbox{Poisson}(\theta; N_{H_{i}},\eta)f(\eta)d\eta
\end{equation}
where $f$ is the probability density function (PDF) for the nuisance parameter $\eta$.

With those definitions, the test statistic $\mathcal{Q}$ is written as a ratio of likelihoods for a basic counting experiment:
\begin{equation}
\mathcal{Q} = \frac{\mathcal{P}(\theta; N_{H_{1}})}{\mathcal{P}(\theta; N_{H_{0}})}
\end{equation}
Splitting into \ST bins and two channels (\etau, \mutau) gives:
\begin{equation}
\mathcal{Q} = \prod_{i=\etau,\,\mutau}\prod_{j=0}^{n_{\text{bin}}} \frac{\mathcal{P}_{i,j}(\theta; N_{H_{1}})}{\mathcal{P}_{i,j}(\theta; N_{H_{0}})}
\end{equation}
For simplicity of computation, another form of the test statistic can be defined using the log likelihood ratio:
\begin{equation}
q = -2 \ln \mathcal{Q}
\end{equation}

To evaluate the test statistic as a function of the number of observed events $\theta$, many simulated pseudo-experiments are performed. For each hypothesis, $\theta$ is varied according to the probability distribution of that hypothesis, and the value of $\mathcal{Q}$ (or $q$) is kept for each $\theta$ value. To get $\mathcal{Q}$ for the actual number of observed events, $\mathcal{Q}_{\text{obs}}$, the same procedure is followed using $\theta=N_{\text{obs}}$. The $\text{CL}_{s+b}$ and $\text{CL}_{b}$ variables correspond to the probability for $\mathcal{Q_{\text{obs}}}$ to be greater than the $\mathcal{Q}$ values obtained for the hypotheses $H_1$ and $H_0$, respectively. When using $q$ as the test statistic, the observed value should be smaller than the value for the hypothesis. A visual example of these variables is shown in Fig. \ref{fig:q}.
\begin{align}
\text{CL}_{s+b} &= \mathcal{P}(\mathcal{Q}_{H_{1}} \leq \mathcal{Q}_{\text{obs}}) = \mathcal{P}(q_{H_{1}} \geq q_{\text{obs}}) \\
\text{CL}_{b} &= \mathcal{P}(\mathcal{Q}_{H_{0}} \leq \mathcal{Q}_{\text{obs}}) = \mathcal{P}(q_{H_{0}} \geq q_{\text{obs}}) \\
\text{CL}_{s} &= \text{CL}_{s+b}/\text{CL}_{b}
\end{align}

\begin{figure}[hbt]
\begin{center}
\includegraphics[width=0.95\textwidth]{figures/g21013-fig1.pdf}
\caption{Comparison of the observed value (red line) to the probability densities for $H_{0}$ (background only, blue line) and $H_{1}$ (signal + background, brown line) as a function of the log likelihood ratio. Green area: $\text{CL}_{s+b}$, yellow area: $1-\text{CL}_{b}$. From \cite{Read:presentation}.}
\label{fig:q}
\end{center}
\end{figure}

To set a mass limit on the signal hypothesis, the calculation of $\text{CL}_{s}$ is repeated for different signal masses. Masses with $\text{CL}_{s} < 1 - \alpha$ are excluded at the $\alpha$ confidence level, typically 95\%.
% input: [displays.tex]
\chapter{Event Displays
\label{ch:displays}}
% input: [datasets.tex]
\chapter{Table of Monte Carlo Datasets
\label{ch:datasets}}
% input: [collaboration.tex]
\chapter{CMS Collaboration
\label{ch:collaboration}}

\renewcommand{\baselinestretch}{1}
%\begin{singlespace}
\newpage
\normalsize
\phantomsection
%*input: [mainthesis.bbl]
\providecommand{\href}[2]{#2}\begingroup\raggedright\begin{thebibliography}{10}%
\makeatletter
\providecommand{\hrefCMSnoop }[0]{\@secondoftwo}%
\makeatother
\providecommand{\doi}{\texttt{doi:}\begingroup \urlstyle{tt}\Url}

\bibitem{Griffiths}
D.~Griffiths, ``{Introduction to Elementary Particles}''.
\newblock Wiley-VCH, 2008.

\bibitem{Peskin}
M.~E. Peskin and D.~V. Schroeder, ``{An Introduction To Quantum Field
  Theory}''.
\newblock Westview Press, 1995.

\bibitem{MissMJ}
\hrefCMSnoop {} {{MissMJ} {et~al.}, ``Standard Model of Elementary
  Particles''.}
  \href{http://creativecommons.org/licenses/by/3.0/deed.en}{Creative Commons
  Attribution 3.0 Unported License}, July, 2014.
\newblock
  \url{http://en.wikipedia.org/wiki/File:Standard_Model_of_Elementary_Particles.svg}.

\bibitem{NewBoson}
\hrefCMSnoop {} {{ CMS} Collaboration, ``Observation of a new boson at a mass
  of 125 GeV with the {CMS} experiment at the {LHC}'',} \textit{ Phys. Lett. B}
  \textbf{ 716} (2012) 30,
  \href{http://dx.doi.org/http://dx.doi.org/10.1016/j.physletb.2012.08.021}{\doi{http://dx.doi.org/10.1016/j.physletb.2012.08.021}},
  \href{http://www.arXiv.org/abs/1207.7235}{\texttt{ arXiv:1207.7235}}.

\bibitem{Drexler}
\hrefCMSnoop {} {E.~Drexler, ``Elementary particle interactions in the Standard
  Model''.}
  \href{http://creativecommons.org/publicdomain/zero/1.0/deed.en}{Creative
  Commons CC0 1.0 Universal Public Domain Dedication}, May, 2014.
\newblock
  \url{http://en.wikipedia.org/wiki/File:Elementary_particle_interactions_in_the_Standard_Model.png}.

\bibitem{Morrissey20121}
\hrefCMSnoop {} {D.~E. Morrissey, T.~Plehn, and T.~M. Tait, ``Physics searches
  at the {LHC}'',} \textit{ Phys. Rept.} \textbf{ 515} (2012) 1,
  \href{http://dx.doi.org/10.1016/j.physrep.2012.02.007}{\doi{10.1016/j.physrep.2012.02.007}},
  \href{http://www.arXiv.org/abs/0912.3259}{\texttt{ arXiv:0912.3259}}.

\bibitem{BulletCluster}
D.~Clowe\hrefCMSnoop {} { {et~al.}, ``A Direct Empirical Proof of the Existence
  of Dark Matter'',} \textit{ ApJL} \textbf{ 648} (2006) L109,
  \href{http://dx.doi.org/10.1086/508162}{\doi{10.1086/508162}},
  \href{http://www.arXiv.org/abs/astro-ph/0608407}{\texttt{
  arXiv:astro-ph/0608407}}.

\bibitem{Supernova98}
A.~G. Riess\hrefCMSnoop {} { {et~al.}, ``Observational Evidence from Supernovae
  for an Accelerating Universe and a Cosmological Constant'',} \textit{ AJ}
  \textbf{ 116} (1998) 1009,
  \href{http://dx.doi.org/10.1086/300499}{\doi{10.1086/300499}},
  \href{http://www.arXiv.org/abs/astro-ph/9805201}{\texttt{
  arXiv:astro-ph/9805201}}.

\bibitem{Supernova99}
S.~Perlmutter\hrefCMSnoop {} { {et~al.}, ``Measurements of $\Omega$ and
  $\Lambda$ from 42 High-Redshift Supernovae'',} \textit{ ApJ} \textbf{ 517}
  (1999) 565, \href{http://dx.doi.org/10.1086/307221}{\doi{10.1086/307221}},
  \href{http://www.arXiv.org/abs/astro-ph/9812133}{\texttt{
  arXiv:astro-ph/9812133}}.

\bibitem{BICEP}
\hrefCMSnoop {} {{ BICEP2} Collaboration, ``Detection of $B$-Mode Polarization
  at Degree Angular Scales by BICEP2'',} \textit{ Phys. Rev. Lett.} \textbf{
  112} (Jun, 2014) 241101,
  \href{http://dx.doi.org/10.1103/PhysRevLett.112.241101}{\doi{10.1103/PhysRevLett.112.241101}},
  \href{http://www.arXiv.org/abs/1403.3985}{\texttt{ arXiv:1403.3985}}.

\bibitem{EvansSigGen}
\hrefCMSnoop {} {J.~A. Evans and Y.~Kats, ``LHC coverage of RPV MSSM with light
  stops'',} \textit{ JHEP} \textbf{ 4} (2013)
  \href{http://dx.doi.org/10.1007/JHEP04(2013)028}{\doi{10.1007/JHEP04(2013)028}},
  \href{http://www.arXiv.org/abs/1209.0764}{\texttt{ arXiv:1209.0764}}.

\bibitem{SU4}
\hrefCMSnoop {} {J.~C. Pati and A.~Salam, ``Lepton number as the fourth
  ``color'''',} \textit{ Phys. Rev. D} \textbf{ 10} (1974) 275,
  \href{http://dx.doi.org/10.1103/PhysRevD.10.275}{\doi{10.1103/PhysRevD.10.275}}.

\bibitem{GUT}
\hrefCMSnoop {} {H.~Georgi and S.~L. Glashow, ``Unity of All
  Elementary-Particle Forces'',} \textit{ Phys. Rev. Lett.} \textbf{ 32} (1974)
  438,
  \href{http://dx.doi.org/10.1103/PhysRevLett.32.438}{\doi{10.1103/PhysRevLett.32.438}}.

\bibitem{SUPERSTR}
\hrefCMSnoop {} {J.~L. Hewett and T.~G. Rizzo, ``{Low-Energy Phenomenology of
  Superstring Inspired $\text{E}_{6}$ Models}'',} \textit{ Phys. Rept.}
  \textbf{ 183} (1989) 193,
\href{http://dx.doi.org/10.1016/0370-1573(89)90071-9}{\doi{10.1016/0370-1573(89)90071-9}}.
%%CITATION = PRPLC,183,193;%%.

\bibitem{LQ3b}
\hrefCMSnoop {} {B.~Gripaios, ``{Composite leptoquarks at the LHC}'',} \textit{
  JHEP} \textbf{ 1002} (2010) 045,
  \href{http://dx.doi.org/10.1007/JHEP02(2010)045}{\doi{10.1007/JHEP02(2010)045}},
\href{http://www.arXiv.org/abs/0910.1789}{\texttt{ arXiv:0910.1789}}.
%%CITATION = ARXIV:0910.1789;%%.

\bibitem{TC3}
\hrefCMSnoop {} {E.~Eichten and K.~Lane, ``Dynamical breaking of weak
  interaction symmetries'',} \textit{ Phys. Lett. B} \textbf{ 90} (1980) 125,
  \href{http://dx.doi.org/10.1016/0370-2693(80)90065-9}{\doi{10.1016/0370-2693(80)90065-9}}.

\bibitem{BRW}
\hrefCMSnoop {} {W.~Buchm{\"u}ller, R.~R{\"u}ckl, and D.~Wyler, ``{Leptoquarks
  in lepton-quark collisions}'',} \textit{ Phys. Lett. B} \textbf{ 191} (1987)
  442,
  \href{http://dx.doi.org/10.1016/0370-2693(87)90637-X}{\doi{10.1016/0370-2693(87)90637-X}}.

\bibitem{BRWerr}
\hrefCMSnoop {} {W.~Buchm{\"u}ller, R.~R{\"u}ckl, and D.~Wyler, ``Erratum:
  Leptoquarks in lepton-quark collisions'',} \textit{ Phys. Lett. B} \textbf{
  448} (1999) 320,
  \href{http://dx.doi.org/10.1016/S0370-2693(99)00014-3}{\doi{10.1016/S0370-2693(99)00014-3}}.

\bibitem{ModelIndLQ}
\hrefCMSnoop {} {S.~Davidson, D.~Bailey, and B.~A. Campbell, ``Model
  independent constraints on leptoquarks from rare processes'',} \textit{ Z.
  Phys. C} \textbf{ 61} (1994) 613,
  \href{http://dx.doi.org/10.1007/BF01552629}{\doi{10.1007/BF01552629}},
  \href{http://www.arXiv.org/abs/hep-ph/9309310}{\texttt{
  arXiv:hep-ph/9309310}}.

\bibitem{Leurer:1993em}
\hrefCMSnoop {} {M.~Leurer, ``A comprehensive study of leptoquark bounds'',}
  \textit{ Phys. Rev. D} \textbf{ 49} (1994) 333,
  \href{http://dx.doi.org/10.1103/PhysRevD.49.333}{\doi{10.1103/PhysRevD.49.333}},
\href{http://www.arXiv.org/abs/hep-ph/9309266}{\texttt{ arXiv:hep-ph/9309266}}.
%%CITATION = HEP-PH/9309266;%%.

\bibitem{MuchAdo}
\hrefCMSnoop {} {J.~L. Hewett and T.~G. Rizzo, ``Much ado about leptoquarks: A
  comprehensive analysis'',} \textit{ Phys. Rev. D} \textbf{ 56} (Nov, 1997)
  5709,
  \href{http://dx.doi.org/10.1103/PhysRevD.56.5709}{\doi{10.1103/PhysRevD.56.5709}},
  \href{http://www.arXiv.org/abs/hep-ph/9703337}{\texttt{
  arXiv:hep-ph/9703337}}.

\bibitem{LQreview}
\hrefCMSnoop {} {D.~E. Acosta and S.~K. Blessing, ``LEPTOQUARK SEARCHES AT HERA
  AND THE TEVATRON'',} \textit{ Annu. Rev. Nucl. Part. Sci.} \textbf{ 49}
  (1999) 389,
  \href{http://dx.doi.org/10.1146/annurev.nucl.49.1.389}{\doi{10.1146/annurev.nucl.49.1.389}}.

\bibitem{LQPairHad}
\hrefCMSnoop {} {J.~Bl{\"u}mlein, E.~Boos, and A.~Kryukov, ``Leptoquark pair
  production in hadronic interactions'',} \textit{ Z. Phys. C} \textbf{ 76}
  (1997) 137,
  \href{http://dx.doi.org/10.1007/s002880050538}{\doi{10.1007/s002880050538}},
  \href{http://www.arXiv.org/abs/hep-ph/9610408}{\texttt{
  arXiv:hep-ph/9610408}}.

\bibitem{CTEQ6r1}
J.~Pumplin\hrefCMSnoop {} { {et~al.}, ``New Generation of Parton Distributions
  with Uncertainties from Global QCD Analysis'',} \textit{ JHEP} \textbf{ 10}
  (2002) 012,
  \href{http://dx.doi.org/10.1088/1126-6708/2002/07/012}{\doi{10.1088/1126-6708/2002/07/012}},
  \href{http://www.arXiv.org/abs/hep-ph/0201195}{\texttt{
  arXiv:hep-ph/0201195}}.

\bibitem{CTEQ6r2}
D.~Stump\hrefCMSnoop {} { {et~al.}, ``Inclusive jet production, parton
  distributions, and the search for new physics'',} \textit{ JHEP} \textbf{ 7}
  (2003) 046,
  \href{http://dx.doi.org/10.1088/1126-6708/2003/10/046}{\doi{10.1088/1126-6708/2003/10/046}},
  \href{http://www.arXiv.org/abs/hep-ph/0303013}{\texttt{
  arXiv:hep-ph/0303013}}.

\bibitem{LQxsec}
\hrefCMSnoop {} {M.~Kr{\"a}mer, T.~Plehn, M.~Spira, and P.~M. Zerwas, ``Pair
  production of scalar leptoquarks at the CERN LHC'',} \textit{ Phys. Rev. D}
  \textbf{ 71} (Mar, 2005) 057503,
  \href{http://dx.doi.org/10.1103/PhysRevD.71.057503}{\doi{10.1103/PhysRevD.71.057503}},
  \href{http://www.arXiv.org/abs/hep-ph/0411038}{\texttt{
  arXiv:hep-ph/0411038}}.

\bibitem{CMSLQ3}
\hrefCMSnoop {} {{CMS Collaboration}, ``Search for Pair Production of
  Third-Generation Leptoquarks and Top Squarks in $pp$ Collisions at
  $\sqrt{s}=7$~TeV'',} \textit{ Phys. Rev. Lett.} \textbf{ 110} (2013) 081801,
  \href{http://dx.doi.org/10.1103/PhysRevLett.110.081801}{\doi{10.1103/PhysRevLett.110.081801}},
  \href{http://www.arXiv.org/abs/1210.5629}{\texttt{ arXiv:1210.5629}}.

\bibitem{ATLASLQ3}
\hrefCMSnoop {} {{ATLAS Collaboration}, ``Search for third generation scalar
  leptoquarks in $pp$ collisions at $\sqrt{s}=7$~TeV with the ATLAS
  detector'',} \textit{ JHEP} \textbf{ 06} (2013) 033,
  \href{http://dx.doi.org/10.1007/JHEP06(2013)033}{\doi{10.1007/JHEP06(2013)033}},
  \href{http://www.arXiv.org/abs/1303.0526}{\texttt{ arXiv:1303.0526}}.

\bibitem{LHCmachine}
\hrefCMSnoop {} {L.~Evans and P.~Bryant, ``{LHC Machine}'',} \textit{ JINST}
  \textbf{ 3} (2008) S08001,
\href{http://dx.doi.org/10.1088/1748-0221/3/08/S08001}{\doi{10.1088/1748-0221/3/08/S08001}}.
%%CITATION = JINST,3,S08001;%%.

\bibitem{CMSJINST}
\hrefCMSnoop {} {{ CMS} Collaboration, ``{The CMS experiment at the CERN
  LHC}'',} \textit{ JINST} \textbf{ 3} (2008) S08004,
\href{http://dx.doi.org/10.1088/1748-0221/3/08/S08004}{\doi{10.1088/1748-0221/3/08/S08004}}.
%%CITATION = JINST,3,S08004;%%.

\bibitem{Jean-Luc:841573}
\href {http://cds.cern.ch/record/841573} {J.-L. Caron, ``{LHC Layout}'',}
  (September, 1997). AC Collection. Legacy of AC. Pictures from 1992 to 2002.

\bibitem{Jean-Luc:841568}
\href {http://cds.cern.ch/record/841568} {J.-L. Caron, ``{The LHC injection
  complex}'',} (May, 1993). AC Collection. Legacy of AC. Pictures from 1992 to
  2002.

\bibitem{Dailler:842253}
\href {http://cds.cern.ch/record/842253} {S.~Dailler, ``{LHC Dipole}'',} (July,
  1998). AC Collection. Legacy of AC. Pictures from 1992 to 2002.

\bibitem{LumiPublic}
\href {https://twiki.cern.ch/twiki/bin/view/CMSPublic/LumiPublicResults} {{
  CMS} Collaboration, ``{Public CMS Luminosity Information}'',} (Jan, 2014).

\bibitem{Veszpremi:2014hpa}
\hrefCMSnoop {} {{ CMS} Collaboration, ``{Operation and performance of the CMS
  tracker}'',} \textit{ JINST} \textbf{ 9} (2014) C03005,
  \href{http://dx.doi.org/10.1088/1748-0221/9/03/C03005}{\doi{10.1088/1748-0221/9/03/C03005}},
\href{http://www.arXiv.org/abs/1402.0675}{\texttt{ arXiv:1402.0675}}.
%%CITATION = ARXIV:1402.0675;%%.

\bibitem{CMS:2013ecal}
\href {https://cds.cern.ch/record/1528235} {{ CMS} Collaboration, ``{2012 ECAL
  detector performance plots}'',} CMS Detector Performance Summary
  CMS-DP-2013-007, CERN, 2013.

\bibitem{Brooke:2013hnf}
\hrefCMSnoop {} {{ CMS} Collaboration, ``{Performance of the CMS Level-1
  Trigger}'',} \textit{ PoS} \textbf{ ICHEP2012} (2013) 508,
\href{http://www.arXiv.org/abs/1302.2469}{\texttt{ arXiv:1302.2469}}.
%%CITATION = ARXIV:1302.2469;%%.

\bibitem{Trocino:2014jya}
\hrefCMSnoop {} {D.~Trocino, ``{The CMS High Level Trigger}'',} \textit{ J.
  Phys. Conf.} \textbf{ 513} (2014) 012036,
\href{http://dx.doi.org/10.1088/1742-6596/513/1/012036}{\doi{10.1088/1742-6596/513/1/012036}}.
%%CITATION = 00462,513,012036;%%.

\bibitem{CMS-PAS-LUM-12-001}
\href {https://cds.cern.ch/record/1482193} {{ CMS} Collaboration, ``{CMS
  Luminosity Based on Pixel Cluster Counting - Summer 2012 Update}'',} CMS
  Physics Analysis Summary CMS-PAS-LUM-12-001, CERN, Geneva, 2012.

\bibitem{CMS-PAS-LUM-13-001}
\href {https://cds.cern.ch/record/1598864} {{ CMS} Collaboration, ``{CMS
  Luminosity Based on Pixel Cluster Counting - Summer 2013 Update}'',} CMS
  Physics Analysis Summary CMS-PAS-LUM-13-001, CERN, Geneva, 2013.

\bibitem{CMS-slice}
\href {https://cms-docdb.cern.ch/cgi-bin/PublicDocDB/ShowDocument?docid=4172}
  {D.~Barney and E.~Quigg, ``{Interactive Slice of the CMS detector}'',} (July,
  2010).

\bibitem{TDR-software}
\href {https://cds.cern.ch/record/922757} {{ CMS} Collaboration, ``{CMS
  Physics: Technical Design Report Volume 1: Detector Performance and
  Software}'',} Technical Report CMS-TDR-8-1, CERN, Geneva, 2006.

\bibitem{MSTW09}
\hrefCMSnoop {} {A.~Martin, W.~Stirling, R.~Thorne, and G.~Watt, ``Parton
  distributions for the LHC'',} \textit{ Eur. Phys. J. C} \textbf{ 63} (2009)
  189,
  \href{http://dx.doi.org/10.1140/epjc/s10052-009-1072-5}{\doi{10.1140/epjc/s10052-009-1072-5}},
  \href{http://www.arXiv.org/abs/0901.0002}{\texttt{ arXiv:0901.0002}}.

\bibitem{QuarkGluon}
\hrefCMSnoop {} {J.~M. Campbell, J.~W. Huston, and W.~J. Stirling, ``Hard
  interactions of quarks and gluons: a primer for LHC physics'',} \textit{ Rep.
  Prog. Phys.} \textbf{ 70} (2007), no.~1, 89,
  \href{http://dx.doi.org/10.1088/0034-4885/70/1/R02}{\doi{10.1088/0034-4885/70/1/R02}}.

\bibitem{Sjostrand:2006za}
\hrefCMSnoop {} {T.~Sj{\"o}strand, S.~Mrenna, and P.~Skands, ``{PYTHIA} 6.4
  physics and manual'',} \textit{ JHEP} \textbf{ 05} (2006) 026,
  \href{http://dx.doi.org/10.1088/1126-6708/2006/05/026}{\doi{10.1088/1126-6708/2006/05/026}},
\href{http://www.arXiv.org/abs/hep-ph/0603175}{\texttt{ arXiv:hep-ph/0603175}}.
%%CITATION = HEP-PH/0603175;%%.

\bibitem{MadGraph}
J.~Alwall\hrefCMSnoop {} { {et~al.}, ``The automated computation of tree-level
  and next-to-leading order differential cross sections, and their matching to
  parton shower simulations'',} (2014).
\href{http://www.arXiv.org/abs/1405.0301}{\texttt{ arXiv:1405.0301}}.
%%CITATION = ARXIV:1405.0301;%%.

\bibitem{Alioli:2010xd}
\hrefCMSnoop {} {S.~Alioli, P.~Nason, C.~Oleari, and E.~Re, ``{A general
  framework for implementing NLO calculations in shower Monte Carlo programs:
  the POWHEG BOX}'',} \textit{ JHEP} \textbf{ 1006} (2010) 043,
  \href{http://dx.doi.org/10.1007/JHEP06(2010)043}{\doi{10.1007/JHEP06(2010)043}},
\href{http://www.arXiv.org/abs/1002.2581}{\texttt{ arXiv:1002.2581}}.
%%CITATION = ARXIV:1002.2581;%%.

\bibitem{TAUOLA}
\hrefCMSnoop {} {Z.~W\c{a}s, ``{TAUOLA} the library for $\tau$ lepton decay,
  and {KKMC/KORALB/KORALZ}\ldots status report'',} \textit{ Nucl. Phys. B,
  Proc. Suppl.} \textbf{ 98} (2001) 96,
  \href{http://dx.doi.org/10.1016/S0920-5632(01)01200-2}{\doi{10.1016/S0920-5632(01)01200-2}}.

\bibitem{geant4nim}
S.~Agostinelli\hrefCMSnoop {} { {et~al.}, ``Geant4 a simulation toolkit'',}
  \textit{ Nucl. Instrum. Meth. A} \textbf{ 506} (2003), no.~3, 250 -- 303,
  \href{http://dx.doi.org/10.1016/S0168-9002(03)01368-8}{\doi{10.1016/S0168-9002(03)01368-8}}.

\bibitem{geant4ieee}
J.~Allison\hrefCMSnoop {} { {et~al.}, ``Geant4 developments and
  applications'',} \textit{ {IEEE} Trans. Nucl. Sci.} \textbf{ 53} (February,
  2006) 270 --278,
  \href{http://dx.doi.org/10.1109/TNS.2006.869826}{\doi{10.1109/TNS.2006.869826}}.

\bibitem{TrackingJINST}
\hrefCMSnoop {} {{ CMS} Collaboration, ``{Description and performance of track
  and primary-vertex reconstruction with the CMS tracker}'',} (May, 2014).
  \href{http://www.arXiv.org/abs/1405.6569}{\texttt{ arXiv:1405.6569}}.
  Submitted to \emph{JINST}.

\bibitem{Tracking2012}
\hrefCMSnoop {} {D.~Giordano and G.~Sguazzoni, ``CMS reconstruction
  improvements for the tracking in large pile-up events'',} \textit{ J. Phys.
  Conf.} \textbf{ 396} (2012), no.~2, 022044,
  \href{http://dx.doi.org/10.1088/1742-6596/396/2/022044}{\doi{10.1088/1742-6596/396/2/022044}}.

\bibitem{CMS-PAS-PFT-09-001}
\href {http://cds.cern.ch/record/1194487} {{ CMS} Collaboration,
  ``Particle-Flow Event Reconstruction in CMS and Performance for Jets, Taus,
  and MET'',} CMS Physics Analysis Summary CMS-PAS-PFT-09-001, CERN, 2009.

\bibitem{CMS-PAS-PFT-10-002}
\href {http://cdsweb.cern.ch/record/1279341} {{ CMS} Collaboration,
  ``Commissioning of the Particle-Flow Reconstruction in Minimum-Bias and Jet
  Events from {$\Pp\Pp$} Collisions at 7 {TeV}'',} CMS Physics Analysis Summary
  CMS-PAS-PFT-10-002, CERN, 2010.

\bibitem{CMS-PAS-PFT-10-003}
\href {http://cdsweb.cern.ch/record/1279347} {{ CMS} Collaboration,
  ``Commissioning of the particle-flow event reconstruction with leptons from
  {J}/$\Psi$ and {W} decays at 7 {TeV}'',} CMS Physics Analysis Summary
  CMS-PAS-PFT-10-003, CERN, 2010.

\bibitem{Beaudette:2014cea}
\hrefCMSnoop {} {F.~Beaudette, ``{The CMS Particle Flow Algorithm}'',} in
  \textit{ {Proceedings, International Conference on Calorimetry for the High
  Energy Frontier (CHEF 2013)}}, p.~295.
\newblock {\'{E}}cole Polytechnique, Paris, 2013.
\newblock
\href{http://www.arXiv.org/abs/1401.8155}{\texttt{ arXiv:1401.8155}}.
\newblock
%%CITATION = ARXIV:1401.8155;%%.

\bibitem{CMS-PAS-EGM-10-004}
\href {http://cdsweb.cern.ch/record/1299116} {{ CMS} Collaboration, ``{Electron
  reconstruction and identification at sqrt(s) = 7 TeV}'',} CMS Physics
  Analysis Summary CMS-PAS-EGM-10-004, CERN, Geneva, 2010.

\bibitem{ElectronReco}
S.~Baffioni\hrefCMSnoop {} { {et~al.}, ``Electron reconstruction in CMS'',}
  \textit{ Eur. Phys. J. C} \textbf{ 49} (2007), no.~4, 1099,
  \href{http://dx.doi.org/10.1140/epjc/s10052-006-0175-5}{\doi{10.1140/epjc/s10052-006-0175-5}}.

\bibitem{ElectronGSF}
\hrefCMSnoop {} {W.~Adam, R.~Fr�hwirth, A.~Strandlie, and T.~Todorov,
  ``Reconstruction of electrons with the Gaussian-sum filter in the CMS tracker
  at the LHC'',} \textit{ J. Phys. G} \textbf{ 31} (2005) N9,
  \href{http://dx.doi.org/10.1088/0954-3899/31/9/N01}{\doi{10.1088/0954-3899/31/9/N01}}.

\bibitem{ElectronCutBased}
\href
  {https://twiki.cern.ch/twiki/bin/viewauth/CMS/EgammaCutBasedIdentification}
  {{ CMS} Collaboration, ``{Cut Based Electron ID}'',} (2013).

\bibitem{EgammaShowerShape}
\href {https://twiki.cern.ch/twiki/bin/view/CMSPublic/SWGuideEgammaShowerShape}
  {{ CMS} Collaboration, ``{SW Guide Egamma Shower Shape}'',} (2011).

\bibitem{CMS-PAS-MUO-10-002}
\href {https://cds.cern.ch/record/1279140/} {{ CMS} Collaboration,
  ``{Performance of muon identification in pp collisions at $\sqrt{s}$ = 7
  TeV}'',} CMS Physics Analysis Summary CMS-PAS-MUO-10-002, CERN, Geneva, 2010.

\bibitem{Salam:2009jx}
\hrefCMSnoop {} {G.~P. Salam, ``{Towards jetography}'',} \textit{ Eur. Phys. J.
  C} \textbf{ 67} (2010) 637,
  \href{http://dx.doi.org/10.1140/epjc/s10052-010-1314-6}{\doi{10.1140/epjc/s10052-010-1314-6}},
\href{http://www.arXiv.org/abs/0906.1833}{\texttt{ arXiv:0906.1833}}.
%%CITATION = 0906.1833;%%.

\bibitem{Cacciari:2008gp}
\hrefCMSnoop {} {M.~Cacciari, G.~P. Salam, and G.~Soyez, ``The anti-$k_t$ jet
  clustering algorithm'',} \textit{ JHEP} \textbf{ 04} (2008) 063,
  \href{http://dx.doi.org/10.1088/1126-6708/2008/04/063}{\doi{10.1088/1126-6708/2008/04/063}},
  \href{http://www.arXiv.org/abs/0802.1189}{\texttt{ arXiv:0802.1189}}.

\bibitem{CMS-JEC}
\hrefCMSnoop {} {{ CMS} Collaboration, ``Determination of jet energy
  calibration and transverse momentum resolution in CMS'',} \textit{ JINST}
  \textbf{ 6} (2011) 11002,
  \href{http://dx.doi.org/10.1088/1748-0221/6/11/P11002}{\doi{10.1088/1748-0221/6/11/P11002}},
  \href{http://www.arXiv.org/abs/1107.4277}{\texttt{ arXiv:1107.4277}}.

\bibitem{CMS-AN-2010-003}
\hrefCMSnoop {} {N.~Saoulidou, ``Particle Flow Jet Identification Criteria'',}
  CMS Analysis Note CMS-AN-2010-003, CERN, Geneva, Jun, 2010.

\bibitem{PFJetID}
\href {https://twiki.cern.ch/twiki/bin/viewauth/CMS/JetID} {{ CMS}
  Collaboration, ``Jet Identification'',} (2012).

\bibitem{PDG}
\href {http://pdg.lbl.gov} {K.~A. Olive {et~al.}, ``The Review of Particle
  Physics'',} \textit{ Chin. Phys. C} \textbf{ 38} (2014) 090001.

\bibitem{TauPerfCMS}
\hrefCMSnoop {} {{ CMS} Collaboration, ``Performance of t-lepton reconstruction
  and identification in CMS'',} \textit{ JINST} \textbf{ 7} (2012) P01001,
  \href{http://dx.doi.org/10.1088/1748-0221/7/01/P01001}{\doi{10.1088/1748-0221/7/01/P01001}}.

\bibitem{Calabria:1516071}
\href {https://cds.cern.ch/record/1516071} {C.~Calabria, ``{Tau trigger,
  reconstruction and identification at CMS}'',} CMS Conference Report
  CMS-CR-2012-344, CERN, Geneva, Nov, 2012.

\bibitem{CMS-AN-2014-008}
\hrefCMSnoop {} {{The Tau Physics Object Group}, ``Performance of tau
  reconstruction and identification in pp collisions at $\sqrt{s}$ = 8 TeV'',}
  CMS Analysis Note CMS-AN-2014-008, CERN, Geneva, Jun, 2014.

\bibitem{CMS-DP-2014-015}
\href {https://cds.cern.ch/record/1704439} {{ CMS} Collaboration, ``{Tau ID
  Performance Plots}'',} CMS Detector Performance Summary CMS-DP-2014-015,
  CERN, Geneva, Apr, 2014.

\bibitem{BTV-12-001}
\hrefCMSnoop {} {{ CMS} Collaboration, ``{Identification of b-quark jets with
  the CMS experiment}'',} \textit{ JINST} \textbf{ 8} (2013) P04013,
  \href{http://dx.doi.org/10.1088/1748-0221/8/04/P04013}{\doi{10.1088/1748-0221/8/04/P04013}},
\href{http://www.arXiv.org/abs/1211.4462}{\texttt{ arXiv:1211.4462}}.
%%CITATION = ARXIV:1211.4462;%%.

\bibitem{CMS-PAS-BTV-13-001}
\href {http://cds.cern.ch/record/1581306/} {{ CMS} Collaboration,
  ``{Performance of b tagging at $\sqrt{s} =$ 8 TeV in multijet, ttbar and
  boosted topology events}'',} CMS Physics Analysis Summary CMS-PAS-BTV-13-001,
  CERN, Geneva, 2013.

\bibitem{Read:CLs}
\href {http://cdsweb.cern.ch/record/451614} {A.~L. Read, ``Modified frequentist
  analysis of search results (the CLs method)'',} {CERN} Report
  CERN-OPEN-2000-005, CERN, 2000.

\bibitem{Read:presentation}
\hrefCMSnoop {} {A.~L. Read, ``{Presentation of search results: the $CL_{s}$
  technique}'',} \textit{ J. Phys. G} \textbf{ 28} (2002) 2693,
  \href{http://dx.doi.org/10.1088/0954-3899/28/10/313}{\doi{10.1088/0954-3899/28/10/313}}.

\end{thebibliography}\endgroup
%FLATEX-REM:\bibliographystyle{lucas_unsrt} 
%FLATEX-REM:\bibliography{mybib}
%\end{singlespace}

\end{document}
