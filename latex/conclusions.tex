\chapter{Conclusions
\label{ch:conclusions}}

This dissertation has presented a search for pair production of third-generation scalar leptoquarks with each leptoquark decaying to a tau lepton and a bottom quark. The search used 19.7\fbinv of proton-proton collision data collected with the Compact Muon Solenoid experiment during the 2012 run of the Large Hadron Collider at a center-of-mass energy of $\sqrt{s}=8\TeV$. The existence of these leptoquarks is excluded at the 95\% confidence level for masses up to 740\GeV. This mass limit applies directly to pair production of top squarks decaying through the R-parity violating coupling $\lambda^{\prime}_{333}$, which has the same final-state signature and kinematic distributions as the third-generation scalar leptoquarks. This limit is a significant improvement over the previous limit of 530\GeV obtained using 7\TeV data \cite{CMSLQ3,ATLASLQ3}. Limits are also set for varying leptoquark branching fraction, with the area of low branching fraction constrained by a reinterpretation of a search for top squarks decaying to a top quark and a neutralino \cite{SUS-13-011}. 

The search is extended to cover top squarks undergoing a chargino-mediated decay involving the R-parity violating coupling $\lambda^{\prime}_{3jk}$, in which each top squark decays to a final state including a tau lepton, a bottom quark, and two light quarks. Top squarks undergoing this decay are excluded at the 95\% confidence level in the mass range 200--580\GeV. This is the first direct search for the top squark decay involving the coupling $\lambda^{\prime}_{3jk}$.

In 2015, Run 2 of the LHC will begin at approximately the design center-of-mass energy $\sqrt{s}=13\text{--}14\TeV$. This increase in energy corresponds to an order-of-magnitude increase in the pair production cross section for leptoquarks at high masses. The cross section for $\MLQ=1000\GeV$ will increase from $4.01\times10^{-4}\unit{pb}$ at $\sqrt{s}=8\TeV$ to $8.36\times10^{-3}\unit{pb}$ \cite{LQxsec}. With this significant increase in the cross section, the exclusion of leptoquarks at the \TeVns scale will be in reach with only a moderate amount of data \cite{LQPairHad}. Additionally, searches for single production of leptoquarks will become feasible, as the limits on the leptoquark Yukawa coupling only extend to the \TeVns scale \cite{Leurer:1993em, MuchAdo, LQreview}.

The searches for R-parity violating supersymmetry were motivated by the existing limits on R-parity conserving supersymmetry from searches requiring large missing transverse energy. The limits set in these searches, which are the most stringent to date for the selected couplings, similarly approach the high edge of the conditions for naturalness \cite{NaturalSUSY}. However, supersymmetry is not fully excluded yet; significant regions of the parameter space remain unexamined. Run 2 of the LHC will have a high potential for either the discovery or more complete exclusion of supersymmetry \cite{CMS:2013xfa}.